% ========== SYMMETRY SECTION ==========

\subsection{Lie groups}

Lie groups, and their corresponding Lie algebras, are used to define the
symmetries obeyed by a given theory. Lie groups represent a set of
transformations that can be applied to a state, and are used to represent
symmetries in theories when transformations between these states should be
invariant.  Two types of group are prominent in particle physics theory: unitary
ans special unitary groups.  A unitary group of degree $n$, denoted $U(n)$, is
the infinite group of all unitary $n\times n$ matrices under matrix
multiplication.  A special unitary group of degree $n$, $SU(n)$, is a subgroup
of the corresponding unitary group and contains all $n\times n$ matrices with a
determinant of 1. An $SU(n)$ group has $n^2-1$ members, or `generators'.

Of interest to the theories discussed here are the groups $U(1)$, $SU(2)$, and
$SU(3)$. The $U(1)$ group contains all complex numbers with a magnitude of 1; a
$U(1)$ transformation is equivalent to a change in complex phase. As complex
numbers commute, $U(1)$ forms an Abelian group.

The $SU(2)$ group contains three $2\times 2$ matrices, $T^a$, which may be
expressed in terms of the Pauli spin matrices as $T^a = \sigma^a/2$. The
generators of $SU(2)$ are non-commutative, with the commutator
\begin{equation*}
  [T^a,T^b] = i \varepsilon^{abc} T^c,
\end{equation*}
where $\varepsilon^{abc}$ is the totally antisymmetric Levi-Civita tensor;
the $SU(2)$ group is therefore non-Abelian.

More generally, the commutator for generators of an $SU(n)$ algebra are given by
\begin{equation}
  [T^a,T^b] = if^{abc}T^{c},
  \label{eqn:theory-symmetry-gen-commutator}
\end{equation}
where $f^{abc}$ is a totally antisymmetric tensor specifying the structure
constants of the Lie algebra. For the $SU(2)$ definition given above,
$f^{abc}=\varepsilon^{abc}$.

The third group of interest is $SU(3)$; this group is also non-Abelian and
contains 8 generators, which in this instance are $3\times 3$ matrices. A
typical basis for the generators of $SU(3)$ would give structure constant values
\begin{equation*}
  \begin{split}
    f^{123} &= 1, \\
    f^{147} = f^{246} = f^{257} = f^{345} = -f^{367} = -f^{156} &= \frac12,
    \hspace{10em}\\
    f^{458} = f^{678} &= \frac{\sqrt{3}}2, \\
  \end{split}
\end{equation*}
%
with all other elements equal to zero \cite{gellmann1961}.

\subsection{Gauge transformations}

A gauge, in theoretical particle physics, is an abstract frame of reference that
often manifests as a mathematical simplification made by modifying fields or
operators, known as a gauge transformation. These transformations lead to a
deeper symmetry required in theories: if a gauge transformation is allowed
within a theory then it must not affect the observables of the theory, the
theory must be symmetric under these transformations. This gauge symmetry is a
core concept of modern \acp{QFT}, and such gauge-symmetric \acp{QFT} are known
as gauge theories.

A gauge transformation can local or global. A global gauge transformation is
where the parameter controlling the transformation is constant across
space-time. Local transformations are a more general case where the parameter is
a function of spatial coordinates and can vary between locations.

The transformations between allowed gauges in a theory form a Lie group. For
each generator in the Lie group a gauge field is introduced. Inclusion of these
gauge fields in the Lagrangian ensure that the theory is invariant under gauge
transformations. If a theory is local gauge invariant, these gauge fields can
vary across space-time and allow for interactions between particles in the
theory; this is how forces are introduced in gauge theories. The quanta of a
gauge field in a \ac{QFT} is called a gauge boson.

For an example of constructing a \ac{QFT} satisfying local gauge invariance see
Reference \cite[pp.242-3]{Thomson2013}.

% Work out difference between global and local gauge
