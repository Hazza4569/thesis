% ========== MONTE CARLO PREDICTIONS ==========

The \ac{SM} is tested by comparing its predicted cross-sections for a set of
measurable physics processes with the rate observed in data. To isolate
processes of interest, this is often done in complicated phase spaces and
differentially across distributions, so calculating the predicted cross-sections
is quite complex.

\acused{MC}
The solution to this problem is to generate Monte Carlo (MC) events representing
the \ac{SM} prediction. For a given final state (i.e. the set of particles
produced in the interaction) a sample of events is generated for each
contributing process, containing particles with random kinematic properties
generated in such a way that the overall distribution matches the expectation
from the model. If all processes are accounted for, taking the sum of events
from all of these samples in a certain final state and phase space, the
yield of \ac{MC} events should match the yield of data events.

%Talk about specific generators
Production of \ac{MC} samples is a very complicated process and there are
several implementations of the process commonly used in the field, known as
\ac{MC} generators. Generators discussed in this thesis include \madgraph
\cite{madgraph5amc}, \sherpa \cite{sherpa2dot2}, \pythia \cite{pythia8dot2},
and \powheg \cite{powheg}.

% physics process > hadronisation
After the initial generation of an event in its simplest form, with only
fundamental particles in the final state, a few additional steps are needed to
create something comparable to what can be observed in experiments. The first
step is hadronisation: taking free partons in the final state and simulating the
effects of the strong force confining these into hadrons. This will involve
repeated emission and reabsorption of gluons, with many additional quarks
created along the way. The result is a collimated stream of hadronic particles
that could be observed in a detector.

After hadronisation, further steps are taken to match simulations to results for
a specific experiment; these are discussed in Section [detector:simulation].
%TODO update reference

% What is a tune?
% TODO Truth record < discuss in detector section, kinda only makes sense if compared
% to reco
