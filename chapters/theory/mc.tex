% ========== MONTE CARLO PREDICTIONS ==========

The \ac{SM} is tested by comparing its predicted cross sections for a set of
measurable physics processes with the rate observed in data. To isolate
processes of interest, this is often done in complicated phase spaces and
differentially across distributions, so calculating the predicted rate
is quite complex. Real measurements are also subject to the limitations and
effects of the detector; for accurate comparisons to be made between data and
predictions, these need to be accounted for.

The solution to this problem is to generate Monte Carlo (\acsfirst{MC}) events
representing the \ac{SM} prediction. For a given final state (i.e. the set of
particles produced in the interaction) a sample of events is generated for each
contributing process, containing particles with random kinematic properties
generated in such a way that the overall distribution matches the expectation
from the model. If all processes are accounted for, taking the sum of events
from all of these samples in the desired phase space gives an estimate for the
\ac{SM} prediction.

%Talk about specific generators
Production of \ac{MC} samples is a very complicated process and there are
several implementations commonly used in the field, known as
\ac{MC} generators. Generators discussed in this thesis include \madgraph
\cite{madgraph5amc}, \sherpa \cite{sherpa2dot2}, \pythia \cite{pythia8dot2},
\powheg \cite{powheg}, and \herwig \cite{herwigpp}.

% PDFs
The first step of the process is simulating the hard scatter. These rely on
matrix element calculations, at a given perturbative order,
and \ac{PDF} sets in order to accurately simulate the desired
processes at the given centre-of-mass energy.

% What needs to be added to hard scatter?
The hard scatter alone cannot mimic a full event in the detector, and so several
additional steps are needed: parton showering, applied to any strongly
interacting particles; hadronisation, converting these showers into composite
hadrons; adding the `underlying event', activity expected in the collision from
sources other than the hard scatter; pileup overlay, to account for the number
of simultaneous proton-proton interactions; and detector simulation, accounting
for effects of particles being measured by the detector.

% parton showering and hadronisation: pythia and herwig
For each parton from the hard-scatter process, a shower of \ac{QCD} activity is
produced from repeated strong interactions. This continually creates more, lower
energy, partons until the energies reach a regime where confinement effects
become relevant. Confining the shower products into colourless hadrons is
handled by a hadronisation model, such as string fragmentation
\cite{Andersson1983}. Both parton showering and hadronisation are incorporated
into \ac{MC} event generators. Some generators, such as \sherpa, \pythia, or
\herwig, can simulate the hard-scatter process, parton showering, and
hadronisation all in one. In other cases the hard-scatter process is created
with one generator, e.g.  \madgraph, and then another generator is used to add
parton showering and hadronisation to it, e.g. \pythia or \herwig.

% merging and resummation scales
The models used to simulate parton showering and hadronisation are not calculable from first
principles, and have parameters that can be adjusted to best describe observed
physics. These parameters include merging and resummation scales, describing the
merging of jets from the parton shower and the hard scatter event and the
resummation of soft gluon emissions \cite{sherpa2dot2}.

% what is the underlying event? Modelling issues => 
% tuning parameters to data to match underlying event description
Underlying event is a term used to describe activity around the hard
scatter in collisions, such as spectator interactions
(introduced in Section \ref{sec:theory-pp}).
%
Modelling the underlying event is dependent on the specific conditions
under which the collisions occur. This is mitigated by tuning certain
parameters of the model to match data; different `tunes' are available from data
collected under different conditions. For \ac{MC} samples, underlying event
modelling is typically handled by the same generators as the parton showering
and hadronisation.

% pileup overlay
Pileup overlay describes the process of adding additional soft events around the
hard scatter to simulate the presence of additional proton-proton collisions.
The number of pileup events added can be configured to match the number of
proton-proton collisions expected per bunch crossing in the detector.

% detector simulation
Having simulated the hard scatter, parton showering, hadronisation, the
underlying event, and pileup, the particles and their kinematics should be
established. The remaining step is determining how this event would be detected
in an experiment, if it were from a real collision. For the \acs{ATLAS} detector
(introduced in Section \ref{sec:detector-atlas}) this is done using \geantfour
\cite{geant4, ATLASsim1}.

% truth and reco information in MC samples
A distinction is made between information from \ac{MC} samples before and after
detector simulation. Simulated events before detector simulation is applied are
known as `truth' events, they contain only the physics processes and are not
subject to any inefficiencies or misidentification of the simulated detector.
Samples or variables with truth information are often described as `truth-level'
(or `particle-level').
Events completing the full simulation, and subsequent reconstruction (see Section
\ref{sec:methods-reconstruction}), chain are known as reconstructed \ac{MC}
events. These are typically linked so that the truth properties (or `truth
record') of reconstructed \ac{MC} events are accessible. Analyses are typically
performed using reconstructed \ac{MC} events, but in certain cases making
distinctions based on truth information is necessary.

% Link into my analyses?
Events simulated in the manner discussed here are used extensively for the two
analyses presented in this thesis (as well as for studies presented in Chapter
\ref{sec:trig}). The signal process and all backgrounds in the final state have
dedicated \ac{MC} simulation samples generated in order to model the kinematics
of events. Generation of these specific samples is discussed in Section
\ref{sec:methods-samples}.
