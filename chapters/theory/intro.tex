As with the study of science as a whole, the field of particle physics relies on
two pillars: experiment and theory. This thesis focuses on developments in
experimental particle physics, but one cannot be discussed without the other.
This opening chapter gives the theoretical background needed to put the
experimental work in context, exploring the mathematical origin of the
interactions being studied and introducing the concepts needed to perform an
analysis on data from particle physics collisions.

An introduction to the concepts of quantum field theory is given in Section
\ref{sec:theory-qft} and an overview of some of the necessary mathematical
transformations and symmetries in Section \ref{sec:theory-symmetry}. Section
\ref{sec:theory-theories} then introduces specific theories describing
different components of particle interactions, leading to the introduction of
the Standard Model of particle physics in Section \ref{sec:theory-sm}. Some
practicalities of making predictions in real experimental environments are then
discussed in Sections \ref{sec:theory-pp} and \ref{sec:theory-mc}.
