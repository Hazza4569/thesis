% ========== QFT SECTION ==========

The theoretical description of elementary particle physics is built on the
foundations laid by \ac{QFT}. This mathematical framework describes particles as
excitations in quantised fields, and the nature of these fields governs the
interactions between particles.

\ac{QFT} is the only way to reconcile the principles of quantum mechanics and
special relativity\footnote{
  Aside from theories that introduce infinite types of particle, e.g. string
  theory.
}
\cite{Weinberg1995}. Attempts at relativistic quantum wave mechanics, such as
the Dirac theory \cite{Dirac1930}, failed to explain the mechanics of
antiparticles; with the theory relying on the Pauli exclusion principle
\cite{Pauli1925} preventing `regular' particles from falling into negative
energy states, and thus working only for fermions and not bosons
\cite[p.14]{Weinberg1995}.  \ac{QFT} solves this problem and others by
introducing a quantum field, in which particles and antiparticles can be created
and annihilated; the creation and annihilation of particles represents the
interactions that \ac{QFT} describes.
The promotion of the wave function to a field gives a natural description for
many-particle systems. This better equips QFT to describe real states in nature
as, at the microscopic level, there are no true one-particle systems
\cite{Weisskopf1981}.

% Introduce lagrangian (density?)
\newcommand\Lden{\ensuremath{\mathcal{L}}\xspace}
Many different \acp{QFT} can be formulated, e.g. to describe the interactions
of different forces. The Lagrangian density, \Lden, is typically used to define
the dynamics of a given \ac{QFT}. Lagrangian densities are a necessary tool to
describe many-particle systems, related to the Lagrangian, $L$, by
%
\begin{equation*}
  L = \int d^3x \, \Lden .
\end{equation*}
%Motivating the use of the Lagrangian is beyond the scope of this thesis, but for
%an introduction to Lagrangian mechanics see Reference
%\cite[pp.461-3]{Thomson2013} and for an explanation of Lagrangian formalism in
%\ac{QFT} see Reference \cite[pp.298-306]{Weinberg1995}.
The terms Lagrangian and Lagrangian density will be used interchangeably for the
remainder of this chapter.

% QFT is perfect low-energy effective theory (Weinberg)
There is no guarantee that the `theory of everything' that we need to describe
the fundamentals of nature is a \ac{QFT}. It is true, however, that any
relativistic quantum theory applied to particles at sufficiently low energy will
look like a \ac{QFT} \cite{Weinberg1995}. Even if the theory of everything is
not a \ac{QFT}, it is clear from observations that modern particle physics is
still in a realm where energies are sufficiently low (relatively speaking) that
\ac{QFT} is accurate as an effective field theory. This has been demonstrated by
the success of the \ac{SM}, discussed in Section \ref{sec:theory-sm}.
