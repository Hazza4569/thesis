% ========== PROTON PROTON COLLISIONS SECTION ==========

In collider physics, observable interactions between \ac{SM} particles are
induced by colliding particles in a controlled environment. In the case of the
\acs{LHC} (introduced in Section \ref{sec:detector-lhc}), these collisions are
between two high-energy protons.

As protons are composite particles, interactions due to proton-proton collisions
are initiated by constituent partons in the protons. Modelling these
interactions requires knowledge of the fraction of the total proton energy
carried by its partons. The distribution of these energies is given by
\acp{PDF}, which are determined experimentally from deep inelastic scattering
measurements \cite{NNPDF3dot1}.

Interesting interactions are typically produced when two partons with high
energy fractions collide, this is known as a hard scatter. In many collisions
this will not occur, and only soft low-energy interactions take place. Even when
there is a hard scatter it will be surrounded by `spectator interactions', soft
collisions between other partons in the colliding protons. This complicates
measurements made from proton-proton collisions.

A further complication is the introduction of pileup. In order to increase the
rate of hard-scatter events, colliders are often configured to create multiple
proton-proton collisions at once. This results in many, typically soft,
interactions being produced around any hard scatter that is detected, termed
`pileup interactions'.

These effects all complicate the procedure of measuring and understanding events
from proton-proton collisions. The methods used to make practical measurements
of \ac{SM} processes under such conditions are discussed in the following
section.
