%  ========== THE STANDARD MODEL OF PARTICLE PHYSICS SECTION ==========

% TODO add citations for experimental `success` of SM

The \ac{SM} of particle physics is a gauge \ac{QFT} which combines
all of the theories discussed in Section \ref{sec:theory-theories} into a
single theoretical description. As a result, the \ac{SM} respects a symmetry of
$SU(3) \times SU(2) \times U(1)$, a combination of the \ac{QCD} and \ac{EW}
theories.

The \ac{SM} models the interactions of 12 fermions (and 12 antifermions),
mediated by 12 gauge bosons and an additional scalar boson, the Higgs. The 12
fundamental fermions are split into six leptons and six quarks, each can be
paired up across three generations.  The leptons come in charged lepton-neutrino
pairs in electron, muon, and tau families. There are three generations of
up-type and down-type quark pairs: up-down, charm-strange, and top-bottom. These
fermions all have different masses, although the values of the masses are not
derivable from the theory.

The 12 gauge bosons are those introduced by the imposed symmetries: eight gluons
from the $SU(3)$ \ac{QCD} sector and 4 electroweak bosons (W$^+$, W$^-$, Z,
$\gamma$) from the $SU(2)\times U(1)$ electroweak sector.  The gluons and photon
are observed to be massless and the W and Z bosons are massive, as discussed in
Section \ref{sec:theory-theories-ew}. The additional boson, the Higgs boson, is
also massive, although, as with the fermions, the theory does not explicitly
constrain its mass. All 17 varieties of fundamental particle (grouping the eight
gluons and 2 W bosons) are shown in Figure \ref{fig:theory-sm-particles}.

\begin{figure}[tb]
  \centering
  \includegraphics[width=.95\textwidth]{\resource{standard-model.pdf}}
  \caption{
    All fundamental particles described by the Standard Model shown with their
    masses, or limits on masses, measured from experiments. Particles are
    grouped into quarks, leptons, and bosons.
    \cite{wiki2019}
  }
  \label{fig:theory-sm-particles}
\end{figure}

Of particular interest to analyses presented in this thesis are \ac{EW} direct
multiboson interactions. These are interactions introduced in the Lagrangian
involving multiple \ac{EW} bosons; as explained in Section
\ref{sec:theory-theories}, these arise from the non-Abelian construction of the
electroweak sector. These interactions involve either three or four bosons at a
single vertex, termed \acp{TGC} and \acp{QGC} respectively. In the \ac{SM},
there are two \ac{TGC} vertices, W$^+$W$^-$Z and W$^+$W$^-$$\gamma$, and four
\ac{QGC} vertices, W$^+$W$^-$W$^+$W$^-$, W$^+$W$^-$ZZ, W$^+$W$^-$$\gamma\gamma$,
and W$^+$W$^-$Z$\gamma$ \cite[p.541]{Thomson2013}\cite{Thomson2018}. Figure
\ref{fig:theory-sm-mbis} shows examples of these as Feynman diagrams.

\begin{figure}[htbp]
  \centering
  \includegraphics[width=.8\textwidth]{\resource{mbis.pdf}}
  \caption{
    Example \acs{SM} multiboson interactions: a three-boson vertex (left) and a
    four-boson vertex (right) are shown.
  }
  \label{fig:theory-sm-mbis}
\end{figure}

% SM is effective theory, breaks down at higher energies (no gravity)
Whilst the \ac{SM} has been very successful so far when its predictions are
compared to experimental observations, it does not fully describe elementary
particle physics. One missing piece is the fourth fundamental force, gravity.
The current best theory of gravity, general relativity, is not quantisable and
thus incompatible with the \ac{QFT} structure of the \ac{SM}. At sufficiently
high energies, the \ac{SM} description of physics will break down since it
does not account for the effects of gravity. 

Other signs point to the \ac{SM} being merely an effective theory, a low-energy
approximation of some more complete theory. The \ac{SM} has $25$ parameters
\cite{Thomson2013}, such as the fermion masses, with values that have to be
constrained by experiment rather than being dictated by the theory itself.
Moreover, there appears to be some structure linking sets of parameter values
\cite{Thomson2013}, perhaps indicating that a more fundamental theory exists to
explain these patterns.

Contemporary experimental particle physics is dedicated to both testing
predictions of the \ac{SM} and searching for signatures of physics beyond it.
Either by finding evidence of a more fundamental theory, or by identifying a
breakdown in the \ac{SM} description, a significant goal in the field is to
`break' the \ac{SM}.

\subsection{Cross sections}

For each process predicted by the \ac{SM}, its cross section, $\sigma$, can be
calculated from the theory. The cross section is a measure of the probability
for the process to occur; cross sections have the dimension of area and are
typically quoted in units of `barns', where $1~\text{b} \equiv
10^{-28}~\text{m}^{2}$.

A useful tool in calculating the cross section of a process is the Feynman
diagram. A theory defines all of the allowed interactions between particles,
which can be interpreted as the set of allowed vertices in Feynman diagrams
(e.g. the vector gauge boson vertices in Figure \ref{fig:theory-sm-mbis}). An
interaction taking a certain initial state to given final state can then proceed
through all mechanisms that can be drawn as valid Feynman diagrams.

The cross section for an interaction is proportional to the amplitude squared of
the transition matrix element of the interaction. This matrix element is the
sum of the individual matrix elements for each mechanism through which the
process can proceed, i.e. for each Feynman diagram. Matrix elements are
calculated from Feynman diagrams via the Feynman rules \cite{Feynman1949}.

In principle, an infinite number of Feynman diagrams can be drawn for any
process, as more intermediary vertices can always be added. However, as more
vertices are added and the process becomes more complex, the associated
probability becomes smaller. Cross sections are therefore calculated
perturbatively: contributing Feynman diagrams are considered up to a certain
order to allow for a finite set of matrix elements to be included.

The simplest calculation for the cross section of a process is at \ac{LO}.
This involves all diagrams with the
fewest number of vertices possible to get between the desired initial and final
states.
Allowing additional diagrams which are only one step more complicated (e.g.
adding two more vertices through a loop) gives a \ac{NLO} calculation, including
a further set of diagrams makes a \ac{NNLO} calculation, and so on.

The rate of experimental observations of a process will depend on its cross
section, the total number of occurrences expected for a process in data is given
by
\begin{equation*}
  N = L\sigma,
\end{equation*}
where $L$ is the integrated luminosity, a measure of the size of the dataset
introduced in Section \ref{sec:detector-lhc}. Processes with low cross sections,
such as the subjects of the analyses in Chapters \ref{sec:vbs} and
\ref{sec:vzy}, therefore occur very infrequently and can be difficult to
measure.

%\subsection{Standard Model processes}

A selection of \ac{SM} processes are shown in Figure \ref{fig:theory-sm-xsecs},
with measurements and predictions of their `fiducial' cross sections.  A
fiducial cross section describes the rate a process in a specific phase space,
and is typically measured because analyses are limited to making measurements
only in regions sensitive to the process of interest. Cross sections relevant to
\ac{VBS} \Zy production and \VZy production are given in the figure.

\begin{sidewaysfigure}[p]
  \centering
  \includegraphics[width=.8\textwidth,clip,trim=0 0 0 45px]{\resource{xsecs.pdf}}
  \caption{
    Cross sections for \acs{SM} processes, with measurements from the
    \acs{ATLAS} experiment shown alongside predictions from theory.
    Cross sections are corrected for branching fractions and subject to fiducial
    phase spaces of the analyses. From the processes discussed in this thesis,
    \acs{VBS} \Zy (labelled as \Zyjj EWK) and W\Zy, one component of \VZy, are
    included.
    \cite{ATLASsmsummary2023}
  }
  \label{fig:theory-sm-xsecs}
\end{sidewaysfigure}
