%  ========== THE STANDARD MODEL OF PARTICLE PHYSICS SECTION ==========

\acused{SM}
The Standard Model (SM) of particle physics is a gauge \ac{QFT} which combines
all of the theories discussed in Section \ref{sec:theory-theories} into a
single theoretical description. As a result, the \ac{SM} respects a symmetry of
$SU(3) \times SU(2) \times U(1)$, a combination of the \ac{QCD} and \ac{EW}
theories.

The \ac{SM} models the interactions of 12 fermions, mediated by 12 gauge bosons
and an additional scalar boson, the Higgs. The 12 fundamental fermions are split
into six leptons and six quarks, each can be paired up across three generations.
The leptons come in charged lepton-neutrino pairs in electron, muon, and tau
families. There are three generations of up-type and down-type quark pairs:
up-down, charm-strange, and top-bottom. These fermions all have unique masses,
although the values of the masses are not derivable from the theory.

The 12 gauge bosons are those introduced by the imposed symmetries: eight gluons
from $SU(3)$, 3 weak bosons (W$^+$, W$^-$, Z) from $SU(2)$ and the photo
($\gamma$) from $U(1)$. The gluons and photon are all required to be massless by
the theory and the weak bosons have related masses, as discussed in Section
\ref{sec:theory-theories-ew}. The additional boson, the Higgs boson, is also
massive, although the theory does not explicitly constrain its mass.  All 17
varieties of fundamental particle (grouping the eight gluons and 2 W bosons) are
shown in Figure \ref{fig:theory-sm-particles}.

\begin{figure}[htbp]
  \centering
  \includegraphics{\resource{standard-model.pdf}}
  \caption{
    All fundamental particles described by the Standard Model shown with their
    masses, or limits on masses, measured from experiments. Particles are
    grouped into quarks, leptons, and bosons, with the grey outlines grouping
    each gauge boson with the fermions it acts on.
  }
  \label{fig:theory-sm-particles}
\end{figure}

Of particular interest to analyses presented in this thesis are \ac{EW} direct
multiboson interactions. These are interactions introduced in the Lagrangian
involving multiple \ac{EW} bosons. As explained in Section
\ref{sec:theory-theories}, these arise from the non-Abelian construction of the
electroweak sector. These interactions involve either three or four bosons at a
single vertex, termed \acp{TGC} and \acp{QGC} respectively. In the \ac{SM},
there are two \ac{TGC} vertices, WWZ and WW$\gamma$, and four \ac{QGC} vertices,
WWWW, WWZZ, WW$\gamma\gamma$, and WWZ$\gamma$.

% MULTIBOSON INTERACTIONS FEYNMAN

% SM is effective theory, breaks down at higher energies (no gravity)
Whilst the \ac{SM} has been very successful so far when its predictions are
compared to experimental observations, it does not fully describe elementary
particle physics. One missing piece is the fourth fundamental force, gravity.
The current best theory of gravity, general relativity, is not quantisable and
thus incompatible with the \ac{QFT} structure of the \ac{SM}. At sufficiently
high energies, the \ac{SM} description of physics will break down without
accounting for the effects of gravity. 

Other signs point to the \ac{SM} being merely an effective theory, a low-energy
approximation of some more complete theory. The \ac{SM} has $\sim25$ parameters,
such as the fermion masses, with values that have to be constrained by
experiment rather than being dictated by the theory itself. This is not
characteristic of a truly fundamental theory as it leaves many questions
unanswered.

Contemporary experimental particle physics is dedicated to both testing
predictions of the \ac{SM} and searching for signatures of physics beyond it.


% Goal of particle physics is to `break` SM
