% Go into subsections below on different sub-QFTs of the SM.
\subsection{Quantum electrodynamics}

\acronymused{QED}
Quantum electrodynamics (QED) describes electromagnetic interactions
between charged particles.  As a gauge theory respecting local $U(1)$
transformations it introduces a single massless gauge boson, the photon. When
formulated as a \ac{QFT}, \ac{QED} must therefore describe fermions, photons,
and the interactions between them.  The \ac{QED} Lagrangian can be built from
three terms:
\newcommand\LQED{\ensuremath{\Lden_\text{QED}}\xspace}
\newcommand\LDirac{\ensuremath{\Lden_\text{Dirac}}\xspace}
\newcommand\LEM{\ensuremath{\Lden_\text{EM}}\xspace}
\newcommand\Lint{\ensuremath{\Lden_\text{int}}\xspace}
%
\begin{equation*}
  \LQED = \LDirac + \LEM + \Lint.
\end{equation*}
%

The \LDirac term describes the dynamics of a fermion under Dirac theory. This is
given by
%
\begin{equation*}
  \LDirac = \overline{\psi}(i\gamma^\mu\partial_\mu - m)\psi,
\end{equation*}
where $\psi$ is a Dirac spinor, a four-component fermion field representing
up-down and particle-antiparticle states for a fermion of mass $m$. The
$\gamma^\mu$ are a set of matrices accounting for fermion spin, these are
commonly absorbed into the covariant derivative using the Feynman slash
notation, $\gamma^\mu\partial_\mu \to \slashed{\partial}$.

Maxwell's equations provide the terms \LEM and \Lint, describing the kinematics
of the photon and their interaction with charged fermions:
%
\begin{align*}
  \LEM  &= -\frac14 F^{\mu\nu}F_{\mu\nu} = -\frac14 (F_{\mu\nu})^2, \\
  \Lint &= -J^\mu A_\mu = -q\overline{\psi}\gamma^\mu\psi A_\mu.
\end{align*}
%
Here $A_\mu$ is the electromagnetic vector potential; $F_{\mu\nu}$ is the
electromagnetic field tensor, given by
\begin{equation}
  F_{\mu\nu} = \partial_\mu A_\nu - \partial_\nu A_\mu;
  \label{eqn:theory-theories-qed-fieldstrength}
\end{equation}
and $J^\mu = q\overline{\psi}\gamma^\mu\psi$ is a conserved current, satisfying
$\partial_\mu J^\mu = 0$, for a fermion of charge $q$.

Combining these, and simplifying by defining a gauge covariant derivative
\begin{equation}
  D_\mu = \partial_\mu - iQA_\mu,
  \label{eqn:theory-theories-qed-covariant}
\end{equation}
gives the full \ac{QED} Lagrangian:
%
\begin{equation*}
  \LQED = \overline{\psi}(i\slashed{D} - m)\psi
                   - \frac14 (F_{\mu\nu})^2.
  \label{eqn:theory-theories-qed-lagrangian}
\end{equation*}

This result could instead be obtained by starting from the Dirac Lagrangian and
enforcing local gauge invariance through the covariant derivative transformation
in Equation \ref{eqn:theory-theories-qed-covariant}. The form of Equation
\ref{eqn:theory-theories-qed-lagrangian} is recovered with the inclusion of the
\LEM term, which is the only locally gauge invariant formulation of a kinetic
term for the field $A_\mu$. Its invariance can be demonstrated from the local
gauge invariance of the commutator $[D_\mu,D_\nu]$, given
\begin{equation}
  \begin{split}
    [D_\mu,D_\nu] &= iq(\partial_\mu A_\nu - \partial_\nu A_\mu)\\
                  &= iqF_{\mu\nu}.
  \end{split}
  \label{eqn:theory-theories-qed-covariant-commutator}
\end{equation}
%
This technique for deriving field tensors from potentials will be relevant in
discussion of other theories. For the complete derivation of \LQED through the
requirement of gauge invariance, see \cite[pp.482-6]{Peskin1995}.

% Local or global U(1)? What is the difference??
% U(1) transformation is transformation by a Unitary 1x1 matrix -- a phase
% change
%
% Gauge-fixing term citation: \cite[p.205]{Aitchson2013}

\subsection{Yang-Mills theory}
\label{sec:theory-theories-ym}

The \ac{QED} theory corresponds to a $U(1)$ gauge symmetry, and
as such is Abelian. Constructing a non-Abelian gauge theory respecting $SU(n)$
symmetries is more complex, but generically solved by the Yang-Mills theory
\cite{Yang1954}.

For generators of the Lie algebra $T^a$ and structure constant $f^{abc}$,
a gauge covariant derivative can be defined by
\begin{equation*}
  D_\mu = \partial_\mu - igT^a A^a_\mu,
\end{equation*}
where a vector field $A^a_\mu$ is required for each generator of the $SU(n)$
group, and $g$ is a coupling constant. Here the exponents $a,b,c$ index the
generators of the Lie algebra, whilst $\mu,\nu$ index space-time dimensions, as
per convention. 

This gauge covariant derivative is a generalisation of the Abelian form, given in
Equation \ref{eqn:theory-theories-qed-covariant}, and acts on an $n$-plet,
$\psi$, of spinors, $\psi_i$, rather than on a single spinor as in the \ac{QED}
case. The generators of $SU(n)$ serve to transform $\psi_i$ into one-another
through abstract rotations.

In analogy to the Abelian case in Equation
\ref{eqn:theory-theories-qed-covariant-commutator}, the commutator is used to
define a set of field strength tensors: the commutator
\begin{equation*}
  [D_\mu, D_\nu] = -igF^a_{\mu\nu}T^a
\end{equation*}
holds for a field strength satisfying
\begin{equation*}
  F^a_{\mu\nu}T^a = \partial_\mu A^a_\nu T^a - \partial_\nu A^a_\mu T^a
                  - ig[A^a_\mu T^a, A^b_\nu T^b].
\end{equation*}
%
This shows explicitly that for an Abelian symmetry group, with $[T^a,T^b]=0$,
the form of Equation \ref{eqn:theory-theories-qed-fieldstrength} is recovered.
For non-Abelian gauge theories, however, the additional term is introduced, with
dependence on the potentials from other gauge fields. This is known as a
`nonlinear' term, and represents self-interaction of gauge fields.

Substituting in the general form for the commutator of $SU(n)$ generators given
in Equation \label{eqn:theory-symmetry-gen-commutator}, the general form of a
field strength for such a non-Abelian gauge theory is given by
\begin{equation}
  F^a_{\mu\nu} = \partial_\mu A^a_\nu - \partial_\nu A^a_\mu
               + gf^{abc}A^b_\mu A^c_\nu.
  \label{eqn:theory-theories-ym-fieldstrength}
\end{equation}
%
This leads to the Yang-Mills Lagrangian, the most general renormalisable
Lagrangian for a theory respecting $SU(n)$ symmetry expressed in terms of these
field strengths,
\begin{equation}
  \Lden_\text{YM} = \overline{\psi}(i\slashed{D} - m)\psi -\frac{1}{4} (F^a_{\mu\nu})^2.
  \label{eqn:theory-theories-ym-lagrangian}
\end{equation}

A more detailed derivation for the Yang-Mills theory can be found in Reference
\cite[pp.486-91]{Peskin1995}.


\subsection{Quantum chromodynamics}

The strong force, or \ac{QCD}, is modelled by a non-Abelian gauge theory. The
fundamental particles that \ac{QCD} acts on are quarks.
A charge is attached to quarks that works as the electric charge in \ac{QED},
but to describe the observed dynamics of the strong force this charge comes in
three flavours. This charge is known as colour charge, and the three flavours
are red, green, and blue.

Transformation of a quark from one quark to another acts as a gauge symmetry in
the theory; if all red quarks became green, green became blue, and blue became
red the predictions of the theory would remain unchanged. This rotation of
colour charge is described perfectly by the $SU(3)$ symmetry.

\ac{QCD} can therefore be constructed as a Yang-Mills theory under $SU(3)$,
with the Lagrangian following that of Equation
\ref{eqn:theory-theories-ym-lagrangian}. Given that the $SU(3)$ group has 8
generators, the theory of \ac{QCD} relies on 8 gauge bosons to mediate
interactions. These bosons are called gluons, and the 8 varieties are
represented as different colour combinations of the gluons.

%Asymptotic freedom
It can be shown that the coupling constant (appearing
\ref{eqn:theory-theories-ym-fieldstrength}) for \ac{QCD} is actually not a
constant, and is dependent on the energy scale of interactions \cite{PDG2022}.
For high energies (i.e. large distance scales) the coupling strength tends to
zero, leading to `asymptotic freedom'. Quarks can only exist as free particles
in the high energy limit, whereas at and below energies of $\sim1$ GeV the
coupling strengths are sufficiently high that quarks are exclusively confined in
composite particles, hadrons.


\subsection{The electroweak theory}
\label{sec:theory-theories-ew}

%747
The \ac{EW} theory describes the weak interaction, and also supersedes \ac{QED}
as it is unifies the weak and \ac{EM} forces.  It combines the local $U(1)$
invariance of \ac{QED} with the symmetry of the weak interaction under local
$SU(2)$ transformations, and thus is described by $SU(2)\times U(1)$ symmetry.
The gauge covariant derivative for the theory is given by
%
\begin{equation*}
  D_\mu = \partial_\mu - igT^a W^a_\mu - \frac12 ig'B_\mu,
\end{equation*}
where $a$ indexes the 3 generators of the $SU(2)$ algebra, $T^a$, and their
corresponding gauge fields, $W^a_\mu$. The field $B_\mu$ is required for
invariance under $U(1)$ symmetry.

The corresponding field strengths are then
\begin{equation*}
  \begin{split}
  W^a_{\mu\nu} &= \partial_\mu W_\nu - \partial_\nu W_\mu
               + g \varepsilon^{abc}W^b_\mu W^c_\nu \\
  B_{\mu\nu}   &= \partial_\mu B_\nu - \partial_\nu B_\mu \\
  \end{split}
\end{equation*}

To treat this as a conventional Yang-Mills theory would result in four massless
gauge bosons, three corresponding to the three $W^a_\mu$ fields mediating the
weak interaction, and one from the $B_\mu$ field mediating \ac{EM} interactions.
The weak interaction being mediated by massless bosons would result in
asymptotic freedom of weakly interacting particles, as seen in \ac{QCD}; this is
not what is observed experimentally, so this Yang-Mills theory cannot form a
complete description of the interaction.

The missing ingredient is  spontaneous \ac{EW} symmetry breaking through the
Higgs mechanism. When the Higgs field takes a non-zero vacuum expectation value,
a rotated set of the potentials, $W^a_\mu$ and $B_\mu$, appears in the
Lagrangian instead, resulting in gauge fields:
\begin{equation*}
  \begin{split}
    W_\mu^\pm &= \frac1{\sqrt2} (W_\mu^1 \mp iW_\mu^2), \\
    Z_\mu     &= \frac1{\sqrt{g^2 + g'^2}}( gW_\mu^3 - g' B_\mu ), \\
    A_\mu     &= \frac1{\sqrt{g^2 + g'^2}}( g'W_\mu^3 + g B_\mu ). \\
  \end{split}
\end{equation*}
%
The $W^\pm_\mu$ and $Z_\mu$ fields have mass terms in the Lagrangian, with
masses parametrised by
\begin{equation*}
  m_W = \frac{gv}2, \,\,\, m_Z = \frac{\sqrt{g^2 + g'^2}v}2,
\end{equation*}
where $v$ is the vacuum expectation value.

This converts the previous form of the theory with four massless bosons to a
description with three massive bosons, the two charged W bosons and the
neutral Z boson, and one massless boson, which is identified as the photon.

Due to the non-Abelian $SU(2)$ symmetry group used to build this theory, we
see self-interaction terms arising in the Lagrangian, as explained in Section
\ref{sec:theory-theories-ym}. The resulting interactions are discussed in
Section \ref{sec:theory-sm}.

Another result of the Higgs mechanism is the introduction of a massive scalar
boson, the Higgs boson.  A detailed description of the Higgs mechanism is beyond
the scope of this thesis, but Reference \cite{Peskin1995} explains the form of
equations given here as well as giving much more detail on the topic.
