%%% Peak size subsubsection %%%

The first parameter to investigate is the size of the area used to calculate
energies. In the algorithm as described in Section
\ref{sec:trig-eratio-algorithm-initial}, the energies used in comparisons and in
the final \eratio calculation are always the energies of a single cell. This
could be modified by instead summing the energy of a cell with that of its
neighbours in $\eta$ to reduce sensitivity to small fluctuations. The number of
cells summed is labelled the `peak size', where the default algorithm would have
a peak size of one. With a peak size greater than one the algorithm uses a
`sliding window' approach, so the step size is still a single cell despite the
energy value coming from a larger area. For an odd numbered peak size the energy
of a cell is added to that of its neighbours on each side. For an even numbered
peak size, neighbours in positive $\eta$ are preferred. Figure
\ref{fig:trig-eratio-algorithm-peak-diagram}
shows how cells are included in the calculated energy.

\begin{figure}
  \centering
  \includegraphics[width=\textwidth]{\resource{eratio-peak-diag.pdf}}
  \caption{
    Diagram showing which cells contribute to the energy sum for the seed cell
    (bright yellow) for different peak sizes. The horizontal axis represents
    $\eta$ and the vertical axis $\phi$. In each case, the calculated energy
    would be the sum of the energies of the cells contained within the red box.
  }
  \label{fig:trig-eratio-algorithm-peak-diagram}
\end{figure}

Performance for the \eratio algorithm was tested with peak size values from one
to five. The results are shown in Figure
\ref{fig:trig-eratio-algorithm-peak-results}.
Comparing the benchmark value of background rejection at 95\% signal efficiency,
it is clear that a peak size of one (i.e. the same as the baseline algorithm)
gives the best results, with performance degrading as more cells are added to
the energy sum. This appears to be generally true for background rejection at
all signal efficiencies. This suggests that the benefit of the fine granularity
of each energy measurement outweighs the negative impact of any potential
fluctuations that the increased peak size would smear out.

\begin{figure}
  \centering
  \includegraphics[width=\textwidth]{\resource{eratio-peak-res.pdf}}
  \caption{
    Results for calculating \eratio with different peak size options.  Plots
    show background rejection as a function of signal efficiency for each peak
    size tested (left) and background rejection at 95\% signal efficiency as a
    function of peak size (right).
  }
  \label{fig:trig-eratio-algorithm-peak-results}
\end{figure}
