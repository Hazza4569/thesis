%%% Initial algorithm subsubsection %%%

Identifying the two highest energy cells is done in three stages: locating the
seed, identifying candidate secondary maxima, and comparing results.

The \ac{GEP} will receive a seed location from the \ac{eFEX} identifying which
SuperCell has the highest energy. The cells within this SuperCell are compared
with one another to find which has the highest energy, this becomes the seed
cell for the \eratio algorithm.

The algorithm will then perform a stepwise search from the seed outwards to
identify peaks in energy. On each step the energy gradient is calculated as
%
$\Delta E = E^\text{cell}_\text{next} - E^\text{cell}_\text{prev}$,
%
where $E^\text{cell}_\text{next}$ is the energy of the cell being stepped to,
and $E^\text{cell}_\text{prev}$ is the energy of the cell being stepped from.
From the first step $\Delta E$ should be negative, as the seed will have a
higher energy than the surrounding cells, but on subsequent steps $\Delta E$ may
become positive, marking that a minimum-energy point has been passed. If, after
this, $\Delta E$ becomes negative again it indicates that the previous cell was
a local maximum; in this case that cell is added to a list of candidate
secondary maxima, and the search stops along this route. If the edge of the
available range of cells is reached before $\Delta E$ turns positive then no
candidate is saved. If the edge is reached after $\Delta E$ turns positive, but
before it turns negative again, then the last cell in the range is taken to be
the candidate.

This stepwise search is done in six different routes from the seed: one route
where each step from the seed is in positive $\eta$, one in negative $\eta$, two
where the first step is in positive $\phi$ before proceeding in positive or
negative $\eta$, and two following the same pattern with the first step in
negative $\phi$. Figure \ref{fig:trig-eratio-algorithm-routes} shows a schematic
of these six routes, alongside a schematic depicting the peak location strategy.

\begin{figure}
  \centering
  \includegraphics[width=.95\textwidth]{\resource{eratio-initial.pdf}}
  \caption{
    Diagram showing the 6 different routes in which the \eratio algorithm
    searches for secondary maxima (left) and how the algorithm identifies
    secondary maxima by tracking energy gradients along each step (right).
  }
  \label{fig:trig-eratio-algorithm-routes}
\end{figure}

Once the stepwise search is complete, up to 6 candidate secondary maxima will
have been identified. The candidate with the largest energy is taken as the
secondary maximum and, with the seed as the maximum, \eratio can be calculated
using Equation \ref{eqn:trig-eratio-algorithm-eratio}.

The performance of this baseline algorithm was investigated using simulations.
Figure \ref{fig:trig-eratio-algorithm-baseline-results} shows the results,
comparing the response in signal and background as a function of the calculated
\eratio value and the fraction of each that would pass a given \eratio
threshold. The background rejection as a function of signal efficiency is also
shown, the baseline algorithm achieves a background rejection of 2.3 at 95\%
signal efficiency.

\begin{figure}[tbph]
  \centering
  \begin{subfigure}{.48\textwidth}
    \includegraphics[width=\textwidth]{\resource{ET20_defparams_ERplot.pdf}}
    \vspace{-2em}\caption{} %blank captions add letter labels
  \end{subfigure}
  \hfill
  \begin{subfigure}{.48\textwidth}
    \includegraphics[width=\textwidth]{\resource{ET20_defparams_integral.pdf}}
    \vspace{-2em}\caption{} %blank captions add letter labels
  \end{subfigure}
  \\
  \begin{subfigure}{.48\textwidth}
    \includegraphics[width=\textwidth]{\resource{ET20_defparams_effrej.pdf}}
    \vspace{-2em}\caption{} %blank captions add letter labels
  \end{subfigure}
  \caption{
    Performance of baseline \eratio algorithm on signal ($Z\to ee$) and
    background ($JZ0W$) clusters. Plots show (a) a histogram of calculated
    \eratio values for each cluster, (b) the integral of (a) with a grey dashed
    line indicating the values at 95\% signal efficiency, and (c) the background
    rejection of an \eratio threshold corresponding to a given signal
    efficiency.
  }
  \label{fig:trig-eratio-algorithm-baseline-results}
\end{figure}
