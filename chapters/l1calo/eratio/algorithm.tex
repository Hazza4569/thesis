%%% Algorithm Design Subsection %%%

The focus for this study is on the impact of a single variable in \egamma
discrimination, \eratio.
\eratio is a shower-shape variable, already used in the \ac{HLT}. The definition used here is
%
\begin{equation}
  \label{eqn:trig-eratio-algorithm-eratio}
  E_\text{ratio} = \frac{E_2}{E_1} %\frac{E_1 - E_2}{E_1 + E_2},
\end{equation}
%
where $E_1$ and $E_2$ are the first and second most energetic cells in Layer 1
of the \ac{EM} calorimeter in an area around the centre of the shower.%
\footnote{
  This is different to the definition used in the \ac{HLT}, which instead is
  \eratio$ = (E_1 - E_2)/(E_1+E_2)$. The simpler definition is preferred here in
  the spirit of reducing calculation in firmware, though the two forms are a
  transformation of one another.
}

The \eratio variable is designed to discriminate against substructure in a
shower.  A shower with multiple distinct branches (e.g. $\pi^0\to\gamma\gamma$)
might produce two peaks of similar energy and give an \eratio value close to
one, whereas a shower with a single peak (as expected from \egamma clusters)
would give an \eratio value close to zero.

Calculating this variable in high-level software is straightforward and requires
no optimisation. However, identifying the two required maxima involves a large
number of comparisons between cell energies. An algorithm developed to run on
hardware should be as simple as possible, therefore designing an alternate
implementation is beneficial to minimise the impact of this algorithm on the
latency of the \ac{GEP} system.

A simple approach to finding the two highest energy cells in a cluster is to
form a sorted list of all energies from Layer 1 cells, or at least sufficiently
sorted to be confident in the highest two energies. Sorting algorithms are a
very well-understood problem and heavily optimised but this approach is very
rigid, not allowing for any tuning of the algorithm. For example, a cluster may
have no substructure but fall on the boundary between two cells, depositing a
similar amount in each. This would result in a high, background-like, \eratio
value. To avoid this the algorithm could include a minimum distance between
cells considered to be the two maxima, or try to identify minima between the
two, this would greatly complicate a list-sorting approach.

The most complete, but resource-heavy, method might consist of fitting some
functional form to the energies as a function of $\eta$ and $\phi$ to extract
the peak energies. This might work in software but is very computationally
expensive, even if possible to implement in firmware it is likely not worth the
latency it would require.

The desired solution is an algorithm for calculating an \eratio-type variable
that comes somewhere between these two options, more adaptable than the
list-sorting approach and less resource-heavy than the peak fitting. The
following sections explore such an algorithm:
a baseline algorithm for finding secondary maxima in clusters in the \ac{GEP}
is established in Section \ref{sec:trig-eratio-algorithm-initial}; parameters of
the algorithm are tuned using simulations in Sections
\ref{sec:trig-eratio-algorithm-peak}, \ref{sec:trig-eratio-algorithm-exclusion},
and \ref{sec:trig-eratio-algorithm-limit}; and a summary of the results and
recommended parameters, as well as additional adjustments that could be
made with further studies,
is given in Section \ref{sec:trig-eratio-algorithm-summary}.

\subsubsection{Initial algorithm}
\label{sec:trig-eratio-algorithm-initial}
\input{\relpath{algorithm-initial.tex}}

\subsubsection{Peak size}
\label{sec:trig-eratio-algorithm-peak}
\input{\relpath{algorithm-peak.tex}}

\subsubsection{Exclusion region}
\label{sec:trig-eratio-algorithm-exclusion}
\input{\relpath{algorithm-exclusion.tex}}

\subsubsection{Search limit}
\label{sec:trig-eratio-algorithm-limit}
\input{\relpath{algorithm-limit.tex}}


%TODO details that are not included:
% - Selection on clusters before eratio calculation applied
% - Pileup limitations in samples used
