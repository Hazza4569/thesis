%% INTRODUCTION %%

The \ac{GEP}, when introduced in the Phase-II upgrade, will aim to improve
discrimination in the hardware trigger for many signatures, but notably for
\egamma objects. The \ac{GEP} will be working alongside the \ac{eFEX} system,
introduced in Phase I, but will have access to more information, giving it
potential to improve upon decisions made by the \ac{eFEX}. To realise this
improvement, new algorithms will need to be implemented in the \ac{GEP} to take
advantage of the finer granularity information available to it.

%Paragraph about performance studies and how they're conducted in general?
Designing algorithms to be used in future hardware systems is achieved through
prospective performance studies. Performance studies use simulations of the
expected response of a system to evaluate the performance of individual
algorithms. These studies benefit from the ease of implementation of algorithms
in high-level software but it is still important to consider the complexity of
implementation in firmware when designing algorithms. Performance studies are
typically the first step in designing a system as evaluating performance in
simulations before a system is built can inform the design of the hardware.

This section explores the specific implementation and possible performance of
the \eratio algorithm in the \ac{GEP}, expected to significantly improve
discrimination for \egamma by making use of fine granularity input information
\cite[p.~126]{ATLAS-TDR-TDAQ-PhaseII}.
%
Section \ref{sec:trig-eratio-samples} details the samples used for evaluating
algorithm performance, Section \ref{sec:trig-eratio-simulation} discusses how
the \ac{GEP} itself is simulated, Section \ref{sec:trig-eratio-benchmarks} gives
metrics used to evaluate performance, then Section
\ref{sec:trig-eratio-algorithm} goes through the process of designing an
algorithm, the outcome of which is evaluated in Section
\ref{sec:trig-eratio-algorithm-summary}.
