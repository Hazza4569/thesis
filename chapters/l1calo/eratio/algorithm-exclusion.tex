%%% Exclusion region subsubsection %%%

Another alteration tested on the \eratio algorithm is an `exclusion region'
around the seed, i.e. a number of cells close to the seed in which secondary
maxima will not be searched for. 
%
An $n$-cell exclusion region means making the first step along any route $n$
cells away from the seed in $\eta$.
%
Since secondary maxima can
be found as soon as two steps have been taken from the seed cell (they cannot be
found on the first step as it will always be a step down from the seed), this
excludes all cells in an $\eta$ range from $-n$ to $n$ (in relative
coordinates) from being considered secondary maxima. Comparatively, the baseline
algorithm with no exclusion region can find secondary maxima anywhere but the
four cells directly adjacent to the seed. Figure
\ref{fig:trig-eratio-algorithm-exclusion-diagram} highlights the effect of the
exclusion region.

\begin{figure}
  \centering
  \includegraphics[width=\textwidth]{\resource{eratio-exclude.pdf}}
  \caption{
    Diagram showing how the introduction of an exclusion region to the \eratio
    algorithm prevents secondary maxima close to the seed from being selected.
    %the baseline algorithm with no exclusion region is shown alongside the
    %algorithm with a one or two cell exclusion region.
    Red arrows mark each of the six paths traversed by the stepwise algorithm.
    Blue dots mark each step where the energy gradient is calculated. The shaded
    grey area shows cells that cannot be selected as a candidate secondary
    maximum, due to either being skipped over or being the first step from the
    seed.
  }
  \label{fig:trig-eratio-algorithm-exclusion-diagram}
\end{figure}

Performance for the \eratio algorithm was tested with exclusion regions from
between one and five cells, shown in Figure
\ref{fig:trig-eratio-algorithm-exclusion-results} alongside the baseline
algorithm with no exclusion region. This time a clear increase in performance is
visible compared to the initial form of the algorithm, with a one-cell exclusion
region attaining a background rejection of 3.1 at 95\% signal efficiency. For
most signal efficiencies the one-cell exclusion still seems to perform best,
though perhaps competing with a two-cell exclusion region for very high signal
efficiencies. Since the only difference between no exclusion region and the
one-cell case is that cells diagonally adjacent to the seed are excluded, these
results suggest signal clusters frequently create secondary peaks on these
diagonals; this could stem from incident particles falling close to the corner of
a cell.

\begin{figure}
  \centering
  \includegraphics[width=\textwidth]{\resource{eratio-exclude-res.pdf}}
  \caption{
    Results for calculating \eratio with different or no exclusion region
    definitions. Plots show background rejection as a function of signal
    efficiency for each tested exclusion region (left) and background rejection
    at 95\% signal efficiency as a function of exclusion region size (right).
  }
  \label{fig:trig-eratio-algorithm-exclusion-results}
\end{figure}

Given that cell widths vary significantly in different regions of the
calorimeter, the performance of the \eratio algorithm with different exclusion
widths was also tested as a function of $\eta$. Figure
\ref{fig:trig-eratio-algorithm-exclusion-etadep} compares background rejection
at 95\% signal efficiency in several $\eta$ regions. It is evident that the
one-cell exclusion region performs best regardless of calorimeter geometry. The
difference between one-cell and two-cell exclusion regions is much more drastic in the high-$\eta$
endcap regions, here the strips are less granular so likely the larger exclusion
regions are starting to miss real secondary peaks in background clusters.

\begin{figure}
  \centering
  \includegraphics[width=.9\textwidth]{\resource{eratio-exclude-eta.pdf}}
  \caption{
    Plot of background rejection at 95\% signal efficiency as a function of
    pseudorapidity, $\eta$, for \eratio algorithms with different exclusion
    regions.
    %NOTE: bins don't correspond precisely with calorimeter geometry regions.
    %Not sure what the motivation was for this binning and I can't easily change
    %it.
  }
  \label{fig:trig-eratio-algorithm-exclusion-etadep}
\end{figure}
