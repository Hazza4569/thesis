%INTRODUCE THE TRIGGER
%...
Protons from the LHC collide in the ATLAS detector around 40 million times every
second \cite{ATLAS2020a}.
The data associated with all of these collisions would be impossible to read from the detector
in real time and equally impossible to store to be used in analyses later on.
The trigger is an element of the detector which reduces the number of events that are
kept by throwing away those that will be less interesting for physicists to investigate.
The end result is around 1000 events accepted by the trigger every second.

%HARD VS SOFT SCATTER
%TODO need to find a source on this really for some more precise vocab
When two protons collide it can happen at a variety of different angles; it is rare
that the collision is direct and involves the full energy of each proton, most of the time
the collision is more akin to the protons glancing off each other. Direct collisions are
described as hard collisions or hard-scatter processes, whereas soft collisions
are non-direct. It turns out that the most interesting physics happens in the hard-scatter
processes where higher energy systems will be created. This can help solve the problem of
which events to keep; the trigger can be tuned to select hard-scatter events to be saved
and discard collision events with only soft scatter processes.

%describe rough format of the trigger, how it selects events and reduces rate.
%multi-level etc.
To handle the rate of incoming data, the trigger needs to make decisions very quickly,
while the full data is stored in temporary buffers.
This is achieved using a multi-level system which can be broken down into two main parts:
the Level-1 (L1) trigger which makes fast and loose cuts to reduce the rate significantly,
from the initial $\sim40$ MHz down to $100$ kHz,
and the High Level Trigger (HLT) which takes this reduced rate
and refines the decisions using more detailed information. The L1 trigger uses
custom-built electronics to allow it to make the relevant calculations as fast as
possible. The HLT consists of typical off-the-shelf computer components, and can make use
of the reduced rate coming through Level 1 to make more complicated calculations.

The L1 trigger uses a limited subset of information from the detector. There are two main L1
systems each dedicated to a detector component: L1Calo uses the calorimeters, and L1Muon
uses the Muon spectrometer. L1Calo is further split into systems looking for specific
particle signatures to trigger on: jets, electrons/photons, tau leptons, and global event quantities
such as missing energy.

The work discussed in this report is on the electron and photon trigger. The following
sections will therefore focus on the L1Calo \egamma trigger.

\subsection{Upgrades}\label{sec:upgrade}
%Talk in general about upgrades?
%TODO discuss LHC schedule in earlier section?
As technology advances the ATLAS detector and trigger undergo upgrades to enhance their
physics capabilities. There are two upgrades ongoing: the Phase-I upgrade currently being
commissioned going into Run 3 of the LHC and the Phase-II upgrade planned for the
transition to the High Luminosity LHC (HL-LHC), i.e. Run 4 of the LHC. The systems planned
for these upgrades will be referred to as the Phase-I and Phase-II systems, respectively.
The system in place in the most recent run, Run 2, will be referred to as the legacy
system. Work presented here is on the Phase-II system, but relevant parts of earlier
systems will also be discussed for context.

In the legacy system \egamma candidates are handled by the Cluster Processor (CP). The CP
receives energies for each tower in the calorimeter, achieved by summing over energies of
all cells in the tower. This gives a reduced granularity picture of events, allowing it to
be processed as fast as possible. The CP makes decisions on \egamma candidates based on
simple transverse energy and isolation calculations.

The Phase-I system improves on this by introducing a new module for the \egamma trigger,
the electromagnetic Feature Extractor (eFEX). The eFEX receives more granular data from
the calorimeter than the CP received, now in the form of ``SuperCells'' where a
calorimeter tower will typically contain 10 SuperCells. Each SuperCell energy is still
formed by summing over the energy of the constituent cells, where there are between four
and eight cells in a SuperCell. With this more segmented view of the energy deposits in an
event the eFEX uses more sophisticated criteria for selecting \egamma objects: a
transverse energy is still used but now the isolation is handled by three discriminating
variables targeting differences between \egamma objects and generic QCD background.

\begin{figure}
  \centering
  \includegraphics[width=.9\linewidth]{\resource{p2-ey.pdf}}
  \caption{
    Planned structure of the L1 trigger for the Phase-II upgrade (Referred to
    as Level 0 in the diagram for technical reasons). The components and data flow
    necessary for the \egamma trigger are highlighted in red.
  }
  \label{fig:p2diag}
\end{figure}

%TODO discuss pileup in detector section
With the move to HL-LHC at Phase-II there will be a significant increase in pileup as the
number of collisions per bunch crossing is increased: from the initial design value of 25,
to 50-80 in Phase-I, up to as high as 200 $pp$ collisions per bunch crossing in Phase-II.
To maintain sensitivity to
electroweak physics it is important that the acceptance of the trigger is kept as high as
possible whilst rejecting enough events to meet the required rates.
To achieve this, the Phase-II system adds the Global Event Processor (GEP), a new module
which will work in tandem with the eFEX introduced in Phase I. The GEP will have access
to individual cells of the electromagnetic calorimeter, even more granular than the
Phase-I upgrade. It will also have more time available in which to make decisions than previous
trigger systems. This time and information can be used to exploit fine differences in the shower
shapes between \egamma objects and background QCD. Trigger objects from the
GEP and eFEX will then be merged to give a final decision on each event.
Figure \ref{fig:p2diag} shows the structure of the Level 1 trigger in Phase II.

\subsection{Discriminating variables} \label{sec:trig-vars}

The legacy system uses a single isolation value for a candidate \egamma object, the sum of
energies in the 12-tower ring surrounding a $2\times2$ tower cluster. This is based on the
idea that \egamma objects will be isolated (i.e. little other activity around them) and
background QCD objects will not.
The Phase-I and II systems take a more complex approach to the problem and use multiple
discriminating variables in place of the isolation. Each variable is designed to exploit
more specific differences between \egamma and background objects. Three such variables are
used in the Phase-I system: $R_{\eta}$, $R_\mathrm{had}$, and $w_{s,\mathrm{tot}}$.

Choice of discriminating variables in Phase II is a work in progress. The study presented
in this report focuses on the use of \eratio as a discriminating variable in the GEP.
\eratio compares the energy of the secondary maximum in a cluster to that of the primary
maximum. With the finer granularity data available in Phase II, particularly with the
strips in Layer 1 of the calorimeter, \eratio has potential to be a strong discriminant
against QCD background where it is common to have multiple peaks of activity separated by
a small distance. For the studies presented here, \eratio is defined as
%
\begin{equation}
  E_\mathsf{ratio} = \frac{E_2}{E_1},
  \label{eqn:eratio}
\end{equation}
%
where $E_1$ and $E_2$ are the energies of the primary and secondary maxima respectively.
