%%% PHASE-II SIMULATION SOFTWARE %%%

Producing prospective results for the Phase-II trigger requires simulation of
the requisite algorithms. On top of the existing simulations of the Phase-I
simulations, two things are needed to produce the results possible with Phase
II: collection of the higher granularity calorimeter data that will be available
to the \ac{GEP}, and any algorithms that the \ac{GEP} will run on its input
data.

The first of these tasks is done by taking the location of \egamma candidate
\ac{TOB}s identified by the simulated \ac{eFEX}, collecting calorimeter cell
energies in a region around this location, and storing it in a cluster. This
method was chosen over storing calorimeter cell output in the entire detector to
reduce computing requirements. The size of stored clusters is $0.3\times0.3$ in
$\eta\times\phi$, centred on the seed \ac{TOB} location, chosen conservatively
to be sufficiently large that it will contain all information required by any
algorithm.

Samples with these clusters of high-granularity calorimeter data included are
then used for developing prospective algorithms for the \ac{GEP}, explored in
detail in Section \ref{sec:trig-eratio-algorithm}.
