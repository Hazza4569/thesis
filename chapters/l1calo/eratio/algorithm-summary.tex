%%% Algorithm summary subsection %%%

Given the results presented in Section
\ref{sec:trig-eratio-algorithm}, the
most performant and resource-efficient algorithm for calculating \eratio in the
\ac{GEP} is the baseline algorithm established in Section
\ref{sec:trig-eratio-algorithm-initial} with an exclusion region of one cell and
a search limit of $\Delta\eta=0.1$. No further improvement was found by varying
the peak size. This algorithm achieves a background rejection of 3.1 for 95\%
signal efficiency.

%%% Further adjustments %%%

The \eratio algorithm presented here is functionally complete and serves as an
option for \egamma discrimination in the \ac{GEP}. Additionally, further improvements
could likely be made with more studies.
More parameters for this \eratio algorithm could be conceived and tested to
potentially improve performance.
One example is a threshold in the energy gradient between steps to allow a
change in gradient to be identified, which might improve the response of the
algorithm to noise or statistical fluctuations.

This study focused on the design of the \eratio algorithm, using background
rejection as a metric for performance. Due to technical limitations, the \ac{MC}
samples used do not represent the projected pileup conditions of \ac{ATLAS}
during the \ac{HL-LHC}.  Further study would be needed for a full evaluation of
the performance possible in these conditions, alongside other components of the
hardware trigger.

  %TODO
  % - more parameter options to test?
  % - discuss improved simulation (pileup, noise) as well as other parameters
  %   - this may need discussion beforehand to introduce the limitations of
  %     simulations used in this study.
