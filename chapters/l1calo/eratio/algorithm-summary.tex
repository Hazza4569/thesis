%%% Algorithm summary subsection %%%

Given the results presented in Sections
\ref{sec:trig-eratio-algorithm-peak}-\ref{sec:trig-eratio-algorithm-limit}, the
most performant and resource-efficient algorithm for calculating \eratio in the
\ac{GEP} is the baseline algorithm established in Section
\ref{sec:trig-eratio-algorithm-initial} with an exclusion region of one cell and
a search limit of $\Delta\eta=0.1$. No further improvement was found by varying
the peak size. This algorithm achieves a background rejection of 3.1 for 95\%
signal efficiency.

%%% Further adjustments %%%

The \eratio algorithm presented here is functionally complete and serves as an
option for \egamma discrimination in the \ac{GEP}, but further improvements
could be made with additional study.
More parameters could be varied and tested for performance against simulations.
One example is a threshold in the energy gradient between steps to allow a
change in gradient to be identified, which might improve the response of the
algorithm to noise or statistical fluctuations.

  %TODO
  % - more parameter options to test?
  % - discuss improved simulation (pileup, noise) as well as other parameters
  %   - this may need discussion beforehand to introduce the limitations of
  %     simulations used in this study.
