%%% Search limit subsubsection %%%

In the baseline \eratio algorithm, the stepwise search for secondary maxima
extends as far as the available data allows, in this case to the edge of the,
conservatively large, stored cluster size. To minimise the amount of processing
required by the algorithm, and potentially improve performance by reducing
overlap with other clusters, a limit can be placed on the distance this search
will traverse. Since the $\phi$ range of the search is already limited to one
cell either side of the peak, this search limit is implemented as a maximum
distance traversed in $\eta$. This distance is calculated in pseudorapidity
units rather than number of cells to give a consistent response across
calorimeter regions.

Performance for the \eratio algorithm with different search limit values was
tested on simulations, with the results presented in Figure
\ref{fig:trig-eratio-algorith-limit-result}. Distances in $\Delta\eta$ from
0.025 up to 0.15 were tested, with 0.15 being the width of the clusters and thus
the limit in place in the baseline algorithm. While no performance gains are
seen by reducing the search limit, there is a plateau in performance from
$\Delta\eta > 0.1$. This means the required cluster size, and thus the amount of
computation required, can be reduced without degrading performance of the
algorithm.

\begin{figure}
  \centering
  \includegraphics[width=\textwidth]{\resource{eratio-limit.pdf}}
  \caption{
    Results for calculating \eratio after varying the search limit parameter,
    given as a distance in $\eta$ from the seed cell.  Plots show background
    rejection as a function of signal efficiency for each tested search limit
    (left) and background rejection at 95\% signal efficiency as a function of
    the search limit (right).
  }
  \label{fig:trig-eratio-algorith-limit-result}
\end{figure}

%TODO what was the bug changing the peak value??????
