%%% Phase-II upgrade subsection %%%

As luminosity and pileup is increased even further with the high-luminosity era
of the \ac{LHC} in Run 4, the trigger again needs to be improved to operate in
increasingly difficult conditions. The Phase-II upgrade to the hardware trigger
aims to do this primarily by adding a new component, the \ac{GEP} (or Global
Trigger). The \ac{GEP} will be downstream of the Phase-I \ac{FEX} modules, which
will continue to contribute to the trigger, and it will refine decisions made by
employing additional information: information from a larger area than typically
available to a single \ac{FEX} and also finer in granularity.

% [P2 DIAGRAM]
\begin{figure}
  \centering
  \includegraphics[width=.95\textwidth]{\resource{p2-ey.pdf}}
  \caption{
    Schematic of the \acs{ATLAS} hardware trigger as planned for the Phase-II
    upgrade in Run 4 of the \acs{LHC}. The red lines highlight the main parts
    relevant to the \egamma trigger, with the addition of the Global Trigger
    being and the use of calorimeter cell information being the main changes
    with respect to the Phase-I system.
    \cite{ATLAS-TDR-TDAQ-PhaseII}
  }
  \label{fig:trig-upgrade-p2-diagram}
\end{figure}

An outline of how the \ac{GEP} fits in with the existing systems is shown in
Figure \ref{fig:trig-upgrade-p2-diagram}. Information from the calorimeters will
be sent directly to the \ac{GEP} in finer granularity than is available to the
\ac{eFEX}, with energies in each individual cell at the full detector
granularity. This gives a 4-8 times increase in granularity over SuperCells,
depending on the region of the calorimeter.

The additional information available to the \ac{GEP} means it can work together
with the eFEX to further refine the result. The \ac{eFEX} will create \acp{TOB}
with associated variables (discussed in Section \ref{sec:trig-upgrade-p1}) which
are sent to the \ac{GEP}.  The \ac{GEP} can then further probe the same region
of the calorimeter to determine if the candidate object should be accepted. The
algorithms used by the \ac{GEP} to do this are the topic of the study in Section
\ref{sec:trig-eratio}.
