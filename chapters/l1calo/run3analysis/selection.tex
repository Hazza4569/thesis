%% TOB and RoI selection subsection

Phase-I \acp{TOB} and Run-2 \acp{RoI} in events are compared to find instances
in the same event that have the same, or very similar, $\eta$-$\phi$
coordinates. 
A pair is formed by selecting, for each \ac{TOB}, the nearest \ac{RoI} that
has not already been matched to a \ac{TOB}.
A match is considered to be a pair of objects within $\pm$1 trigger
tower in both $\eta$ and $\phi$, i.e. an \ac{RoI} matches a \ac{TOB} if it falls
within the $3\times3$ area of trigger towers centred on the tower containing the
\ac{TOB}. Matched objects are considered to be the same physics object,
identified independently by both systems. 
%
Instances where there is a \ac{TOB} or \ac{RoI} with no analogue in the opposing
system are also tracked.
%

Only the barrel region was considered for this as a preliminary investigation,
since it has a simpler geometry and as such it is easier to isolate bugs. At the
time of analysing, only half of the eFEX modules were installed in the detector;
this was due to delays in production caused by the global semiconductor shortage
\cite{Sweney2021}. As a result, the Phase-I system at that time had coverage for
just half of the $\phi$ range. Therefore only \acp{RoI} inside of this coverage
are accepted.
