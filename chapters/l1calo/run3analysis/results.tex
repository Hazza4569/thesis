%% RESULTS SUBSECTION

From the 1,636,636 events in Run 423433, 292,498 \ac{RoI}/\ac{TOB} pairs are
selected. Of these, 271,854 matched in $\eta-\phi$ coordinates, giving a total
match rate of 93\%.
%
For Run 427885, 22,337 of 27,973 pairs were matched for a match rate of 80\%.
Figures \ref{fig:trig-r3anal-matchrate_CPM} and
\ref{fig:trig-r3anal-matchrate_eFEX} show the match rate for objects in Run
427885 as a function of energy, using \ac{CPM}-measured and \ac{eFEX}-measured
energies respectively. This shows that the bulk of the mismatches come from
low-energy objects, with a notably higher match rate at higher energies. At all
energies the match rate is worse here than in the earlier run.

\begin{figure}[tb]
  \centering
  \includegraphics[width=.88\textwidth]{\resource{matchrate_427885_CPM}}
  \caption{
    Match rate for \acsp{TOB}/\acsp{RoI} as a function of energy as measured by
    the \acs{CPM}. Objects grouped in 20 GeV bins, with the last bin including
    all overflow.
  }
  \label{fig:trig-r3anal-matchrate_CPM}
\end{figure}

\begin{figure}[bt]
  \centering
  \includegraphics[width=.88\textwidth]{\resource{matchrate_427885_eFEX}}
  \caption{
    Match rate for \acsp{TOB}/\acsp{RoI} as a function of energy as measured by
    the \acs{eFEX}. Objects grouped in 20 GeV bins, with the last bin including
    all overflow.
  }
  \label{fig:trig-r3anal-matchrate_eFEX}
\end{figure}

Figures \ref{fig:trig-r3anal-matches1} and \ref{fig:trig-r3anal-matches2}
compare the energies recorded by the Run-2 and the Phase-1 systems for matched
objects in the two runs. In Run 423433 it is clear that the majority of matched
objects have approximately the same energy, with an additional cluster where in
a few cases the \ac{eFEX}-measured energy is much lower than the
\ac{CPM}-measured.

\begin{figure}[tb]
  \centering
  \includegraphics[width=.88\textwidth]{\resource{matcheshist_423433}}
  \caption{
    Comparison of energies for matched \acsp{TOB}/\acsp{RoI} with the
    energy as measured by the \acs{CPM} given on the $x$-axis and as measured by
    the \acs{eFEX} on the $y$-axis. Contains data for all matched objects in Run
    423433. The dashed line marks the set of points where the \acs{CPM} and
    \acs{eFEX} energies are equal.
  }
  \label{fig:trig-r3anal-matches1}
\end{figure}

\begin{figure}[bt]
  \centering
  \includegraphics[width=.88\textwidth]{\resource{matcheshist_427885}}
  \caption{
    Comparison of energies for matched \acsp{TOB}/\acsp{RoI} with the
    energy as measured by the \acs{CPM} given on the $x$-axis and as measured by
    the \acs{eFEX} on the $y$-axis. Contains data for all matched objects in Run
    427885. The dashed line marks the set of points where the \acs{CPM} and
    \acs{eFEX} energies are equal.
  }
  \label{fig:trig-r3anal-matches2}
\end{figure}

In the later run, Run 427885, however, there is no longer such a strong
correlation in energies. It seems that in general the \ac{eFEX} energies are
lower than the \ac{CPM} energies -- seen by the gradient of the area containing
the majority of objects being less than the equal-energies line. Once again
there is another cluster of objects with very low \ac{eFEX} energies at high
\ac{CPM} energies.

The general trend is a high but imperfect match rate and decreased performance
in the later run compared to the earlier run, both in terms of match rate of
objects and energy correlation between the two systems.
%
From this information, issues in the system could be identified and solved.  In
the case of the degraded performance for Run 427885, the different beam
conditions in this run (bunch trains, that were not present for Run 423433) were
deemed to have caused issues with the \ac{BCID} on the \ac{LATOME} modules
providing the \ac{eFEX} with digitised energies from the calorimeter.

% TODO perhaps another subsection here with an update on the system and how
% these issues were solved
