%% RESULTS SUBSECTION

\begin{figure}[p]
  \centering
  \includegraphics[width=.88\textwidth]{\resource{matchrate_corrections_423433_CPM}} \\
  \includegraphics[width=.88\textwidth]{\resource{matchrate_corrections_423433_eFEX}}
  \caption{
    Match rate for \acsp{TOB}/\acsp{RoI} as a function of energy as measured by
    the \acs{CPM} (top) and \acs{eFEX} (bottom) for Run 423433. Objects grouped in 20 GeV bins, with the last bin including
    all overflow.
  }
  \label{fig:trig-r3anal-matchrate-1}
\end{figure}

\begin{figure}[p]
  \centering
  \includegraphics[width=.88\textwidth]{\resource{matchrate_corrections_427885_CPM}} \\
  \includegraphics[width=.88\textwidth]{\resource{matchrate_corrections_427885_eFEX}}
  \caption{
    Match rate for \acsp{TOB}/\acsp{RoI} as a function of energy as measured by
    the \acs{CPM} (top) and \acs{eFEX} (bottom) for Run 427885. Objects grouped in 20 GeV bins, with the last bin including
    all overflow.
  }
  \label{fig:trig-r3anal-matchrate-2}
\end{figure}

From the 1,636,636 events in Run 423433, 292,498 \ac{RoI}/\ac{TOB} pairs are
selected. Of these, 271,854 matched in $\eta-\phi$ coordinates, giving a total
match rate of 93\%.
%
For Run 427885, 22,337 of 27,973 pairs were matched for a match rate of 80\%.
Figures \ref{fig:trig-r3anal-matchrate-1} and \ref{fig:trig-r3anal-matchrate-2}
show the fraction of objects matched as a function of energy, with histograms
using both the \ac{CP}-measured and \ac{eFEX}-measured energies shown for each
run. Uncertainties on these match rates are due to statistical uncertainties in
the number of matched objects and the total number of objects, and are
calculated using the Clopper-Pearson interval \cite{Clopper1934} with a
confidence level of 68\%.

%
%shows the match rate for objects in Run
%427885 as a function of energy, using \ac{CP}-measured and \ac{eFEX}-measured
%energies.
For Run 427885 (Figure \ref{fig:trig-r3anal-matchrate-2}) it can be seen 
that the bulk of the mismatches come from
low-energy objects, with a plateau in match rate above $E_T > 20$ GeV.
This issue was not present in the earlier Run 423433 (Figure
\ref{fig:trig-r3anal-matchrate-1}), which shows a relatively consistent match
rate across all energies. This is reflected in the overall match rate, 
which is considerably lower in the later run.

Figure \ref{fig:trig-r3anal-matches}
compares the energies recorded by the Run-2 and the Phase-I systems for matched
objects in the two runs. In Run 423433 it is clear that the majority of matched
objects have approximately the same energy, with an additional cluster where in
a few cases the \ac{eFEX}-measured energy is much lower than the
\ac{CP}-measured.

\begin{figure}[p]
  \centering
  \includegraphics[width=.88\textwidth]{\resource{matcheshist_423433}} \\
  \includegraphics[width=.88\textwidth]{\resource{matcheshist_427885}}
  \caption{
    Comparison of energies for matched \acsp{TOB}/\acsp{RoI} with the
    energy as measured by the \acf{CPM} given on the $x$-axis and as measured by
    the \acs{eFEX} on the $y$-axis. Contains data for all matched objects in Run
    423433 (top) and 427885 (bottom). The dashed line marks the set of points where the \acs{CPM} and
    \acs{eFEX} energies are equal.
  }
  \label{fig:trig-r3anal-matches}
\end{figure}

In the later run, Run 427885, however, there is no longer such a strong
correlation in energies. It seems that in general the \ac{eFEX} energies are
lower than the \ac{CP} energies -- seen by the gradient of the area containing
the majority of objects being less than the equal-energies line. Once again
there is another cluster of objects with very low \ac{eFEX} energies at high
\ac{CP} energies.

The general trend is a high but imperfect match rate and decreased performance
in the later run compared to the earlier run, both in terms of match rate of
objects and energy correlation between the two systems.

From the information provided by this analysis, issues in the system were
identified and solved. In the case of the degraded performance for Run 427885,
the different beam conditions in this run (bunch trains, that were not present
for Run 423433) were understood to have caused issues with the \ac{BCID} on the
\ac{LATOME} modules which provide the \ac{eFEX} with digitised energies from the
calorimeter.

Many initial problems with the Phase-I \egamma trigger have now been understood
and fixed, in part thanks to the work presented here. The \ac{eFEX} is now in
use in the Run-3 trigger menu and performing better than the Run-2 system,
evidenced by the efficiency curves shown in Figure
\ref{fig:trig-r3anal-efficiency}. The consistency of the new and old systems
after fixes were implemented is shown by Figure \ref{fig:trig-r3anal-public},
showing the same \ac{TOB}-\ac{RoI} energy comparisons in a later run.

\begin{figure}[p]
  % 2in1 figure
  \centering
  \includegraphics[width=.92\textwidth]{\resource{efficiency.pdf}}
  \vspace{-.9cm}
  \caption{
    Comparison of single electron trigger efficiencies for the Run-2 and Phase-I
    \acs{L1Calo} triggers, as a function of electron $p_T$ (as recorded in
    offline reconstruction).
    \cite{L1Calopublicplots}
  }
  \label{fig:trig-r3anal-efficiency}
  \vspace{.3cm}
  %
  \includegraphics[width=.92\textwidth]{\resource{public.pdf}}
  \vspace{-.2cm}
  \caption{
    Comparison of transverse energies for matched \acs{EM} objects between the
    Run-2 \acs{CP} and the Phase-I \acs{eFEX}. Shown are matches between leading
    electrons satisfying $|\eta|<0.8$ in each event in Run 438532. The dashed
    line marks the set of points where the \acs{CP} and \acs{eFEX} energies are
    equal.
    \cite{L1Calopublicplots}
  }
  \label{fig:trig-r3anal-public}
\end{figure}
