%% RESULTS SUBSECTION

From the 1,636,636 events in Run 423433, 292,498 \ac{RoI}/\ac{TOB} pairs are
selected. Of these, 271,854 matched in $\eta-\phi$ coordinates, giving a total
match rate of 93\%.
%
For Run 427885, 22,337 of 27,973 pairs were matched for a match rate of 80\%.
Figures \ref{fig:trig-r3anal-matchrate_CPM} and
\ref{fig:trig-r3anal-matchrate_eFEX} show the match rate for objects in Run
427885 as a function of energy, using \ac{CPM}-measured and \ac{eFEX}-measured
energies respectively. This shows that the bulk of the mismatches come from
low-energy objects, with a notably higher match rate at higher energies. At all
energies though the match rate is worse here than in the earlier run.

\begin{figure}[tbph]
  \centering
  \includegraphics[width=.90\textwidth]{\resource{matchrate_427885_CPM}}
  \caption{
    Match rate for \ac{TOB}s/\ac{RoI}s as a function of energy as measured by
    the \ac{CPM}. Objects grouped in 20 GeV bins, with the last bin including
    all overflow.
  }
  \label{fig:trig-r3anal-matchrate_CPM}
\end{figure}

\begin{figure}[tbph]
  \centering
  \includegraphics[width=.90\textwidth]{\resource{matchrate_427885_eFEX}}
  \caption{
    Match rate for \ac{TOB}s/\ac{RoI}s as a function of energy as measured by
    the \ac{eFEX}. Objects grouped in 20 GeV bins, with the last bin including
    all overflow.
  }
  \label{fig:trig-r3anal-matchrate_eFEX}
\end{figure}

Figures \ref{fig:trig-r3anal-matches1} and \ref{fig:trig-r3anal-matches2}
compare the energies recorded by the legacy and the Phase-1 systems for matched
objects in the two runs. In Run 423433 it is clear that the majority of matched
objects have approximately the same energy, with an additional cluster where in
a few cases the \ac{eFEX}-measured energy is much lower than the
\ac{CPM}-measured.

\begin{figure}[tbph]
  \centering
  \includegraphics[width=.9\textwidth]{\resource{matcheshist_423433}}
  \caption{
    Comparison of energies for matched \ac{TOB}s/\ac{RoI}s with the
    energy as measured by the \ac{CPM} given on the $x$-axis and as measured by
    the \ac{eFEX} on the $y$-axis. Contains data for all matched objects in Run
    423433. The dashed line marks the set of points where the \ac{CPM} and
    \ac{eFEX} energies are equal.
  }
  \label{fig:trig-r3anal-matches1}
\end{figure}

\begin{figure}[tbph]
  \centering
  \includegraphics[width=.9\textwidth]{\resource{matcheshist_427885}}
  \caption{
    Comparison of energies for matched \ac{TOB}s/\ac{RoI}s with the
    energy as measured by the \ac{CPM} given on the $x$-axis and as measured by
    the \ac{eFEX} on the $y$-axis. Contains data for all matched objects in Run
    427885. The dashed line marks the set of points where the \ac{CPM} and
    \ac{eFEX} energies are equal.
  }
  \label{fig:trig-r3anal-matches2}
\end{figure}

In the later run, Run 427885, however, there is no longer such a strong
correlation in energies. It seems that in general the \ac{eFEX} energies are
measured to be some fraction of the \ac{CPM} energies -- seen by the gradient
of the area containing the majority of objects being less than the
equal-energies line. Once again there is another cluster of objects with very
low \ac{eFEX} energies at high \ac{CPM} energies.

So the general trends are a high but imperfect match rate; and decreased
performance in the later run compared to the earlier run, both in terms of match
rate of objects and energy correlation between the two systems. This information
was fed back to people working on the systems to help them track down the root
issues. Likely the reason for degraded performance in the later run, despite
constant improvements to the system in the time between the two runs, is due to
the different beam conditions. For example, the bunch trains present in Run
427885 (but not Run 423433) could have caused issues with \ac{BCID} on
the \ac{LATOME} modules feeding information to the \ac{eFEX}.

[PERHAPS ANOTHER SUBSECTION HERE WITH AN UPDATE ON THE SYSTEM AND HOW THESE
ISSUES WERE SOLVED]
