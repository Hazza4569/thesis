%% ALGORITHMS SECTION
For each \ac{TOB} processed (i.e. each particular event, coordinate location,
and seed), several algorithms are run to calculate the quantities displayed
on-screen. These are the same algorithms used by the eFEX to calculate \ac{TOB}
energies and isolations. The following variables are calculated: EM cluster
energy, EM \reta, EM \rhad, EM \wstot, tau cluster energy,
tau \reta, and tau \rhad. All of these are either sums of SuperCell
energies (EM and tau cluster energies), ratios of sums of SuperCell energies
(\reta and \rhad), or a ratio with weighted sums (\wstot).

The values of these variables are calculated immediately after the data for a
given \ac{TOB} is collected, or once the seed is re-specified, and displayed
on-screen below the grid. If the user clicks on a displayed quantity, the
SuperCells involved in the sums for the corresponding algorithm are highlighted
with colours corresponding to whether those cells are used in the numerator
(lime green); the denominator (gold); or, in the case of \wstot, in the
numerator with a larger weight (dark green). Figures
\ref{fig:trig-vis-highlights-tau} and \ref{fig:trig-vis-highlights-em}
demonstrate the highlighting for all of the algorithms.

\begin{figure}[tbp]
  \centering
  %
  \begin{subfigure}{.44\textwidth}
    \fbox{\includegraphics[width=\textwidth,trim=0 0 0 5px,clip]{\resource{highlight_none}}}
    \caption{}
  \end{subfigure}
  \hfill
  \begin{subfigure}{.44\textwidth}
    \fbox{\includegraphics[width=\textwidth,trim=0 0 0 5px,clip]{\resource{highlight_tau_cluster}}}
    \caption{}
  \end{subfigure}
  \\[1em]%
  \begin{subfigure}{.44\textwidth}
    \fbox{\includegraphics[width=\textwidth,trim=0 0 0 5px,clip]{\resource{highlight_tau_reta}}}
    \caption{}
  \end{subfigure}
  \hfill
  \begin{subfigure}{.44\textwidth}
    \fbox{\includegraphics[width=\textwidth,trim=0 0 0 5px,clip]{\resource{highlight_tau_rhad}}}
    \caption{}
  \end{subfigure}
  %
  \caption{
    Demonstration of highlighting used to visualise algorithms, all shown
    for the same \acs{TOB}. Showing (a) initial view without highlighting, (b)
    highlighting for tau cluster energy, (c) highlighting for tau \reta, and (d)
    highlighting for tau \rhad.
  }
  \label{fig:trig-vis-highlights-tau}
\end{figure}

\begin{figure}[tbph]
  \centering
  %
  \begin{subfigure}{.44\textwidth}
    \fbox{\includegraphics[width=\textwidth,trim=0 0 0 5px,clip]{\resource{highlight_em_cluster}}}
    \caption{}
  \end{subfigure}
  \hfill
  \begin{subfigure}{.44\textwidth}
    \fbox{\includegraphics[width=\textwidth,trim=0 0 0 5px,clip]{\resource{highlight_em_reta}}}
    \caption{}
  \end{subfigure}
  \\[1em]%
  \begin{subfigure}{.44\textwidth}
    \fbox{\includegraphics[width=\textwidth,trim=0 0 0 5px,clip]{\resource{highlight_em_rhad}}}
    \caption{}
  \end{subfigure}
  \hfill
  \begin{subfigure}{.44\textwidth}
    \fbox{\includegraphics[width=\textwidth,trim=0 0 0 5px,clip]{\resource{highlight_em_wstot}}}
    \caption{}
  \end{subfigure}
  %
  \caption{
    Demonstration of highlighting used to visualise algorithms, all shown
    for the same \acs{TOB}. Showing (a) highlighting for EM cluster energy,
    (b) highlighting for EM \reta, (c) highlighting for EM \rhad, and (d)
    highlighting for EM \wstot.
  }
  \label{fig:trig-vis-highlights-em}
\end{figure}
