%%% USER INTERFACE SUBSECTION %%%
The \ac{eFEX} Visualiser program provides a simple \ac{UI} to explore input
data and results of the \ac{eFEX} algorithms. The basic interface is shown in
Figure \ref{fig:trig-vis-initial}.  It prompts the user to specify an input
file, $(\eta,\phi)$ centre-tower coordinates, and an event number, then on
receipt of these inputs it reads the information and displays the requested
energies in a grid.

\begin{figure}[tbph]
  \centering
  \fbox{
    \includegraphics[width=.9\textwidth,trim=0 850px 0 5px,clip]{\resource{default_view_blank}}
  }
  \\[1em]
  \fbox{
    \includegraphics[width=.9\textwidth,trim=0 0 0 5px,clip]{\resource{default_view_filled}}
  }
  \caption{
    Initial interface on launching the \acs{eFEX} Visualiser program (top) and the default
    view once data is read from a file (bottom).
  }
  \label{fig:trig-vis-initial}
\end{figure}

The full interface becomes visible after the grid is displayed. The grid itself
is a $3\times3$ area divided by bold lines, with each segment representing a
trigger tower, and each trigger tower square divided further into SuperCells.
%
The horizontal axis represents the $\eta$ coordinate of the tower or SuperCell,
and the vertical axis represents the $\phi$ coordinate. These coordinates are
labelled with the same indices the user gave as initial input.
%
In order to show all layers simultaneously, in the default view layers are
stacked (in the $\phi$-axis) on top of each other within each tower. Controls are
provided to instead view each layer individually if preferred.

Below the grid, a list of all the quantities calculated for the current \ac{TOB}
is displayed. Clicking on one of these quantities will highlight all of the
SuperCells involved in the calculation.  The details of how these values are
calculated and how the algorithms are visualised are discussed in more detail in
Section \ref{sec:trig-vis-algorithms}.

Above the grid, alongside the layer selection buttons, are options to manually
set the seed SuperCell and to select the units used to display energies.
The unit selection input is a drop-down box that allows the user to
choose between 25 MeV (default units in firmware) or GeV units. Changing this
option instantly updates all displayed energies.
%
Pressing the ``Select seed'' button will toggle the layer view to display Layer
2, prompt the user to click on the SuperCell with the highest energy, and then
on its $\phi$-neighbour with the highest energy. This aids the user in selecting
the correct seed for \ac{TOB} generation, but is not normally necessary as the
program will apply these criteria to automatically set the seed as soon as the
grid is loaded. The manual override is included in case the automatic selection
is wrong, or if looking at algorithms with a different seed may help debugging.
