This thesis has presented research work carried out between September 2019 and
December 2023, on both upgrades to the \ac{ATLAS} \ac{L1Calo} trigger and
analysis of data in search of rare \ac{EW} \ac{SM} processes in the \Zyjj final
state.

The presented developments to the \ac{L1Calo} trigger will, alongside the work
done by the rest of the \ac{L1Calo} community, improve the amount of data
\ac{ATLAS} is able to record across two phases of trigger upgrades, 
lasting for more than a decade. The \ac{eFEX} visualisation tool has already
been used to highlight bugs in the Run-3 system, which were subsequently
corrected. The early Run-3 data
analysis contributed to the fine-tuning of the \egamma trigger. Meanwhile, the
\eratio algorithm development and performance studies establish an algorithm
available for the next iteration of the Level-1 \egamma trigger.

Together with an analysis team, the \ac{VBS} \Zy analysis presented in Chapter
\ref{sec:vbs} was able to observe the targeted process with a significance
greatly exceeding five standard deviations. The fiducial cross section of this
process is measured as
\begin{equation*}
  \sigma_\text{EW} = 4.49 \pm 0.40 \, (\text{stat.}) \pm 0.42 \, (\text{syst.})
  ~\text{fb}.
\end{equation*}
This process had not been previously
observed by \ac{ATLAS} and contributes to a programme of \ac{VBS} studies that
probe the \ac{SM} multiboson interactions.

Finally, the semileptonic \VZy analysis produced a measurement for the signal
strength of this relatively unexplored process, presenting an upper limit on its
production at 3.5 times the \ac{SM}-predicted rate, at the 95\% confidence
level. This analysis tackled a difficult phase space with very large
backgrounds, but also provides strong indication of the feasibility of future
studies of the process. Finding evidence or an observation with a combined Run-2
and Run-3 dataset is expected to be very challenging, but with a re-optimised
analysis and additional channels, and perhaps some upgraded analysis techniques,
it could be possible.
