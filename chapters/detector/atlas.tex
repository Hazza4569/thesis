% ========== THE ATLAS DETECTOR SECTION ==========

% Overview
\acsfirst{ATLAS} (\acl{ATLAS}) is one of the four large detectors built around interaction
points at the \ac{LHC}. The \ac{ATLAS} detector is the largest of the four, and
designed as a general-purpose detector to measure as many different processes as
possible. In order to do this, it surrounds the interaction point almost
entirely. Full angular acceptance would in principle allow any event to be fully
reconstructed from its detected decay products, the goal is to get as close to
this as is reasonably possible. The detector itself is built from
several sub-detectors, each dedicated to measuring specific properties of
particles, aided by a system of superconducting magnets.
Each of these sub-detectors is discussed in detail below.
Figure \ref{fig:detector-atlas-atlas} gives an overview of the \ac{ATLAS}
detector and its components.

\begin{figure}[tb]
  \centering
  \includegraphics[width=\textwidth]{\resource{atlas-schematic.pdf}}
  \caption{
    Cut-away view of the \acs{ATLAS} detector. Dimensions and components of the
    detector are labelled. \cite{ATLASpaper}
  }
  \label{fig:detector-atlas-atlas}
\end{figure}

\subsection{Coordinate system}

A specific coordinate system is used to describe the \ac{ATLAS} detector and
interactions within it. The $z$-axis runs along the beamline, the $x$-axis
points, in the positive direction, towards the centre of the \ac{LHC} ring, and
the $y$-axis points vertically upwards. The azimuthal angle, $\phi$, is measured
around the beamline in the $x$-$y$ plane and the polar angle, $\theta$, is
measured from the beam axis. A pseudorapidity coordinate, $\eta$, is defined as
\begin{equation*}
  \eta = -\ln\tan(\theta/2).
\end{equation*}
Transverse momentum, $p_T$, and transverse energy, $E_T$, are used to define the
momentum or energy in the $x$-$y$ plane, transverse to the beamline.

The set of coordinates $p_T$, $\eta$, and $\phi$ are typically preferred to
describe the kinematics of objects in the detector as they are invariant under
Lorentz boosts along the beamline\footnote{
  Pseudorapidity itself is not invariant, but is approximately equal to rapidity
  (in the relativistic limit). Rapidity transforms additively under
  such Lorentz boosts and so differences in rapidity are Lorentz invariant.
}.
Without this Lorentz invariance, differences in kinematics between events could
be introduced depending on the relative momenta of the colliding partons.

Angular differences between objects are typically expressed in terms of the
quantity $\Delta R = \sqrt{\Delta\eta^2 + \Delta\phi^2}$.

% Subdetectors:
\subsection{Inner detector}

The innermost detector system, known as the \ac{ID} or tracker, contains three
sub-detectors designed to track the location of charged particles, to measure
their momentum based on the curvature of their tracks. This is enabled by a
solenoid magnet which surrounds the \ac{ID}, generating a 2 T magnetic field
coaxial with the beam direction in
order to bend the tracks of charged particles travelling through.

\begin{figure}[tb]
  \centering
  \includegraphics[width=.9\textwidth]{\resource{tracker-schematic.jpg}}
  \caption{
    Diagram showing the components of the inner detector. The IBL label
    represents the insertable B-layer, the innermost part of the pixel detector
    which was added between Run 1 and Run 2.  The $r$ values label radial
    distances from the centre of the beam pipe. \cite{ATLAS2017densetracking}
  }
  \label{fig:detector-atlas-tracker}
\end{figure}

% hits and tracks, general?
As particles travel through the components of the \ac{ID}, `hits' are registered
for each location where the particle is detected. Hits across the tracker are
fitted to reconstruct the track of the particle. The momentum of this particle
is calculated from the radial arc of this track, and the sign of its charge is
deduced from the direction of the curve. Extrapolating the track towards its
origin allows it to be associated with a specific collision vertex location on
the beamline.

% ref figure
Figure \ref{fig:detector-atlas-tracker} shows a cross section of a sector of the
\ac{ID}; the sub-detectors of which the tracker is composed are shown.
From the beamline outwards, these are: the pixel detector,
the \ac{SCT}, and the \ac{TRT}. Each of these are detailed below. The pixel and
\ac{SCT} both cover an acceptance of $|\eta| < 2.5$ while the \ac{TRT} has an
acceptance of $|\eta| < 2$.

%Pixel
The closest component to the beamline is the pixel detector. The pixel detector
is designed to measure particles as close to the beamline as possible, with very
high granularity and precision. The detector is made up of 1968 silicon sensor
modules, with a combined total of $8.6\times10^7$ pixels across all sensors
\cite{IBLTDR}. The nominal pixel size is $50\times400~\mu$m (in $r\phi\times z$)
and 250 $\mu$m thick, with some variation in different regions
\cite{ATLASpaper}. As a charged particle passes through a pixel on the sensors
it ionises the atoms in the silicon, creating electron-hole pairs; these charges
are collected to generate a signal indicating a hit in that pixel.

%SCT
The next component out, along the path of a particle, is the \ac{SCT}.
The \ac{SCT} uses silicon strip sensors, which operate on the same principles as
the pixels. These strips have typical dimensions of $80\mu\text{m}\times6.4$cm
with a thickness of 285 $\mu$m \cite{ATLASpaper}. The \ac{SCT} consists of four
layers, where each layer has two sets of strips back-to-back with a relative
angle of 40 mrad between the strips. The rotation between strips within a layer
improves resolution along the long axis of the strip. In the barrel strips are
placed with their long axis parallel to the beamline (in the $z$-direction) and
in the end-caps strips are placed in the $r$-direction.

%TRT
The last \ac{ID} component encountered by incident particles is the \ac{TRT}.
The \ac{TRT} is composed of $3.7\times10^5$ straw detectors, with a diameter of
4 mm and a length of 144 cm (barrel region) or 37 cm (end-cap region).  In the
barrel region straws are placed parallel to the beamline. In the end-cap region
straws are arranged radially in wheels.  The straw detectors contain a
gold-plated tungsten wire surrounded by a xenon-carbon dioxide-oxygen gas
mixture. The space between straws is filled with a polymer fibre.
Charged particles passing through a straw can ionise the gas and generate a
readout on the wire to give a hit. Additionally, charged
particles crossing the boundaries between materials emit transition radiation,
dependent on their $\gamma = E/m$. This transition radiation ionises xenon atoms in
the gas mixture and gives a larger readout on the wire. 
The \ac{TRT} therefore provides hits and also information on the $E/m$ ratio of
incident particles; this is used for particle identifications, particularly for
electrons.

%  > momentum resolution?

\subsection{Calorimeters}

The \ac{ATLAS} detector has two distinct calorimeter systems: the \ac{LAr}
calorimeter and the tile hadronic calorimeter. These are both sampling
calorimeters, employing alternating absorbing and active layers to induce and
measure the energies of \ac{EM} and hadronic showers, respectively. Showers can
be reconstructed from a `cluster' of energy deposits in the calorimeter,
adjacent cells giving energy readouts to indicate that a particle deposited
energy there.  Figure \ref{fig:detector-atlas-calo} shows the location of the
calorimeter components in the context of the detector.

\begin{figure}[htbp]
  \centering
  \includegraphics[width=.95\textwidth]{\resource{calo-schematic.jpg}}
  \caption{Cut-away view of the \acs{ATLAS} calorimeters, with each component
  labelled. \cite{Pequenao2008a}}
  \label{fig:detector-atlas-calo}
\end{figure}

% LAr
The \ac{LAr} calorimeter has four components, the barrel, the \ac{EM} end-cap
(\acsfirst{EMEC}), the hadronic end-cap (\acsfirst{HEC}), and the forward
calorimeter (\acsfirst{FCal}).
All of these components use liquid argon in the active layers, where low-energy
shower particles will ionise argon atoms and produce a charge which is collected
in order to measure the deposited energy. The barrel and \ac{EMEC} use lead for
the absorbing layers, the \ac{HEC} uses copper absorbers, and the \ac{FCal} has
a combination of copper and tungsten.

% Barrel and EMEC
The \ac{LAr} barrel and \ac{EMEC} each have three layers of calorimeter cells of
differing sizes. These sizes vary by region, but Figure
\ref{fig:detector-atlas-accordion} shows the layout in the centre of the barrel.
There is also an additional `presampler' layer in front of these three layers,
to correct for energy loss due to material in front of the calorimeter.
The full specification of the calorimeter cell granularity is given in Reference
\cite[p.9]{ATLASpaper}.

\begin{figure}[htbp]
  \centering
  \includegraphics[width=.8\textwidth]{\resource{accordion.pdf}}
  \caption{
    Diagram showing layout of calorimeter cells in the barrel of the \acs{LAr}
    calorimeter. The $X_0$ units measure \ac{EM} radiation lengths. \cite{ATLASpaper}
  }
  \label{fig:detector-atlas-accordion}
\end{figure}

The first of these calorimeter layers in the barrel for $|\eta|<1.4$ consists of
strips of cells with very high $\eta$ granularity. This layer should be at or
near the start of \ac{EM} showers induced by incident particles, and provides more precise
determination of the shape of this shower.
This layer continues into the \ac{EMEC} for higher $|\eta|$ values but with
decreasing $\eta$-granularity.
This strip layer is an important part of the studies presented in Section
\ref{sec:trig-eratio}. The second layer covers most of the radial depth of the
calorimeter, and is designed to contain and measure most of the energy of an
\ac{EM} shower. The third layer provides additional measurements useful for
high-energy \ac{EM} objects or to help reject against hadronic showers.

%HEC FCal
The \ac{HEC} and \ac{FCal} provide forward coverage for measuring hadronic
showers, and the \ac{FCal} also extends the coverage for \ac{EM} showers. The
\ac{HEC} consists of two wheels per end-cap, with each wheel split into two
longitudinal sections. This gives the \ac{HEC} four detection layers, with a
granularity of between $0.1\times0.1$ and $0.2\times0.2$ (in $\eta\times\phi$)
across its coverage.  The \ac{FCal} consists of three modules. The innermost
module is for \ac{EM} objects and uses copper absorbers around the active liquid
argon layers. The remaining two modules of the \ac{FCal} extend the hadronic
coverage and use tungsten absorbers.

% EM Coverage
The barrel and \ac{EMEC} combined give coverage for \ac{EM} objects over $|\eta| <
3.2$. Combined with the inner \ac{FCal} module, the full coverage of the \ac{EM}
calorimetry is $|\eta| < 4.9$.

% Tile cal
The tile hadronic calorimeter uses scintillating plastic tiles for the active
layers and steel absorbing layers. The tile barrel and extended barrel combined
give coverage over $|\eta| < 1.7$. The barrel and extended barrel are each divided
into three layers longitudinally, and have a total thickness of 7.4 hadronic
interaction lengths. 
Between the tile, \ac{HEC}, and \ac{FCal}, hadronic calorimetry acceptance is
$|\eta| < 4.9$.
% Probably more to say?

\subsection{Muon spectrometer}

% TODO explain MS station somewhere
The outermost component of the \ac{ATLAS} detector is the \ac{MS}. Muons will
typically generate tracks in the \ac{ID} which would already allow their momenta
to be measured. Adding additional tracking for muons in the outer part of the
detector allows for rejection against decay-in-flight backgrounds, where a
hadron might leave a track in the \ac{ID} and then decay to a muon, but this
muon can be rejected if the momentum measurement in the \ac{MS} is incompatible
with the original \ac{ID} measurement.  The \ac{MS} also improves resolution of
muon momentum measurements, and allows enhancements to identification and
triggering for muons. The acceptance of the \ac{MS} is $|\eta| < 2.7$.

A large toroidal magnet system is used to bend the tracks of muons passing
through the \ac{MS}, allowing tracking systems to measure their momentum.
This is handled by a barrel toroid, which wraps around the barrel \ac{MS} systems
to create a 0.5 T toroidal magnetic field, and two end-cap toroids, which
generate a 1 T toroidal magnetic field.

Tracking information is primarily provided by the \ac{MDT} and \ac{CSC} systems.
The \ac{RPC} and \ac{TGCm} detectors give additional tracking and also provide
triggering capabilities. These detectors are arranged into layers called
`stations', with three stations stacked radially in the barrel region and three
stations along the $z$-axis in each end-cap.  A schematic of the full \ac{MS}
system is shown in Figure \ref{fig:detector-atlas-ms}.

\begin{figure}[htbp]
  \centering
  \includegraphics[width=.9\textwidth]{\resource{ms-schematic.jpg}}
  \caption{
    Cut-away view of the \acs{ATLAS} detector with muon spectrometer components
    labelled.
    \cite{Pequenao2008b}
  }
  \label{fig:detector-atlas-ms}
\end{figure}

The \acp{MDT} provide tracking across the full acceptance of the \ac{MS}, with
multiple layers of \acp{MDT} in both the barrel and the end-caps. These function
similarly to the straw detectors of the \ac{TRT}, but with a larger diameter of
30 mm, containing a tungsten-rhenium wire surrounded by an
argon-methane-nitrogen gas mixture.

In the most radiation-prone region, the inner section of the most central
end-cap layers, \acp{CSC} are used in place of \acp{MDT}. The \acp{CSC} cover
pseudorapidities $2 < |\eta| < 2.7$. These are multiwire proportional counters
with cathodes segmented into strips, and benefit from higher granularity than
the \acp{MDT}.

%375
The role of the \acp{MDT} and \acp{CSC} is to provide precision measurements of
muon coordinates in the `bending' plane of the magnets, i.e. the $z$ or $\eta$
coordinate. They achieve this with a precision of $<100~\mu$m \cite{ATLASmuonTDR}.

The \acp{RPC} are placed in the barrel region and consists of two resistive plates
separated by a 2 mm gas-filled gap. In the end-cap region, \acp{TGCm} are used;
these are similar to \acp{CSC} but designed with a faster readout suitable for
triggering. Both the \acp{RPC} and \acp{TGCm} are used to give real-time readout
of track information for the Level-1 trigger. They also give a measurement of the
`second coordinate', the coordinate orthogonal that measured by the \acp{MDT}
and \acp{CSC}. This gives the $\phi$ coordinate with a spacial precision of
5-10 mm \cite{ATLASmuonTDR}.

\subsection{Trigger and data acquisition}

Running at maximum capacity, the \ac{ATLAS} detector measures a bunch crossing every 25
ns, i.e. a rate of 40 MHz, and in the majority of collisions no `interesting'
physics happens. 
There is no realistic way to read out the data from every
single bunch crossing at this rate, but even if this was possible it would create an
impossibly large storage requirement for events that will probably never be used for
physics studies.

The trigger is the solution to this problem; events are quickly processed to
determine if they have any signatures that might indicate the presence of
interesting physics processes. This is done in two stages. First, a low-level hardware
trigger (the Level-1 trigger) to make very fast but loose selection on events,
using coarse granularity information from a subset of detectors,
reducing the input rate to at most 100 kHz. Then a high-level trigger
(\acsfirst{HLT}) which uses more information and more complex reconstruction to
reduce the rate further down to a few kHz.

To prescribe how events may be accepted and to manage the output rate, a trigger
`menu' is used; this gives the set of requirements for events to pass the
trigger in a given run, for both Level 1 and the \ac{HLT}.

The trigger works in tandem with the data acquisition system, which is responsible for
reading out events passing trigger selections. This is done with front-end
hardware read-out devices that collect the detector information and process it
after receiving accept signals from the trigger systems. The entire trigger and
data acquisition (\acsfirst{TDAQ}) system is summarised in Figure
\ref{fig:detector-atlas-tdaq-tdaq}.

\begin{figure}[htbp]
  \centering
  \includegraphics[width=.9\textwidth]{\resource{tdaq.pdf}}
  \caption{
    Diagram showing components and data-flow of the \acs{ATLAS} Run-2 \acs{TDAQ}
    system. This diagram includes the `Fast TracKer' (FTK) component, but it was
    never implemented.
    \cite{ATLAStrigops2020}
  }
  \label{fig:detector-atlas-tdaq-tdaq}
\end{figure}

\subsubsection{Level-1 trigger}
\label{sec:detector-atlas-tdaq-l1}

The Level-1 trigger system is built of four main components: \ac{L1Calo}, L1Muon,
L1Topo, and the \ac{CTP}. Each of these are built from bespoke hardware modules designed to
perform the necessary algorithms as quickly as possible to keep up with the rate
of input data, running on \acp{FPGA} to ensure a fixed latency is used.

L1Calo takes input from the calorimeters to give triggers for \ac{EM} objects
(electrons and photons) and jets. From both the \ac{LAr} and tile calorimeters,
the energies in each trigger tower (a $0.1\times0.1$ area in $\eta\times\phi$)
are sent as analogue sums to a \ac{PPM}. The \ac{PPM} digitises and sends these energies to two modules: the
\ac{CP} and \ac{JEP}. The \ac{CP} analyses a $4\times4$ area of trigger towers
to calculate energies and isolations for \egamma or $\tau$ candidates. The
\ac{JEP} employs a similar process over larger areas and with lower granularity,
in order to calculate energies of jet candidates and estimate missing transverse
energy. For each event, L1Calo sends a set of `threshold bits' to the \ac{CTP}
indicating multiplicities and energies of different objects with respect to
trigger menu thresholds \cite{Achenbach2008}.

L1Muon uses tracking information from the \acp{RPC} and \acp{TGC} to make a rough
estimate of the transverse momentum of muons. Any set of two or more hits that are
consistent with a track originating at the interaction point are considered as
candidate muons. Muon candidates are sent to \ac{CTP} to contribute to the
trigger decision for the event.

Taking input from both L1Calo and L1Muon, L1Topo calculates topological
variables with a more holistic view of the event. These calculations include
quantities such as invariant masses of, or angular separation between, multiple
objects, and allow to trigger on more complex signatures.

The \ac{CTP} takes inputs from L1Calo, L1Muon, and L1Topo and, based on the
trigger menu, decides whether an event should be accepted. If an event passes
the checks, a `Level-1 accept' signal is sent to indicate to the data
acquisition system that this event should be read out and sent to the \ac{HLT}.
For events passing the Level-1 trigger threshold, \acp{RoI} are passed to the
\ac{HLT} in order to seed more complex trigger calculations.

\subsubsection{High-level trigger}

The \ac{HLT} runs algorithms in software on a dedicated server farm, afforded
looser timing constraints due the reduced input rate from the Level-1 trigger
and as such can run without fixed latency.
Algorithms are grouped into `chains', with each chain seeded by a Level-1
\ac{RoI}. Algorithms that require less processing time are typically run earlier
in the chain to enable faster rejection of bad events.

Events passing the \ac{HLT} selection are organised into `streams' where each stream contains events that pass
a set of trigger chains. These streams give events that pass the trigger
and are saved for offline processing. The main stream (\verb|physics_Main|)
consists of events passing the trigger menu intended for physics analyses. A subset of accepted events are
sent to an express stream which is sent for immediate offline reconstruction to
test data quality and allow calibration updates.

% Luminosity measurement
\subsection{Luminosity and pileup}

Given the relationship between the cross section of a physics process and the
luminosity of a dataset ($\sigma = N/L$), any cross-section measurement is
dependent on the measured luminosity, and its precision.
There are multiple methods of luminosity measurement employed by \ac{ATLAS},
this includes the use of the LUCID detector \cite{Avoni2018}, designed for the
sole purpose of making such measurements.

\begin{figure}[htb]
  \centering
  \includegraphics[width=.95\textwidth]{\resource{intlumivstimeRun2DQall.pdf}}
  \caption{
    Integrated luminosity as a function of time for Run 2. Shown are the total
    luminosity delivered by the \acs{LHC}, the luminosity recorded by
    \acs{ATLAS}, and the amount satisfying requirements to be used for physics
    analyses.
    \cite{ATLASdq2020}
  }
  \label{fig:detector-atlas-lumi-lumi}
\end{figure}

The LUCID detector consists of two stations, each 17m along the beam pipe either side
of the interaction point. Each station uses a set of Cherenkov tubes to detect
protons displaced through inelastic scattering. The number of detected protons
should be proportional to the number of interactions per bunch crossing, on
average, and thus proportional to the integrated luminosity; this is calibrated
using van der Meer scans. For the Run-2 integrated luminosity, the combined
\ac{ATLAS} measurement (including measurements from LUCID) has an uncertainty of
0.8\% \cite{ATLASlumi2023}.

The total integrated luminosity recorded by \ac{ATLAS} throughout Run 2 is shown
in Figure \ref{fig:detector-atlas-lumi-lumi}. This marks the total luminosity
delivered by the \ac{LHC}, the amount of that recorded by \ac{ATLAS}, and the
amount which is `good for physics'. The differences in these amounts are due to
down-time in the detector or its subsystems. Events are only marked good for
physics if all systems were functional, within accepted tolerances, during
data-taking. The deficit in \ac{ATLAS} recorded luminosity from the total
delivered represents inefficiencies in data acquisition.

One of the factors impacting the instantaneous luminosity recorded by the
\ac{ATLAS} detector is the number of proton-proton interactions per bunch
crossing, denoted $\mu$. The expected value of $\mu$ can be increased or
decreased by adjusting the crossing angle of the beams. A greater number of
interactions per bunch crossing results in an increase in luminosity but also
gives an increase in pileup, since for every recorded event there are more
`background' collisions happening around it. The average number of interactions
per bunch crossing in \ac{ATLAS} throughout Run 2 was 33.7 \cite{ATLASdq2020},
although a range of values were used at different points as shown by Figure
\ref{fig:detector-atlas-muvalues}.

\begin{figure}[tb]
  \centering
  \includegraphics[width=.9\textwidth]{\resource{muvalues.pdf}}
  \caption{
    Distribution of the mean number of proton-proton interactions per bunch
    crossing throughout Run 2. The distributions for each individual year of
    running are shown, as well as the total representing the whole of Run 2.
    Averages across these periods are also given.
    \cite{ATLASdq2020}
  }
  \label{fig:detector-atlas-muvalues}
\end{figure}
