% ========== THE ATLAS DETECTOR SECTION ==========

% Overview
\acsfirst{ATLAS} (\acl{ATLAS}) is one of the four detectors built around interaction
points at the \ac{LHC}. The \ac{ATLAS} detector is the largest of the four, and
designed as a general-purpose detector to measure as many different processes as
possible. In order to do this, it surrounds the interaction point almost
entirely; full angular acceptance would allow any event to be fully
reconstructed from its detected decay products. The detector itself is built of
several sub-detectors, each dedicated to measuring specific properties of
particles. Each of these sub-detectors is discussed in detail below.
Figure \ref{fig:detector-atlas-atlas} gives an overview of the \ac{ATLAS}
detector and its components.

\begin{figure}[htbp]
  \centering
  \includegraphics[width=\textwidth]{\resource{atlas-schematic.pdf}}
  \caption{
    Cut-away view of the \acs{ATLAS} detector. Dimensions and components of the
    detector are labelled. \cite{ATLASpaper}
  }
  \label{fig:detector-atlas-atlas}
\end{figure}

\subsection{Coordinate system}
A specific coordinate system is used to describe the \ac{ATLAS} detector and
interactions within it. The $z$-axis runs along the beamline, the $x$-axis
points, in the positive direction, towards the centre of the \ac{LHC} ring, and
the $y$-axis points vertically upwards. The azimuthal angle, $\phi$, is measured
around the beamline in the $x$-$y$ plane and the polar angle, $theta$, is
measured from the beam axis. A pseudorapidity coordinate, $\eta$, is defined as
\begin{equation*}
  \eta = -\ln\tan(\theta/2).
\end{equation*}
Transverse momentum, $p_T$, and transverse energy, $E_T$, are used to define the
momentum or energy in the $x$-$y$ plane, transverse to the beamline.

The set of coordinates $p_T$, $\eta$, and $\phi$ are typically preferred to
describe the kinematics of objects in the detector as they are invariant under
Lorentz boosts along the beamline\footnote{
  Pseudorapidity itself is not invariant, but transforms additively under
  Lorentz boosts; differences in pseudorapidity are therefore Lorentz invariant.
}.
Without this Lorentz invariance, differences in kinematics between events could
be introduced depending on the relative momenta of the colliding partons.

Angular differences between objects are typically expressed in terms of the
quantity $\Delta R = \sqrt{\Delta\eta^2 + \Delta\phi^2}$.

% Subdetectors:
\subsection{Inner detector}

The innermost detector system, known as the \ac{ID} or tracker, contains three
sub-detectors designed to track the location of charged particles, to measure
their momentum based on the curvature of their tracks. This is enabled by a
solenoid magnet which surrounds the \ac{ID}, generating a 2 T magnetic field in
order to curve the tracks of charged particles travelling through.

\begin{figure}[htbp]
  \centering
  \includegraphics[width=.9\textwidth]{\resource{tracker-schematic.jpg}}
  \caption{
    Diagram showing the components of the inner detector. The IBL label
    represents the insertable B-layer, the innermost part of the pixel detector
    which was added between Run 1 and Run 2.  The $r$ values label radial
    distances from the centre of the beam pipe. \cite{ATLAS2017densetracking}
  }
  \label{fig:detector-atlas-tracker}
\end{figure}

% hits and tracks, general?
As particles travel through the components of the \ac{ID}, `hits' are registered
for each location where the particle is detected. Hits across the tracker are
fitted to reconstruct the track of the particle. The momentum of this particle
is calculated from the radial arc of this track, and the sign of its charge is
deduced from the direction of the curve. Extrapolating the track towards its
origin allows it to be associated with a specific collision vertex location on
the beamline.

% ref figure
Figure \ref{fig:detector-atlas-tracker} shows a cross-section of the \ac{ID} and
its sub-detectors. From the beamline outwards, these are: the pixel detector,
the \ac{SCT}, and the \ac{TRT}. Each of these are detailed below. The pixel and
\ac{SCT} both cover an acceptance of $|\eta| < 2.5$ while the \ac{TRT} has an
acceptance of $|\eta| < 2$.

%Pixel
The closest component to the beamline is the pixel detector. The pixel detector
is designed to measure particles as close to the beamline as possible, with very
high granularity and precision. The detector is made up of 1968 silicon sensor
modules, with a combined total of $8.6\times10^7$ pixels across all sensors
\cite{IBLTDR}. The nominal pixel size is $50\times400~\mu$m (in $r\phi\times z$)
and 250 $\mu$m thick, with some variation in different regions
\cite{ATLASpaper}. As a charged particle passes through a pixel on the sensors
it ionises the atoms in the silicon, creating electron-hole pairs; these charges
are collected to generate a signal indicating a hit in that pixel.

%SCT
The next component out, along the path of a particle, is the \ac{SCT}.
The \ac{SCT} uses silicon strip sensors, which operate on the same principles as
the pixels. These strips have typical dimensions of $80\mu\text{m}\times6.4$cm
with a thickness of 285 $\mu$m \cite{ATLASpaper}. The \ac{SCT} consists of four
layers, where each layer has two sets of strips back-to-back with a relative
angle of 40 mrad between the strips. The rotation between strips within a layer
improves resolution along the long axis of the strip. In the barrel strips are
placed with their long axis parallel to the beamline (in the $z$-direction) and
in the end-caps strips are placed in the $r$-direction.

%TRT
The last \ac{ID} component encountered by incident particles is the \ac{TRT}.
The \ac{TRT} is composed of $3.7\times10^5$ straw detectors, with a diameter of
4 mm and a length of 144 cm (barrel region) or 37 cm (end-cap region).  In the
barrel region straws are placed parallel to the beamline. In the end-cap region
straws are arranged radially in wheels.  The straw detectors contain a
gold-plated tungsten wire surrounded by a xenon-carbon dioxide-oxygen gas
mixture. The space between straws is filled with a polymer fibre. Charged
particles crossing the boundaries between materials emit transition radiation,
dependent on their $\gamma = E/m$, which is ionises atoms in the gas mixture and
gives a readout on the wire. This gives hits for tracking particles but also
provides information on the $E/m$ ratio of the incident particle; this is used
for identification of electrons.

%  > momentum resolution?

\subsection{Calorimeters}

The \ac{ATLAS} detector has two distinct calorimeter systems: the \ac{LAr}
calorimeter and the tile hadronic calorimeter. These are both sampling
calorimeters, employing alternating absorbing and active layers to induce
\ac{EM} or hadronic showers and measure the energies of shower products,
respectively. Figure \ref{fig:detector-atlas-calo} shows the location of the
calorimeter components in the context of the detector.

\begin{figure}[htbp]
  \centering
  \includegraphics[width=.95\textwidth]{\resource{calo-schematic.jpg}}
  \caption{Cut-away view of the \acs{ATLAS} calorimeters, with each component
  labelled. \cite{Pequenao2008a}}
  \label{fig:detector-atlas-calo}
\end{figure}

% LAr
The \ac{LAr} calorimeter has four components, the barrel, the \ac{EM} end-cap
(\acsfirst{EMEC}), the hadronic end-cap (\acsfirst{HEC}), and the forward
calorimeter (\acsfirst{FCal}).
All of these components use liquid argon in the active layers, where low-energy
shower particles will ionise argon atoms and produce a charge which is collected
in order to measure the deposited energy. The barrel and \ac{EMEC} use lead for
the absorbing layers, the \ac{HEC} uses copper absorbers, and the \ac{FCal} has
a combination of copper and tungsten.

The \ac{LAr} barrel and \ac{EMEC} each have three layers of calorimeter cells of
differing sizes. These sizes vary by region, but Figure
\ref{fig:detector-atlas-accordion} shows the layout in the centre of the barrel.
There is also an additional `presampler' layer in front of these three layers.
The full specification for calorimeter cell granularity is given in Reference
\cite[p.9]{ATLASpaper}.

\begin{figure}[htbp]
  \centering
  \includegraphics[width=.8\textwidth]{\resource{accordion.pdf}}
  \caption{
    Diagram showing layout of calorimeter cells in the barrel of the \acs{LAr}
    calorimeter. The $X_0$ units measure \ac{EM} radiation lengths. \cite{ATLASpaper}
  }
  \label{fig:detector-atlas-accordion}
\end{figure}

The first of these calorimeter layers in the barrel for $|\eta|<1.4$ consists of
strips of cells with very high $\eta$ granularity. This provides more precise
determination of the shape of \ac{EM} showers in the calorimeter.

% Tile cal
The tile hadronic calorimeter uses scintillating plastic tiles for the active
layers and steel absorbing layers. The tile barrel and extended barrel combined
give coverage of $|\eta| < 1.7$. The barrel and extended barrel are each divided
into three layers longitudinally, and have a total thickness of 7.4 hadronic
interaction lengths. 
Between the tile, \ac{HEC}, and \ac{FCal}, hadronic calorimetry acceptance is
$|\eta| < 4.9$.
% Probably more to say?

\subsection{Muon spectrometer}

The outermost component of the \ac{ATLAS} detector is the \ac{MS}.
Muons will likely pass through the calorimeters without significant energy loss
and, due to their larger mass, relying on the \ac{ID} alone is imprecise for
measuring their momentum and identifying them. The muon spectrometer solves this
problem, adding additional tracking for muons in an acceptance of $|\eta| <
2.7$.

A large toroidal magnet system is used to curve the tracks of muons passing
through the \ac{MS}, allowing tracking systems to measure their momentum.
Tracking information is primarily provided by the \ac{MDT} and \ac{CSC} systems.
The \ac{RPC} and \ac{TGCm} detectors add to this, but also provide
triggering capabilities. These systems are shown in the context of the detector
in Figure \ref{fig:detector-atlas-ms}.
% Muon figure

\begin{figure}[htbp]
  \centering
  \includegraphics[width=.9\textwidth]{\resource{ms-schematic.jpg}}
  \caption{
    Cut-away view of the \acs{ATLAS} detector with muon spectrometer components
    labelled.
    \cite{Pequenao2008b}
  }
  \label{fig:detector-atlas-ms}
\end{figure}

The \acp{MDT} provide tracking across the full acceptance of the \ac{MS}, with
multiple layers of \acp{MDT} in both the barrel and the end-caps. These function
similarly to the straw detectors of the \ac{TRT}, with a tungsten-rhenium wire
surrounded by an argon-methane-nitrogen gas mixture.

In the most radiation-prone region, the inner section of the most central
end-cap layers, \acp{CSC} are used in place of \ac{MDT}. The \acp{CSC} cover
pseudorapidities $2 < |\eta| < 2.7$. These are multiwire proportional counters
with cathodes segmented into strips, and benefit from higher granularity than
the \acp{MDT}.

The \acp{RPC} are placed in the barrel region and consists of two resistive plates
separated by a 2 mm gas-filled gap. In the end-cap region, \acp{TGCm} are used;
these are similar to \acp{CSC} but designed with a faster readout suitable for
triggering. Both the \ac{RPC} and \ac{TGCm} are used to give real-time readout
of track information for the trigger. They also give a measurement of the
`second coordinate', the coordinate orthogonal to what the \ac{MDT} or \ac{CSC}
are designed to measure in the same region.

\subsection{Trigger and data acquisition}

Running at maximum capacity, the \ac{ATLAS} detector measures an event every 25
ns, i.e. a rate of 40 MHz, and in the majority of events no `interesting'
physics happens. 
There is no realistic way to read out the data from every
single event at this rate, but even if this was possible it would create an
inflated storage requirement for events that will probably never be used for
physics studies.

The trigger is the solution to this problem; events are quickly processed to
determine if they have any signatures that might indicate the presence of
interesting physics processes. This is done in two stages: a low-level hardware
trigger (the Level-1 trigger) to make very fast but loose selection on events,
reducing the input rate to around 100 kHz, and then a high-level trigger
(\acsfirst{HLT}) which uses more information and more complex reconstruction to
reduce the rate further down to 1 kHz.

To adjust how events are accepted and to manage the output rate, a trigger
`menu' is used; this gives the set of requirements for events to pass the
trigger in a given run, for both Level 1 and the \ac{HLT}.

The trigger works in tandem with the data acquisition system, responsible for
reading out events passing trigger selections. This is done with front-end
hardware read-out devices that collect the detector information and process it
after receiving accept signals from the trigger systems. The entire trigger and
data acquisition (\acsfirst{TDAQ}) system is summarised in Figure
\ref{fig:detector-atlas-tdaq-tdaq}.

\begin{figure}[htbp]
  \centering
  \includegraphics[width=.9\textwidth]{\resource{tdaq.pdf}}
  \caption{
    Diagram showing components and data-flow of the \acs{ATLAS} Run-2 \acs{TDAQ}
    system. \cite{ATLAStrigops2020}
  }
  \label{fig:detector-atlas-tdaq-tdaq}
\end{figure}

\subsubsection{Level-1 trigger}

The Level-1 trigger system is built of three main components: L1Calo, L1Muon,
and L1Topo. Each of these are built from bespoke hardware modules designed to
perform the necessary algorithms as quickly as possible to keep up with the rate
of input data.

L1Calo takes input from the calorimeters to give triggers for \ac{EM} objects
and jets. From both the \ac{LAr} and tile calorimeters, the energies in each
trigger tower (a $0.1\times0.1$ area in $\eta\times\phi$) are sent to a
\ac{PPM}. The \ac{PPM} sends these energies to two modules: the \ac{CP} and
\ac{JEP}. The \ac{CP} analyses a $4\times4$ area of trigger towers to calculate
energies and isolations for \egamma or $\tau$ candidates. The \ac{JEP} employs a
similar process over larger areas and with lower granularity, in order to
calculate energies of jet candidates and estimate missing transverse energy. For
each event, L1Calo sends a set of `threshold bits' to the \ac{CTP} indicating
multiplicities and energies of different objects with respect to trigger menu
thresholds \cite{Achenbach2008}.

L1Muon uses tracking information from the \ac{RPC} and \ac{TGC} to make a rough
estimate of transverse momentum of muons. Any set of two or more hits that are
consistent with a track originating at the interaction point are considered as
candidate muons. Comparing the calculated $p_T$ of these muons to the menu
thresholds gives a trigger decision for the event.

Taking input from both L1Calo and L1Muon, L1Topo calculates topological
variables with a more holistic view of the event. These calculations include
quantities such as invariant masses of or angular separation between multiple
objects, and allow to trigger on more complex signatures.

For events passing the Level-1 trigger threshold, \acp{RoI} are passed to the
\ac{HLT} in order to seed more complex trigger calculations.

\subsubsection{High-level trigger}

The \ac{HLT} runs algorithms in software on a dedicated server farm, afforded
looser timing constraints due the reduced input rate from the Level-1 trigger.
Algorithms are grouped into `chains', with each chain seeded by a Level-1
\ac{RoI}. Algorithms that require less processing time are typically run earlier
in the chain to enable faster rejection of bad events.

Events are organised into `streams' where each stream contains events that pass
a set of trigger chains. These streams give events that will pass the trigger
and be saved for offline processing. The main stream (\verb|physics_Main|)
will consist of events passing the trigger menu. A subset of accepted events are
sent to an express stream which is sent for immediate offline reconstruction to
test data quality.

% Luminosity measurement
\subsection{Luminosity}

The luminosity recorded by the \ac{ATLAS} detector is measured to understand
what fraction of the delivered \ac{LHC} luminosity is recorded by the detector.
The precision of this measurement also has an impact on the precision of physics
measurements. There are multiple methods of luminosity measurement employed by
\ac{ATLAS} but the primary measurement is made with the LUCID detector
\cite{Avoni2018}.

\begin{figure}[htb]
  \centering
  \includegraphics[width=.95\textwidth]{\resource{intlumivstimeRun2DQall.pdf}}
  \caption{
    Integrated luminosity as a function of time for Run 2. Shown are the total
    luminosity delivered by the \acs{LHC}, the luminosity recorded by
    \ac{ATLAS}, and the amount satisfying requirements to be used for physics
    analyses.
    \cite{ATLASdq2020}
  }
  \label{fig:detector-atlas-lumi-lumi}
\end{figure}

The LUCID detector consists of two stations 17m along the beam pipe either side
of the interaction point. Each station uses a set of Cherenkov tubes to detect
protons displaced through inelastic scattering. The number of detected protons
should be proportional to the number of interactions per bunch crossing, on
average, and thus proportional to the integrated luminosity; this is
calibrated using van der Meer scans. For the Run-2 integrated luminosity, the
measurement from LUCID and other sources has an uncertainty of 1.7\%
\cite{ATLASlumi2019}.

The total integrated luminosity recorded by \ac{ATLAS} throughout Run 2 is shown
in Figure \ref{fig:detector-atlas-lumi-lumi}. This marks the total luminosity
delivered by the \ac{LHC}, the amount of that recorded by \ac{ATLAS}, and the
amount which is `good for physics'. The differences in these amounts are due to
down-time in the detector or its subsystems. Events are only marked good for
physics if all systems were functional, within accepted tolerances, during
data-taking. The deficit in \ac{ATLAS} recorded luminosity from the total
delivered represents inefficiencies in data acquisition.
