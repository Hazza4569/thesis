% initial note in chapter about discussing details for Run 2 (until section
% at the end that discusses upgrades)? Couldn't make this look good so added a
% footnote
% ========== THE LARGE HADRON COLLIDER SECTION ==========

The Large Hadron Collider (\acsfirst{LHC}) is a circular\footnote{
  Roughly circular, since the ring consists of alternating straight and curved
  sections.
}
particle collider, measuring 27km
in circumference, located at \ac{CERN} in Geneva, Switzerland
\cite{LHCVol1,LHCVol2,LHCVol3}.
As a successor to the \ac{LEP} \cite{Assmann2002},
the \ac{LHC} was designed to study higher energy systems than had previously
been accessible in controlled, high-rate, collisions. One of the key goals of
the \ac{LHC} was discovering the Higgs boson, a goal which was achieved in 2012
\cite{Higgs2012a,Higgs2012b,Higgs2013}. However, the \ac{LHC} research programme
is much more broad than the search and study of the Higgs boson; many aspects of
the \ac{SM} are investigated to find signs of inconsistency between theory and
experiment.

Controlled interactions are created by colliding accelerated beams of protons at
interaction points.  Two beams of protons, travelling in opposite directions
around the \ac{LHC} ring, are accelerated to an energy of 6.5 TeV\footnote{
  Numbers given here correspond to the Run-2 parameters of the machine.
}.
Creating collisions between these two beams at certain interaction points on the
ring results in proton-proton interactions at a centre-of-mass energy of 13 TeV.
Experiments are built around the interaction points to observe the results of
these high-energy interactions.

% How are the protons extracted from hydrogen? Bunches
% Accelerator complex to inject into the LHC
Protons are obtained by ionising hydrogen gas with an electric field. A chain of
many accelerators is then used to take the initial at-rest protons up to an
energy of 450 GeV, when they are injected into the \ac{LHC}. This injector
chain is shown, amongst other \ac{CERN} accelerators, in Figure
\ref{fig:detector-lhc-injectors}. Once in the \ac{LHC}, protons are further
accelerated to the desired beam energy of 6.5 TeV.
% RF Acceleration
This acceleration, both in the injectors and the \ac{LHC} machine itself, is
performed using superconducting radio-frequency cavities; these are
electromagnetic fields that alternate in such a way as to `pull' protons towards
them and then `push' them away.

\begin{figure}[htbp]
  \centering
  \includegraphics[width=.8\textwidth]{\resource{CCC-v2018-print-v2.pdf}}
  \caption{
    Schematic of the \acs{CERN} accelerator complex. The chain of accelerators
    used to inject protons into the \acs{LHC} is
    LINAC2$\to$BOOSTER$\to$PS$\to$SPS \cite{Mobs2018}.
  }
  \label{fig:detector-lhc-injectors}
\end{figure}

% Dipole and quadrupole magnets
Superconducting magnets are used to bend and focus the beam. A total of
1232 dipole magnets, with a field strength of 8.3 T, are used to
bend the path of the beam, as required by the circular design of the collider.
Additionally, 392 quadrupole magnets are placed around the \ac{LHC} to focus the
beam, squeezing the protons together to make the profile of the beam more
compact \cite{LHCVol1}.

% Bunches, bunch trains
Protons are injected into the \ac{LHC} in bunches, with approximately $10^{11}$
protons in a single bunch. Consecutive bunches are injected with a minimum
separation of 25 ns; this is referred to as a bunch train when many bunches are
used at this minimum separation.

% How are the beams made to collide? Bunch crossings
To create collisions between the two proton beams, insertion magnets are used to
cross the paths of the beams \cite{Ostojic2002}. Each colliding pair of bunches,
one from each beam, is called a `bunch crossing'. The magnets can be adjusted in
order to change the crossing angle, affecting the number of proton-proton
collisions induced for each bunch crossing.

% Luminosity
The rate at which collisions occur in the \ac{LHC} is known as instantaneous
luminosity, $\mathcal{L}$, and is given by
\begin{equation*}
  \mathcal{L} = \frac{ N_p^2 n_b f_\text{rev} \gamma }
  { 4 \pi \epsilon_n \beta^*} F,
\end{equation*}
%
where $N_p$ is the number of particles per bunch, $n_b$ is the number of bunches
per beam, $f_\text{rev}$ is the revolution frequency of the beam, $\gamma$ is
the relativistic Lorentz factor, $\epsilon_n$ and $\beta^*$ parametrise the
optics of the beam, and $F$ is a factor describing the crossing angle of the two
beams. The design luminosity for the \ac{LHC} is $\mathcal{L} =
1\times10^{34}~\text{cm}^{-2}\text{s}^{-1}$, and throughout Run 2 the machine
operated between around 0.5 to 2 times this amount \cite{Steerenberg2019}.

% Integrated luminosity
Integrated luminosity, $L$, is used to measure the amount of data in an entire
dataset. This quantity is the integral of instantaneous luminosity over time,
\begin{equation*}
  L = \int dt \mathcal{L}.
\end{equation*}
The expected number of collisions for a particular process is given by the
product of the integrated luminosity with the cross-section of the process,
$\sigma$. Obtaining a large dataset is therefore vital to measure processes with
very low cross sections. The \ac{LHC} produced a dataset of 160 fb$^{-1}$ over
the entirety of Run 2 \cite{Steerenberg2019}.

% Larger schedule, runs and long shutdown
The \ac{LHC} has periods of operations known as runs. Each run consists of
multiple years of data-taking, with some short shutdown periods for maintenance
and minor upgrades. Between each run is a `long shutdown' period, in which more
significant upgrades can take place. Analyses in this thesis use data taken
during Run 2, between 2015 and 2018. Run 3 began in 2022, with an increase in
centre of mass energy to 13.6 TeV, and is currently ongoing at the time of
writing.
 
