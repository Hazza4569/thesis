% ====== Systematic uncertainties section ======

[This will largely be common between the two analyses, could be moved to
analysis methods chapter]

This analysis considers uncertainties from data statistics and from systematics.
Systematic uncertainties come from a variety of sources both theoretical and
experimental.

Experimental systematic uncertainties cover uncertainties in energy scale and
resolution of jets, photons, and electrons; momentum scale and resolution of
muons; scale factors used to reproduce trigger, reconstruction, identification,
and isolation efficiencies from data; suppression of pile-up jets; and flavour
tagging.

Theoretical sources of systematic uncertainty come from the choice of scale,
choice of \ac{PDF} set, modelling, non-closure between \ac{MC} generators,
choice of parton showering and underlying event model, and \ac{EW}-\ac{QCD}
interference. Uncertainty due to scale choice is calculated by varying the
default values of renormalisation and factorisation scales in the nominal QCD
\Zy \ac{MC} sample. Evaluating uncertainty in \ac{PDF} set choice is done using
the eigenvalues of the \ac{PDF} set, for the signal and \ac{QCD} \Zy background.
Uncertainty on modelling of the \ac{QCD} \Zy process comes from choice of
merging (CKKW) and resummation (QSF) scale, these are calculated using the
samples described in Section \ref{sec:methods-samples}. For the \ac{QCD} \Zy
background, the difference between the nominal and alternate samples is taken as
a non-closure uncertainty. For the \ac{EW} signal, parton showering and
underlying event uncertainties are calculated by comparing the default \pythia
hadronisation to an alternate with \herwig. The interference between \ac{EW} and
\ac{QCD} \Zy production is not included in either the signal or background, but
instead taken as an additional uncertainty.

Further systematic uncertainties from limited statistics in \ac{MC} samples and
in data \acp{CR} are also considered.

These detailed systematics are used for both the signal and \ac{QCD} \Zy
background.  A selection of the largest groups of uncertainties are shown in the
\ac{SR} as a function of $m_{jj}$ for both \ac{EW} and \ac{QCD} \Zyjj production
in Figure \ref{fig:vbs-uncertainties-syst-overview}.

\begin{figure}
  \centering
  \includegraphics[width=.49\textwidth]{\resource{systematics_overview_EW.pdf}}
  \hfill
  \includegraphics[width=.49\textwidth]{\resource{systematics_overview_QCD.pdf}}
  \caption{
    Plots of relative variation of yields as a function of dijet mass, $m_{jj}$,
    for \acs{EW} (left) and \acs{QCD} (right) production of \Zyjj. The
    variations shown are the largest groups of systematics in the \acs{SR}.
    %TODO cite conf note
  }
  \label{fig:vbs-uncertainties-syst-overview}
\end{figure}

%TODO somewhere (analysis methods section detailing fake photon ABCD?)
%describe uncertainty estimation for Z+jets
%perhaps add a note here to reference that section 'described in Section X'
Uncertainties on the Z+jets background normalisation are calculated based on the
two-dimensional sideband method, and total 35\%. The \tty and WZ$jj$ backgrounds
are both assigned flat uncertainties, of 15\% and 20\% respectively.
