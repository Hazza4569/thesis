% ====== Particle-flow jet validation section ======

%Rewritten intro
The choice of jet collection for this analysis is non-trivial. Particle-flow
jets have recently become the standard recommendation within \ac{ATLAS} in place
of topo-cluster jets, but rather than applying that recommendation unilaterally,
the specific case for this analysis is considered. For a discussion of jet
collections, see Section \ref{sec:methods-reconstruction-jet}.

The benefits of particle-flow include improved resolution for low-energy jets,
although this only works within the acceptance of the \ac{ID}.  Typical \ac{VBS}
jets are energetic and very forward, perhaps pushing into regions beyond \ac{ID}
acceptance.  Although a great deal of \ac{VBS} jets will still be in range to
benefit from particle-flow, the phase space is very different from one which
would make particle-flow jets an obvious choice. This section presents a
comparison between particle-flow and topo-cluster jets in the analysis phase
space to justify the move to using particle-flow jets.

%Back to it
The procedure for comparing performance between these two jet collections is to
investigate the difference in event yield from applying jet-based selection
criteria with values calculated from either jet collection.  The \Zy selection,
as in Table \ref{tab:anacom-zy-selection}, is applied first then jet-based cuts
are applied and the resulting yields compared.  For simplicity, this was
investigated in the signal \ac{MC} sample for only Z$\to ee$ events.

Looking at the overall yield of events after each cut shows that
the two jet collections give very similar results, always within 1\% of one
another. Table \ref{tbl:vbs-pflow-cutflow} shows these yields.

\begin{table}[tbp]
  \centering
  \caption{
    Yields and efficiencies after each jet cut, compared for both topo-cluster
    and particle-flow jets. Starting from all EW Z($\to ee$)$\gamma jj$
    events that pass the \Zy selection. Efficiencies given are for the
    individual cut, relative to the yield from the previous cut. The difference
    is given as percentage increase from the topo-cluster to the particle-flow
    yields. The third cut is overlap removal between jets and leptons or
    photons.
  }
  \renewcommand\arraystretch{1.3}
  \begin{tabular}{| c || c | c || c | c || c |}
    \hline 
    \multirow{2}{2em}{Cut} & \multicolumn{2}{c||}{Topo-cluster} &
    \multicolumn{2}{c||}{Particle-flow} & \multirow{2}{5em}{Difference}\\
    & \multicolumn{1}{c}{Yield} & Eff. & \multicolumn{1}{c}{Yield} & Eff. & \\
    \hline\hline
    $N_j \geq 2$
    & 51084 & 79.7\% & 51468 & 80.3\% & +0.7\% \\
    $p_T^j > 50~\mathrm{GeV},~|\eta_j| < 4.5$
    & 31362 & 61.4\% & 31604 & 61.4\% & +0.8\% \\
    $\Delta R(l,j), \Delta R(\gamma,j) > 0.4$
    & 31359 & 99.99\% & 31552 & 99.84\% & +0.6\% \\
    $|\Delta\eta_{jj}| > 1.0$
    & 27127 & 86.5\% & 27293 & 86.5\% & +0.6\% \\
    $|m_{jj}| > 150~\mathrm{GeV}$
    & 26752 & 98.6\% & 26885 & 98.5\% & +0.5\% \\
    \hline
  \end{tabular}
  \label{tbl:vbs-pflow-cutflow}
\end{table}

Differences can be further scrutinised by looking at each individual event; most
events should result in the same decision, pass or fail, regardless of the jet
collection chosen. This checks that the similar yields aren't merely a
coincidence, when in fact many events pass only one selection. Figure
\ref{fig:vbs-pflow-pies} shows these per-event differences in decisions.

\begin{figure}[tbh]
  \centering
  \includegraphics[width=\textwidth]{\resource{pflow-pies.pdf}}
  \caption{
    Impact of four of the key jet cuts on the analysis yield when applied
    individually with particle-flow and topo-cluster jets.  Events are divided
    into four categories based on whether or not they pass the particle-flow cut
    and whether or not they pass the topo-cluster cut.  Cuts are applied in the
    same order as presented in Table \ref{tbl:vbs-pflow-cutflow}. Only events
    that pass the previous cut for both particle-flow and topo-cluster jets are
    included in the results, to decorrelate the effects of each individual cut.
    The label `PFlow' is used for particle-flow cuts and `Topo' for topo-cluster
    cuts.
  }
  \label{fig:vbs-pflow-pies}
\end{figure}

For the vast majority of events, the two jet collections make the same
selection, with less than 5\% of events showing differences. These differences
are tested by investigating the distributions in the cut
variables for cases where the two jet collections give a different result.
For example, if an event passes $m_{jj} > 150$ GeV with topo-cluster jets
but not with particle-flow jets then the particle-flow $m_{jj}$ value should be
close to the cut boundary of 150 GeV.

\begin{figure}[tbh]
  \includegraphics[width=.5\textwidth]{\resource{decor_pflow_n_jet_raw.pdf}}
  \includegraphics[width=.5\textwidth]{\resource{decor_pflow_jet_subpt_raw_fix.pdf}}
  \\
  \includegraphics[width=.5\textwidth]{\resource{decor_pflow_m_jj_fix.pdf}}
  \includegraphics[width=.5\textwidth]{\resource{decor_pflow_deta_jj.pdf}}
  \caption{
    Distribution of events passing jet requirements for the particle-flow
    collection but not the topo-cluster collection. In each case the cut is in
    the same distribution as the histogram plotted. Four cuts are shown: $N_j >
    2$ (top left), $p_{T}^{j,2}$ ($p_{T}$ of second most energetic jet) $ > 50$
    GeV (top right), $m_{jj} > 150$ GeV (bottom left), $|\Delta\eta_{jj}| > 1$
    (bottom right).  Only events passing all prior cuts for both particle-flow
    and topo-cluster jets are included.
    The label `PFlow' is used for variables calculated with particle-flow jets
    and `Topo' for topo-cluster jets.
  }
  \label{fig:vbs-pflow-boundary-plots}
\end{figure}

Investigating the case where events are selected using particle-flow jets but
not topo-cluster jets, results for the four key cut variables are shown in
Figure \ref{fig:vbs-pflow-boundary-plots}. The distributions are, largely, as
expected given that values peak on the cut boundary and tail off for more
extreme differences.  However, it is notable that for the dijet variables,
$m_{jj}$ and $|\Delta\eta_{jj}|$, there are some strong outliers.

Events falling very far from the cut boundary for particle-flow jets when the
topo-cluster jet variable fails the cut seem to indicate a significant
disagreement in kinematics between the two jet collections. It is possible that
these outliers happen when the $p_T$ ordering of jets varies between
collections. Dijet variables are calculated using the two highest energy jets,
so a small shift in $p_T$ between the second and third jets could cause dijet
variables to be calculated with a different jet pair and therefore give very
different results.

This hypothesis can be tested by looking at the separation between jets
used in each event for the two collections. The variable used to measure
this is
%
\begin{equation}
  \text{min}\,\Sigma(\Delta R) = \min_{k_i\in\{(1,2),(2,1)\}}\sum_{i = 1}^2
  \Delta R(j^\text{topo-cluster}_i, j^\text{particle-flow}_{k_i}),
  \label{eqn:vbs-pflow-minsumdr}
\end{equation}
\newcommand\mindr{\ensuremath{\text{min}~\Sigma(\Delta R)}\xspace}
%
i.e. the sum of the two $\Delta R$ values between topo-cluster and particle-flow
jets, for whichever pairing of the jets gives the lowest value of the sum.
$\Delta R$ is the sum in quadrature of $\Delta\eta$ and $\Delta\phi$.  Figure
\ref{fig:vbs-pflow-drcut} shows the distribution of this variable for events
passing $m_{jj}$ for only particle-flow jets. There are two clear populations,
separated at \mindr$\sim 0.5$. The lower \mindr population should contain events
where the particle-flow and topo-cluster jets are representing the same physical
objects. Requiring $\text{min}\Sigma(\Delta R) < 0.5$ on top of the existing
selection gives the $m_{jj}$ distribution shown in Figure
\ref{fig:vbs-pflow-drcut}, where now all remaining events are tightly
distributed around the cut boundary.

\begin{figure}[tbh]
  \centering
  \includegraphics[width=\textwidth]{\resource{pflow_drcut.pdf}}
  \caption{
    Distribution of the \mindr variable, defined in Equation
    \ref{eqn:vbs-pflow-minsumdr}, for events passing $m_{jj} > 150$ GeV for
    particle-flow but not topo-cluster jets (left); and the $m_{jj}$
    distribution for these events after requiring $\mindr < 0.5$ (right).
    The label `PFlow' is used for variables calculated with particle-flow jets
    and `Topo' for topo-cluster jets.
  }
  \label{fig:vbs-pflow-drcut}
\end{figure}

This study, although limited in scope, serves to demonstrate that the key jet
variables used in this analysis perform very similarly when calculated with
particle-flow and topo-cluster jets. Any differences seen are sufficiently small
that no meaningful effect on the analysis result is expected.  This is
considered as motivation to use particle-flow jets for this analysis in keeping
with the updated \ac{ATLAS} recommendation.  To really understand if the
improvements in resolution with particle-flow jets are seen in the analysis
phase space, further studies could be conducted on how systematic uncertainties
differ between collections, to determine which would give the most precise
result. This would have been a possible extension given more time.
