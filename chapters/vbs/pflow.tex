% ====== Particle-flow jet validation section ======

A jet, being the name for a hadronic shower observed in the detector rather than
an actual particle itself, can be defined in many different ways. The standard
within \ac{ATLAS} was to use `topo' jets, formed from topological clusters of
cells in the calorimeter \cite{Aad2017b}. More recently a move has been made to
using particle-flow, or `PFlow', jets. PFlow jets improve on the performance of
topo jets by using tracks found in the tracking detector near to deposits in the
calorimeter. This allows PFlow jets to benefit from advantages that the tracker
has over the calorimeter: superior spacial resolution and improved energy
resolution for low momentum particles \cite{Aaboud2017a}.

Whilst PFlow jets have been shown to improve on topo jets in general, they are
not universally better in any scenario, each has its advantages. The most
significant improvements for PFlow jets are in low $p_T$ regimes, and PFlow can
only function in the range of the tracker ($|\eta| < 2.5$). Jets in \ac{VBS}
events would not typically follow these criteria, they are expected to be very
energetic and in the high-$\eta$ forward regions of the detector. This study was
therefore performed to validate that the new PFlow jets were still a good choice
for this analysis in spite of this logical disconnect.

The procedure for comparing performance between these two jet collections is to
investigate the difference in event yield from applying jet-based selection
criteria with values calculated from each jet type. All non-jet selection
criteria discussed in Section \ref{sec:vbs-selection} are applied first, then
jet-based selection criteria are applied and the resulting yields compared. For
simplicity, this was investigated in the signal \ac{MC} sample for Z$\to ee$
(but not Z$\to\mu\mu$) events.

Looking at the overall yield of events after each selection criterion shows that
the two jet collections give very similar results, always equal to within 1\%.
Table \ref{tbl:vbs-pflow-cutflow} shows these yields for the signal \ac{MC}
sample.

\begin{table}
  \centering
  \caption{
    Yields and efficiencies after each jet cut, compared for both topo and PFlow
    jets. Starting from all EW $\mathrm{Z(\to ee)}\gamma\mathrm{jj}$ events that
    pass non-jet selection criteria.
  }
  \renewcommand\arraystretch{1.3}
  \begin{tabular}{| c || c | c || c | c || c |}
    \hline 
    \multirow{2}{2em}{Cut} & \multicolumn{2}{c||}{Topo} &
    \multicolumn{2}{c||}{PFlow} & \multirow{2}{5em}{Difference}\\
    & \multicolumn{1}{c}{Yield} & Eff. & \multicolumn{1}{c}{Yield} & Eff. & \\
    \hline\hline
    $N_j \geq 2$
    & 51084 & 79.7\% & 51468 & 80.3\% & +0.7\% \\
    $p_T^j > 50~\mathrm{GeV},~|\eta_j| < 4.5$
    & 31362 & 61.4\% & 31604 & 61.4\% & +0.8\% \\
    $\Delta R(l,j), \Delta R(\gamma,j) > 0.4$
    & 31359 & 99.99\% & 31552 & 99.84\% & +0.6\% \\
    $|\Delta\eta_{jj}| > 1.0$
    & 27127 & 86.5\% & 27293 & 86.5\% & +0.6\% \\
    $|m_{jj}| > 150~\mathrm{GeV}$
    & 26752 & 98.6\% & 26885 & 98.5\% & +0.5\% \\
    \hline
  \end{tabular}
  \label{tbl:vbs-pflow-cutflow}
\end{table}

Differences can be further scrutinised by looking at each individual event; most
events should result in the same decision, pass or fail, regardless of the jet
collection chosen. This checks that the similar yields aren't merely a
coincidence, when in fact many events pass only one selection. Figure
\ref{fig:vbs-pflow-pies} shows these per-event differences in decisions.

\begin{figure}
  \centering
  \includegraphics[width=\textwidth]{\resource{pflow-pies.pdf}}
  \caption{
    Impact of four of the key jet cuts on the analysis yield when applied in
    both PFlow and topo. Events are divided into four categories based on
    whether or not they pass the PFlow cut and whether or not they pass the topo
    cut. Cuts are applied in the same order as presented in Table
    \ref{tbl:vbs-pflow-cutflow}. Only events that passed the previous cut in
    both PFlow and topo are included in the results, to decorrelate the effects
    of each individual cut.
  }
  \label{fig:vbs-pflow-pies}
\end{figure}

For the vast majority of events, the two jet collections make the same
selection, with less than 5\% of events showing differences. These differences
can be further scrutinised by investigating the distributions in the cut
variables for these cases; if an event passes $m_{jj} > 150$ GeV with topo jets,
but not with PFlow jets, the PFlow $m_{jj}$ value should at least be close to
150 GeV.

\begin{figure}
  \includegraphics[width=.5\textwidth]{\resource{pflow_n_jet_raw.pdf}}
  \includegraphics[width=.5\textwidth]{\resource{pflow_jet_subpt_raw.pdf}}
  \\
  \includegraphics[width=.5\textwidth]{\resource{pflow_m_jj.pdf}}
  \includegraphics[width=.5\textwidth]{\resource{pflow_deta_jj.pdf}}
  \caption{
    Distribution of events passing jet requirements for the PFlow collection
    but not the topo collection. In each case the cut is in the same distribution
    as the histogram being plotted. Four cuts are shown: $N_j > 2$ (top left),
    $p_{T}^{j,2}$ ($p_{T}$ of second most energetic jet) $ > 50$ GeV (top right),
    $m_{jj} > 150$ GeV (bottom left), $|\Delta\eta_{jj}| > 1$ (bottom right).
  }
  \label{fig:vbs-pflow-boundary-plots}
\end{figure}

Investigating the case where events pass selection with PFlow jets but fail with
topo jets, results are shown in Figure \ref{fig:vbs-pflow-boundary-plots}. Here
the four key cut variables are shown. The distributions are, largely, as
expected, with peaks showing most values are close to the cut boundaries. It is
notable, however, that for the two variables using multiple jets, $m_{jj}$ and
$|\Delta\eta_{jj}|$, there are some strong outliers.

On investigation, many of these outliers stem from events where previous
selection criteria were failed with topo jets, and so there were not two jets
available to calculate these variables. In Figure X%TODO
this is corrected so that only events
which pass previous cuts in both variables are considered

%% TODO scrap the plots which don't enforce passing previous cut in both
%% variables? seems more of a journey than a destination and I can still talk
%% about pt-ordering stuff.
%% Need to see if I have equivalent plots with this requirement... think that
%% e.g. /home/harry/Documents/phd/zgam/code/plots/mc16e/EWK_Zyjj_ee/overlap/decor_PFlow_m_jj.pdf
%% is what I need (this is available for all variables) but not sure how
%% /home/harry/Documents/phd/zgam/code/plots/mc16e/EWK_Zyjj_ee/overlap/njtrim_pflow_m_jj.pdf
%% is different. Perhaps 2nd doesn't include pT requirement (other captions
%% claim it does though). 2nd isn't available in all variables so I think I just
%% use the 1st.
