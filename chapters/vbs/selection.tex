% ====== Event selection section ======

% TODO patch up post-AnaCom
Additional requirements are placed on the transverse energies of these objects.
The leading (highest energy) lepton must have $p_T > 30$ GeV, with the second
lepton satisfying $p_T > 20$ GeV. Photons are required to have $p_T > 25$ GeV.
The two highest energy jets must have $p_T > 50$ GeV.

For the dilepton system to be focused on Z$\to ll$ events, a cut of $m_{ll} >
40$ GeV is enforced, removing contributions from low-mass resonances. Events
where the photon arises from final state radiation, emission from one of the
leptons from the Z boson, are undesirable as the photon must exist at the same
time as the Z boson for them to have come from the same vertex. A requirement
on $m_{ll} + m_{ll\gamma} > 182$ GeV will suppress any events with a Z$\to
ll\gamma$ decay producing the photon.

Imposing \ac{VBS}-like kinematics on the jets further reduces the phase space.
The dijet system should have a mass $m_{jj} > 150$ GeV, and a separation between
the two jets of $|\Delta y_{jj}| > 1.0$. A veto is placed on `gap jets', jets
(reconstructed with $p_T > 25$ GeV) found in the rapidity region between the two
VBS tag jets, is applied to exploit the difference between VBS jets and
colour-connected \ac{QCD} jets.  A requirement on the centrality of the
$ll\gamma$ system, relative to the jets, is also used to define the signal
region. The centrality is defined by
%
\begin{equation*}
  \zeta(ll\gamma) = \left|
                      \frac { y_{ll\gamma} - (y_{j_1} + y_{j_2})/2}
                            { y_{j_1} - y_{j_2} }
                    \right|,
\end{equation*}
%
separating the signal region ($\zeta(ll\gamma) < 0.4$) from the control region
for the \ac{QCD} background ($\zeta(ll\gamma) > 0.4$), explained in
Section \ref{sec:vbs-backgrounds}.
