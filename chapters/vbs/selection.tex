% ====== Event selection section ======

Events from data and simulation undergo a selection process to focus on a phase
space that matches the desired final state. Data events are required to pass a
trigger to be recorded, and therefore the same requirement is placed on
simulated events for parity. In this analysis any events passing the single
lepton or dilepton trigger are considered.

For an event to be selected, first the basic objects in the desired final state
need to be present, subject to some reconstruction criteria. There must be at
least one photon passing isolation and identification requirements. There must
be precisely two electrons or muons present, of the same flavour to each other
but opposite charge, with both passing isolation and identification. At least
two jets must have been reconstructed in the event.

Additional requirements are placed on the transverse energies of these objects.
The leading (highest energy) lepton must have $p_T > 30$ GeV, with the second
lepton satisfying $p_T > 20$ GeV. Photons are required to have $p_T > 25$ GeV.
The two highest energy jets must have $p_T > 50$ GeV.

For the dilepton system to be focused on Z$\to ll$ events, a cut of $m_{ll} >
40$ GeV is enforced, removing contributions from low-mass resonances. Events
where the photon arises from final state radiation, emission from one of the
leptons from the Z boson, are undesirable as the photon must exist at the same
time as the Z boson for them to have come from the same vertex. A requirement
on $m_{ll} + m_{ll\gamma} > 182$ GeV will suppress any events with a Z$\to
ll\gamma$ decay producing the photon.

Imposing \ac{VBS}-like kinematics on the jets further reduces the phase space.
The dijet system should have a mass $m_{jj} > 150$ GeV, and a separation between
the two jets of $|\Delta y_{jj}| > 1.0$. A veto is placed on `gap jets', jets
(reconstructed with $p_T > 25$ GeV) found in the rapidity region between the two
VBS tag jets, is applied to exploit the difference between VBS jets and
colour-connected \ac{QCD} jets.  A requirement on the centrality of the
$ll\gamma$ system, relative to the jets, is also used to define the signal
region. The centrality is defined by

\begin{equation*}
  \zeta(ll\gamma) = \left|
                      \frac { y_{ll\gamma} - (y_{j_1} + y_{j_2})/2}
                            { y_{j_1} - y_{j_2} }
                    \right|,
\end{equation*}

separating the signal region ($\zeta(ll\gamma) < 0.4$) from the control region
for the \ac{QCD} background ($\zeta(ll\gamma) > 0.4$), explained in
Section X. %TODO
