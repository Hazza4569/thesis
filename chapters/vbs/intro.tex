% ==== Introduction Section ====

% TODO add discussion of previous and related measurements

Vector-boson scattering (\ac{VBS}) provides a unique experimental signature, producing
decay channels with excellent potential to probe rare \ac{SM} processes; the
high selection efficiency achievable by exploiting kinematics of the \ac{VBS}
tag jets allows measurements to be conducted at lower cross sections than would
otherwise be accessible in the current dataset.

VBS \Zy serves as a production mechanism for the \Zyjj final state,
%The \Zyjj final state can be interpreted as \ac{VBS} \Zy production,
with the Z boson and photon resulting from a direct multiboson interaction
and the jets created as a feature of the \ac{VBS} production.
This provides a robust framework for studying these rare \ac{SM} interactions
introduced in Chapter \ref{sec:theory}.
%
Feynman diagrams for \ac{VBS} \Zy production are represented in Figure
\ref{fig:vbs-vbsfeynman}, showing that \ac{QGC} or \ac{TGC} vertices are the
only \ac{SM} contributions at tree-level.

\begin{figure}
  \centering
  \includegraphics[width=\textwidth]{\resource{zy-vbs-feyn.pdf}}
  \caption{
    Feynman diagram for a \Zy vector-boson scattering event (left). The
    black circle contains the multiboson interaction, which for a tree-level
    \acs{SM} interaction will be one of the two shown (right).
  }
  \label{fig:vbs-vbsfeynman}
\end{figure}

% VBS kinematics
In the archetypal \ac{VBS} event, a quark from each of the two colliding protons
radiates a boson. The two bosons interact to produce the \ac{EW} component of
the final state and the quarks, deflected from their original trajectories after
boson emission, appear as jets in the detector. Since the initial quarks are
usually very energetic, the angle through which they are deflected in the
interaction tends to be small. The final-state jets, known as tag jets, would
therefore be in the very forward regions of the detector, at opposite ends to
one another, and also still carrying large amounts of energy. These kinematics
are typically characterised by a large invariant mass of the dijet system
($m_{jj}$) and a large difference between the rapidities of the jets ($|\Delta
y_{jj}|$).

% Other EWK production
\ac{VBS} \Zy production is one component of the more general \ac{EW} production
of \Zyjj. The \ac{VBS} production modes are not gauge-invariantly separable from
others, so a direct measurement of \ac{VBS} \Zy is not strictly possible.
Instead, \ac{EW} \Zyjj production is measured with a selection designed to
enhance the \ac{VBS} component, matching the kinematics of the jets with the
expected \ac{VBS} signature. Figure \ref{fig:vbs-nonvbsfeynman} gives Feynman
diagrams for some non-\ac{VBS} production modes that contribute to the \ac{EW}
production mechanism.

\begin{figure}
  \centering
  \includegraphics[width=\textwidth]{\resource{ewk-zyjj.pdf}}
  \caption{
    Example Feynman diagrams for non-\acs{VBS} \acs{EW} production of \Zyjj.
    In these instances one or none of the two final-state bosons are produced
    through multiboson interactions.
  }
  \label{fig:vbs-nonvbsfeynman}
\end{figure}

% Introduce QCD as main background
To measure this \ac{EW} \Zyjj production, background processes with the same
final state must be understood. The dominant background for this analysis comes
from \ac{QCD} \Zyjj production. Figure \ref{fig:vbs-qcdfeynman} gives example
Feynman diagrams for this \ac{QCD} production, which differs from the \ac{EW}
mode as the strong force either provides the interaction between the two quarks
or otherwise generates the final-state jets, resulting in colour-connected jets.
%
Additional interactions between the colour-connected jets are
very probable and will affect the observed jet kinematics, allowing these events
to be distinguished from \ac{VBS} events.

\begin{figure}
  \centering
  \includegraphics[width=\textwidth]{\resource{qcd-zyjj.pdf}}
  \caption{
    Example Feynman diagrams for \acs{QCD} production of \Zyjj.
  }
  \label{fig:vbs-qcdfeynman}
\end{figure}

This analysis is the first iteration of a \ac{VBS} \Zy analysis using the full
Run 2 dataset \cite{VBSZy-CONF}. It builds on a measurement made with a 36
fb$^{-1}$ partial Run-2 dataset \cite{ATLASvbszy2020}, and the work has been
continued beyond what is presented here in Reference \cite{VBSZy-PAPER}.
Complementary measurements include one of the same process by the \ac{CMS}
experiment \cite{CMSvbszy2021}, and an \ac{ATLAS} measurement of \ac{VBS} \Zy
with the Z boson decaying to two neutrinos \cite{ATLASvbsznunuy2023}. This is
part of a programme of \ac{ATLAS} \ac{VBS} measurements \cite{%
  ATLASvbsssww2023,%
  ATLASvbszz2023,%
  ATLASvbswz2019,%
  ATLASvbsvv2019%
},
all contributing to the understanding of multiboson interactions.
% 

%overall strategy of measurement:
The goal of this analysis is to measure the fiducial cross section of \ac{EW}
\Zyjj production in a region sensitive to \ac{VBS} \Zy production.
% > Instead of the below, reference the above lit review TODO
%Ideally, if enough precision is obtained, this will constitute the first
%observation of this process by the \ac{ATLAS} experiment.
%
The measurement relies on a cut-based selection, exploiting the \ac{VBS} event
kinematics. Signal and background processes are estimated, through a combination
of \ac{MC} simulation and data-driven estimates, and used to make a template fit
to the dijet mass distribution. This chapter presents the analysis as a whole,
with additional focus given to sections on jet collection investigations
(Section \ref{sec:vbs-pflow}),
controlling jet flavour systematics (Section \ref{sec:vbs-fgluon}), and pruning
  of systematic uncertainties (Section \ref{sec:vbs-prune}).
