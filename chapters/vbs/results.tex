% ====== Results section ======

Fitted $m_{jj}$ distributions are shown in Figure
\ref{fig:vbs-result-postfit-mjj}.
The measured signal strength in the fit is
%
\begin{equation*}
  \begin{split}
  \mu_\text{EW} &= 0.95 ^{+0.14} _{-0.13}
  \\            &= 0.95 \pm 0.08 \, (\text{stat.}) \pm 0.11 \, (\text{syst.}).
  \end{split}
\end{equation*}
%
This corresponds to an observed(expected) significance of 10(11) standard
deviations, and is the first observation of this process by the \ac{ATLAS}
collaboration. As the measurement is consistent with $\mu_\text{EW} = 1$, the
rate seen in data is consistent with the \ac{SM} expectation.

The fiducial cross-section of the \ac{EW} production of \Zyjj in this
\ac{VBS}-like phase space is therefore measured as
\begin{equation*}
  \sigma_\text{EW} = 4.49 \pm 0.40 \, (\text{stat.}) \pm 0.42 \, (\text{syst.})
  ~\text{fb}.
\end{equation*}

\begin{figure}
  \centering
  \includegraphics[width=\textwidth]{\resource{postfit_mjj_SR}}
  \\[-3em]
  \includegraphics[width=\textwidth]{\resource{postfit_mjj_CR}}
  \caption{
    Post-fit distributions of dijet mass, $m_{jj}$, in the \ac{SR} (top) and
    \ac{CR} (bottom). The uncertainty band is the combination of all
    uncertainties, taken from the fit. Overflow events are included in the last
    bin. \cite{VBSZy-CONF}
  }
  \label{fig:vbs-result-postfit-mjj}
\end{figure}

%TODO:
% - systematic ranking
% - separate systematic error by experiment and theory
