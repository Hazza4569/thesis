% ====== Results section ======

\begin{figure}
  \centering
  \includegraphics[width=\textwidth]{\resource{postfit_mjj_SR}}
  \\[-3em]
  \includegraphics[width=\textwidth]{\resource{postfit_mjj_CR}}
  \caption{
    Post-fit distributions of dijet mass, $m_{jj}$, in the \ac{SR} (top) and
    \ac{CR} (bottom). The uncertainty band is the combination of all
    uncertainties, taken from the fit. Overflow events are included in the last
    bin. \cite{VBSZy-CONF}
  }
  \label{fig:vbs-result-postfit-mjj}
\end{figure}

The fit gives a measured signal strength of
%
\begin{equation*}
  \begin{split}
  \mu_\text{EW} &= 0.95 ^{+0.14} _{-0.13}
  \\            &= 0.95 \pm 0.08 \, (\text{stat.}) \pm 0.11 \, (\text{syst.}).
  \end{split}
\end{equation*}
%
This corresponds to an observed(expected) significance of 10(11) standard
deviations, and is the first observation of this process by the \ac{ATLAS}
collaboration. As the measurement is consistent with $\mu_\text{EW} = 1$, the
rate seen in data is consistent with the \ac{SM} expectation.

% Improvements over previous iteration
This result is a marked improvement over the previous iteration of the analysis,
which measured a significance of 4.1 standard deviations \cite{ATLASvbszy2020}.
Nearly four times the amount of data was available for this analysis, which
allows a reduced statistical uncertainty, but the improvement persists beyond
that still, lowering the overall systematic uncertainty from $_{-17}^{+19}\%$ to
$\pm9\%$. This reduced uncertainty comes from several improvements across the
analysis: increased data statistics in control regions and updated background
estimation procedure to reduce uncertainty on the Z+jets background, larger
\ac{MC} simulation samples to reduce the previously significant \ac{MC}
statistical uncertainty, and greatly reduced systematic uncertainties on the
jets thanks to improvements in the analysis methods such as those presented in
Section \ref{sec:vbs-fgluon}.

The total yields from data and signal and background estimates in both the
\ac{SR} and \ac{CR} are given in Table \ref{tab:vbs-results-yields}.
Post-fit $m_{jj}$ distributions in these regions are shown in Figure
\ref{fig:vbs-result-postfit-mjj}.

The fiducial cross section of the \ac{EW} production of \Zyjj in this
\ac{VBS}-like phase space is measured from the fit as
\begin{equation*}
  \sigma_\text{EW} = 4.49 \pm 0.40 \, (\text{stat.}) \pm 0.42 \, (\text{syst.})
  ~\text{fb}.
\end{equation*}

\begin{table}[!b]
  \centering
  \caption{
    Yield estimates and associated post-fit uncertainties for each of the processes
    contributing to the signal region and control region, compared to data. The
    total estimate and its uncertainty is also given.
  }
  \label{tab:vbs-results-yields}
  \begin{tabular}{p{2.5cm}cc}
    \midrule\midrule
    \multirow{2}{*}{Process} & \multicolumn{2}{c}{Yield} \\\cmidrule{2-3}
                             & \ac{SR} & \ac{CR} \\\midrule
    \ac{EW} \Zyjj & $300 \pm 36$  & $55 \pm 7$ \\
    \QCDZy        & $987 \pm 55$  & $1352 \pm 60$ \\
    \tty          & $72 \pm 11$   & $59 \pm 9$ \\
    \WZjj         & $17 \pm 3$    & $14 \pm 3$ \\
    Z+jets        & $85 \pm 30$   & $143 \pm 43$ \\\midrule
    Total         & $1461 \pm 38$ & $1624 \pm 40$ \\\midrule
    Data          & 1461          & 1624 \\
    \midrule\midrule
  \end{tabular}
\end{table}

%TODO:
% - systematic ranking
% - separate systematic error by experiment and theory
