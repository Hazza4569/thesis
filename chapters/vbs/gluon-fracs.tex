% ====== Jet flavour uncertainties section ======
\newcommand\fgluon{$f_g$\xspace}

The uncertainties associated with the flavour composition and response of the
jets in this analysis make a significant contribution to the result. This
section presents measures taken to reduce these uncertainties and in turn
improve the precision of the final measurement.

These jet flavour uncertainties, as discussed in Section
\ref{sec:methods-systematics-jet-flavour}, can be reduced by specifying the
expected fraction of jets initiated by (light) quarks and gluons in the analysis phase
space. This is parameterised by the gluon fraction, given in Equation
\ref{eqn:methods-systematics-fgluon}.

A given jet in a \ac{MC} event is determined to be quark- or gluon-initiated
from the truth record, using the \verb|PartonTruthLabelID| variable.  Measuring
both \fgluon and its associated uncertainty as a function of jet pseudorapidity
and transverse momentum, for events passing the analysis selection, provides the
information needed to reduce the jet flavour uncertainty.

This calculation is performed on events in the \ac{QCD} \Zy \ac{MC} sample, in
the inclusive analysis region (defined by Table \ref{tab:vbs-selection}) and its
subregions, the \ac{SR} and the \ac{QCD} \ac{CR}. This study is not necessary
for other samples as the uncertainties have a lesser impact on the final
measurement.

Uncertainty on \fgluon arises from three sources: a modelling uncertainty
calculated by comparing the \fgluon values obtained from two independent \ac{MC}
generators, statistical uncertainty resulting from the size of the \ac{MC}
sample used, and an additional uncertainty to cover any variations in the value
of \fgluon between regions. The third uncertainty component is necessary due to
technical limitations, which allowed only one \fgluon value to be provided for
samples used to calculate yields in both the \ac{SR} and the \ac{CR}.

Calculation of \fgluon is performed for the \ac{QCD} \Zy sample only, as this is
where the jet flavour uncertainty is largest.

Figure \ref{fig:vbs-fgluon-2dhists} shows the gluon fractions measured in the
nominal and alternate \ac{MC} samples for \ac{QCD} \Zy, as well as calculations
of each of the uncertainty components.  The statistical uncertainty was found to
be an order of magnitude smaller than the other components and so is neglected.
The uncertainty to cover differences between regions is calculated by finding
the largest difference, in each bin, between \fgluon in the inclusive region
(\ac{SR}+\ac{CR}) and either of the two sub-regions. The overall uncertainty
used is then the per-bin quadrature sum of the generator differences and this
inter-region difference.

\begin{figure}[p]
  \centering
  % ROW 1
  \begin{subfigure}{.48\textwidth}
    \includegraphics[width=\textwidth]{\resource{fgluon_secondaryGF.pdf}}
    \caption{}
  \end{subfigure}
  \hfill
  \begin{subfigure}{.48\textwidth}
    \includegraphics[width=\textwidth]{\resource{fgluon_primaryGF.pdf}}
    \caption{}
  \end{subfigure}
  \\[1em]%
  % ROW 2
  \begin{subfigure}{.48\textwidth}
    \includegraphics[width=\textwidth]{\resource{fgluon_GFdiff.pdf}}
    \caption{}
  \end{subfigure}
  \hfill
  \begin{subfigure}{.48\textwidth}
    \includegraphics[width=\textwidth]{\resource{fgluon_primaryGFstat.pdf}}
    \caption{}
  \end{subfigure}
  \\[1em]%
  % ROW 3
  \begin{subfigure}{.48\textwidth}
    \includegraphics[width=\textwidth]{\resource{fgluon_SRdiff.pdf}}
    \caption{}
  \end{subfigure}
  \hfill
  \begin{subfigure}{.48\textwidth}
    \includegraphics[width=\textwidth]{\resource{fgluon_CR1diff.pdf}}
    \caption{}
  \end{subfigure}
  \caption{
    Gluon fractions and uncertainties as a function of jet pseudorapidity and
    transverse momentum. Plotted are gluon fraction in the nominal (a) and
    alternate (b) \acs{QCD} \Zy sample \acs{MC} samples, the difference between gluon
    fractions in these two samples (c), the statistical uncertainty on gluon
    fractions in the alternate sample (d) (this was the larger of the
    statistical uncertainties), and the difference between gluon fraction in the
    \acs{SR}+\acs{CR} region and in the \acs{SR} (e) and \acs{CR} (f) regions.
  }
  \label{fig:vbs-fgluon-2dhists}
\end{figure}

%TODO something on the impact of this? Not sure I have the data to hand
