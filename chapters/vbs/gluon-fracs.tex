% ====== Jet flavour uncertainties section ======
\newcommand\fgluon{$f^\text{gluon}$}

A significant source of systematic uncertainty in this analysis arises from
uncertainty in the flavour of jets, in particular whether they are initiated by
quarks or gluons. In order to minimise the impact of this uncertainty, the
fraction of jets initiated by gluons should be measured.

The gluon fraction is given by
%
\begin{equation*}
  f^\text{gluon} = \frac{ N_j^\text{gluon} }
                        { N_j^\text{gluon} + N_j^\text{udsc}},
\end{equation*}
%
where $N_j^\mathrm{gluon}$ and $N_j^\mathrm{udsc}$ are the number of
jets initiated by gluons and by up/down/strange/charm quarks respectively.
Measuring both \fgluon and its associated uncertainty as a function of jet
pseudorapidity and transverse momentum provides the information needed to reduce
the jet flavour uncertainty.

Uncertainty on \fgluon is from three sources: a modelling uncertainty calculated
by finding the difference in \fgluon between two independent \ac{MC} generators,
statistical uncertainty from the size of the \ac{MC} sample used, and an extra
uncertainty to cover any differences in the value of \fgluon between regions.
The third uncertainty component is necessary as, for technical reasons, only one
\fgluon value could be provided for samples used to calculate yields in both
the \ac{SR} and the \ac{CR}.

Calculation of \fgluon is performed for the \ac{QCD} \Zy sample only, as this is
where the jet flavour uncertainty is largest.

Figure \ref{fig:vbs-fgluon-2dhists} shows the gluon fractions measured in the
nominal and alternate \ac{MC} samples for \ac{QCD} \Zy, as well as calculations
of each of the uncertainty components.  The statistical uncertainty was found to
be an order of magnitude smaller than the other components and so is neglected.
The uncertainty to cover differences between regions is calculated by finding
the largest difference, in each bin, between \fgluon in the inclusive region
(\ac{SR}+\ac{CR}) and either of the two sub-regions. The overall uncertainty
used is then the per-bin quadrature sum of the generator differences and this
inter-region difference.

\begin{figure}
  \centering
  % ROW 1
  \begin{subfigure}{.48\textwidth}
    \includegraphics[width=\textwidth]{\resource{fgluon_secondaryGF.pdf}}
    \caption{}
  \end{subfigure}
  \hfill
  \begin{subfigure}{.48\textwidth}
    \includegraphics[width=\textwidth]{\resource{fgluon_primaryGF.pdf}}
    \caption{}
  \end{subfigure}
  \\[1em]%
  % ROW 2
  \begin{subfigure}{.48\textwidth}
    \includegraphics[width=\textwidth]{\resource{fgluon_GFdiff.pdf}}
    \caption{}
  \end{subfigure}
  \hfill
  \begin{subfigure}{.48\textwidth}
    \includegraphics[width=\textwidth]{\resource{fgluon_primaryGFstat.pdf}}
    \caption{}
  \end{subfigure}
  \\[1em]%
  % ROW 3
  \begin{subfigure}{.48\textwidth}
    \includegraphics[width=\textwidth]{\resource{fgluon_SRdiff.pdf}}
    \caption{}
  \end{subfigure}
  \hfill
  \begin{subfigure}{.48\textwidth}
    \includegraphics[width=\textwidth]{\resource{fgluon_CR1diff.pdf}}
    \caption{}
  \end{subfigure}
  \caption{
    Gluon fractions and uncertainties as a function of jet pseudorapidity and
    transverse momentum. Plotted are gluon fraction in the nominal (a) and
    alternate (b) \acs{QCD} \Zy sample \acs{MC} samples, the difference between gluon
    fractions in these two samples (c), the statistical uncertainty on gluon
    fractions in the alternate sample (d) (this was the larger of the
    statistical uncertainties), and the difference between gluon fraction in the
    \acs{SR}+\acs{CR} region and in the \acs{SR} (e) and \acs{CR} (f) regions.
  }
  \label{fig:vbs-fgluon-2dhists}
\end{figure}
