% ====== Pruning systematic uncertainties section ======

To limit the number of nuisance parameters included in the fit, a system is
developed to rank the impact of different systematics on both the shape and
normalisation of the $m_{jj}$ distribution. Only systematic uncertainties deemed
to be significant are fully accounted for in the fit, and those with less impact
are pruned.

The first test for a systematic uncertainty is how uniform its effect is across
the dijet mass spectrum -- this will indicate whether it will impact the shape
of the $m_{jj}$ distribution. If a systematic is determined to have a
significant impact on shape, by criteria discussed below, then it is included in
the fit with one nuisance parameter for each bin in $m_{jj}$, allowing it to
modify the shape in the fitting process.

Any uncertainty not found to impact the shape should be assessed for how
significant an impact it has on the overall normalisation of events. Systematic
uncertainties with a large enough effect on the event yield will contribute one
nuisance parameter to the fit, and have the ability to scale the overall
normalisation. Any uncertainties with a smaller effect will be pruned, i.e. all
pruned systematics will be added in quadrature as a single extra nuisance
parameter to scale the overall normalisation in the fit.

%% Calculating stat uncertainties
\subsection{Calculating statistical uncertainties}

In order to determine whether the effect of any systematic uncertainty, on shape
or overall yield, is significant, the statistical uncertainty on the value of
the systematic uncertainty must be ascertained. This statistical uncertainty
arises from the finite size of \ac{MC} samples used to evaluate systematic
uncertainties.

The value of a systematic uncertainty on an event yield is given by
%
\begin{equation}
  \sigma_\text{Norm} = \frac{ N_\text{varied} - N_\text{nominal} } {
  N_\text{nominal} },
  \label{eqn:vbs-prune-sigmanorm}
\end{equation}
%
where $N_\text{nominal}$ is the number of events in acceptance for a nominal
\ac{MC} sample and $N_\text{varied}$ is the number of events after the
systematic variation has been applied. Each of $N_\text{varied}$ and
$N_\text{nominal}$ has a statistical uncertainty. However, due to the fact that
these variables are measuring the same set of events under different conditions,
the two yields are highly correlated. The correlation is not known a priori, and
so the uncertainty on $\sigma_\text{Norm}$ cannot be calculated through error
propagation.

The bootstrap method\cite{Efron1979,Efron1987} is instead used in order to
determine statistical uncertainties while preserving correlations. This method
relies on resampling the event set to create replica sets of events of the same
size, with some events duplicated and some omitted. Calculating
$\sigma_\text{Norm}$ in each replica set gives a distribution of results for
which the standard deviation represents the statistical uncertainty on
$\sigma_\text{Norm}$.


%% Shape significance
\subsection{Determining shape impact}

For a systematic uncertainty that has no impact on $m_{jj}$ shape, it would be
expected that the resulting variation would be uniform across the $m_{jj}$
distribution. This is tested by calculating the fractional difference in yield,
$\sigma_\text{Norm}$, and its associated statistical uncertainty in bins of
$m_{jj}$. A chi-squared test from fitting a zeroth order polynomial to these
values provides a test statistic which should be distributed as $\chi^2(3)$ (4
bins minus 1 parameter for 3 degrees of freedom) under the null hypothesis of no
shape impact. A significant shape uncertainty is therefore anything that
deviates from this null hypothesis above a certain threshold.  An example
$m_{jj}$ distribution and fit is shown in Figure \ref{fig:vbs-prune-mjjfit}.

\begin{figure}
  \centering
  \includegraphics[width=.95\textwidth]{\resource{shape_uncert-fit_example.pdf}}
  \caption{
    Binned $m_{jj}$ distribution of measured systematic uncertainty,
    $\sigma_\text{Norm}$, for one systematic variation as labelled on the plot.
    The dashed line shows the zeroth-order polynomial fit attempted, with the
    indicated $\chi^2$ value demonstrating this is clearly a poor assumption and
    this systematic does contribute an uncertainty on the $m_{jj}$ shape.
  }
  \label{fig:vbs-prune-mjjfit}
\end{figure}

The threshold chosen is a $p$-value of 0.05, i.e. chi-squared values
sufficiently high that there is at most a 5\% chance that the deviation arose
from statistical fluctuation. Uncertainties with a $p$-value below this
threshold have the full shape treatment in the fit, with per-bin nuisance
parameters. Figure \ref{fig:vbs-prune-shapeuncerts} shows the results of the
chi-squared test for the largest experimental systematics in the \ac{SR}. 

\begin{figure}
  \centering
  \includegraphics[width=.85\textwidth]{\resource{shape_uncert-ewk-SR.pdf}}
  \\[1em]
  \includegraphics[width=.85\textwidth]{\resource{shape_uncert-qcd-SR.pdf}}
  \caption{
    $\chi^2$ values, representing the impact each systematic has on shape,
    for all experimental systematic variations in the \acs{EW} (top) and
    \acs{QCD} (bottom) samples in the \acs{SR}.
    Only sources with $\chi^2$ above $7$ for either the up or down
    variation are shown. The top axis gives the probabilities of uncertainties
    arising fluctuations under the null hypothesis.
    The largest uncertainties extend beyond the range of the $x$-axis.
  }
  \label{fig:vbs-prune-shapeuncerts}
\end{figure}


\subsection{Determining overall yield impact}

Any systematic uncertainty determined to not impact the shape of the $m_{jj}$
distribution can of course still affect the overall yield of events, and
therefore require sufficient treatment in the fit. This could be done by
assigning a single nuisance parameter to every remaining uncertainty, as all
will have an effect on some scale. To reduce the number of nuisance parameters
required however, the less significant uncertainties are pruned.

In this case significance is simply determined by whether or not a systematic
uncertainty is consistent with zero. Taking the value of the systematic, per
Equation \ref{eqn:vbs-prune-sigmanorm}, and its statistical uncertainty or
standard deviation, if the value is within one standard deviation of zero it is
considered consistent with being zero.

All systematic uncertainties not passing the shape significance test but not
consistent with zero have a dedicated nuisance parameter in the fit. All
remaining uncertainties are pruned.
The relative change in yield from each of the largest experimental systematic
uncertainties is shown in Figure \ref{fig:vbs-prune-scaleuncerts}.

\begin{figure}
  \centering
  \includegraphics[width=.85\textwidth]{\resource{scale_uncert-ewk-SR.pdf}}
  \\[1em]
  \includegraphics[width=.85\textwidth]{\resource{scale_uncert-qcd-SR.pdf}}
  \caption{
    Relative change in yield, $\sigma_\text{Norm}$, for all experimental
    systematic uncertainties in the \acs{SR} for the \acs{EW} (top) and
    \acs{QCD} (bottom) samples. Any with a value below $5\times10^{-4}$ are
    omitted. The black bars represent the statistical uncertainty on the value.
  }
  \label{fig:vbs-prune-scaleuncerts}
\end{figure}
