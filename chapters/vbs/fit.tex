% ====== Template Fit Section ======

The signal strength, $\mu_\text{EW}$, is used to parameterise the fiducial
cross section for the signal process, $\sigma^\text{EW}$, where
%
\begin{equation}
  \mu_\text{EW} = \sigma^\text{EW}_\text{meas} / \sigma^\text{EW}_\text{SM},
  \label{eqn:vbs-fit-mu}
\end{equation}
%
i.e. the ratio of the measured cross section to the \ac{SM} expectation. This
signal strength is extracted from the data with a maximum likelihood fit,
performed on $m_{jj}$ distributions in the \ac{SR} and \ac{CR} simultaneously.
\ac{MC} distributions for backgrounds and signal are used as templates, with
normalisations for the signal and \ac{QCD} \Zy background allowed to float in
the fit.

Electron and muon channels are treated together, using the sum of events from
both as input to the fit. A binned likelihood is built using
the $m_{jj}$ distribution in both the \ac{SR} and \ac{CR}, as described in
Section \ref{sec:methods-stats-llh}, with systematic uncertainties included as
nuisance parameters.
The effect of each uncertainty on the normalisation and shape of the
$m_{jj}$ distribution is considered individually and a pruning system, described
in Section \ref{sec:vbs-prune}, is used to reduce the number of nuisance
parameters needed.

The fit extracts the value of the signal strength, $\mu_\text{EW}$.
From this a significance of the measurement under the background-only hypothesis
is calculated, as described in \ref{sec:methods-stats-llhr}.
If the significance is greater five standard deviations, \ac{EW} \Zy
production is considered to be observed. The value of the signal strength can
also be used to give a measurement of the fiducial cross section of the process
given the of the \ac{SM} expectation of this cross section: calculated from the
nominal \ac{MC} as
%
\begin{equation*}
  \sigma^\text{EW}_\text{SM} = 4.73
  \pm 0.01 \, (\text{stat.})
  \pm 0.15 \, (\acs{PDF})
  ^{+0.23} _{-0.22} \, (\text{scale})
  ~\text{fb}.
\end{equation*}

%TODO define fiducial selection
