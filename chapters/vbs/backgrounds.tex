% ====== Background estimation section ======
% https://atlas.web.cern.ch/Atlas/GROUPS/PHYSICS/CONFNOTES/ATLAS-CONF-2021-038/

The dominant background for this analysis, \QCDZy production, suffers from
known mismodelling for high dijet masses, which is the
region of interest in this analysis. Rather than using the \ac{MC} template to
directly estimate this background, the normalisation is corrected by
comparing with data in a \ac{CR} enriched in this background.  The centrality
variable is used to separate the \ac{SR} ($\zeta(ll\gamma) < 0.4$) from this
\ac{QCD} \ac{CR} ($\zeta(ll\gamma) > 0.4$). The \ac{CR} is rich in the \ac{QCD}
background and has a very small fraction of signal events, as the \ac{EW}
production mechanism peaks at low values of centrality. Figure
\ref{fig:vbs-bkg-centrality} shows the centrality distribution for signal and
background events.

\begin{figure}[tb]
  \centering
  \includegraphics[width=\textwidth]{\resource{centrality_dist.pdf}}
  \caption{
    Centrality distribution for data, signal, and background estimates pre-fit
    (before any data corrections to \QCDZy). The dashed line marks
    the separation between the \acs{SR} and \acs{QCD} \acs{CR}. The uncertainty
    band is the combination of uncertainties from background estimation,
    \acs{MC} statistics, and experimental systematics. Overflow events are
    included in the last bin. \cite{VBSZy-CONF}
  }
  \label{fig:vbs-bkg-centrality}
\end{figure}

This normalisation correction is calculated by fitting a normalisation factor for
the \ac{QCD} background in the \ac{SR} and \ac{CR} simultaneously, allowing the
overall normalisation to be adjusted according to data. The shape of the
background is taken from \ac{MC}, but data in the \ac{CR} is used to validate
this shape and constrain correlated uncertainties.

Estimation of the remaining backgrounds, Z+jets, \tty, and \WZjj, is detailed
in Section \ref{sec:methods-backgrounds}.

% Z+jets
%The second-largest background is from Z+jets events, and enters the final state
%when a jet is misidentified as a photon. This background includes Z+jets
%produced through both \ac{EW} and \ac{QCD} induced processes.  The rate of
%particles being misidentified is not well modelled in \ac{MC}, and so the shape
%and normalisation of this background must be corrected with a data-driven
%method.
%
%The normalisation for the Z+jets background is calculated with the
%two-dimensional sideband, or ABCD, method.
%If the region in which the background is being estimated is region A, three more
%regions are defined by inverting isolation and identification criteria on the
%reconstructed photon. Inverting the photon isolation gives region B, inverting
%identification gives region C, and inverting both criteria gives region D. These
%three control regions are used to infer the amount of Z+jets background in the
%region of interest with the relationship
%%
%\begin{equation*}
%  \newcommand\Zj{\text{Z+jets}}
%  N_A^\Zj = R\frac{N_B^\Zj \times N_C^\Zj}{N_D^\Zj},
%\end{equation*}
%%
%where $N_X^\text{Z+jets}$ is the number of Z+jets events in the given region
%calculated by subtracting background and signal leakage from the data events
%i.e.
%%
%\begin{equation*}
%  \newcommand\Zj{\text{Z+jets}}
%  N_X^\Zj = N_X^\text{data} - N_X^\text{bg} - c_X N_A^\text{sig,data},
%  ~~\text{for}~X=B,C,D.
%\end{equation*}
%%
%The correlation factor, $R$, is given by
%%
%\begin{equation*}
%  \newcommand\Zj{\text{Z+jets}}
%  R = \frac{ N_A^\Zj \times N_D^\Zj }{ N_B^\Zj \times N_C^\Zj },
%\end{equation*}
%%
%where in this case each $N_X^\text{Z+jets}$ is the event yield observed in
%Z+jets \ac{MC} in this region. Also defined are signal leakage parameters,
%$c_X$, as
%\begin{equation*}
%  \newcommand\tsig{\text{sig}}
%  c_X = \frac{N_X^\tsig}{N_A^\tsig},
%  ~~\text{for}~X=B,C,D,
%\end{equation*}
%calculated from \ac{QCD} and \ac{EW} \Zy \ac{MC}.
%
%The shape of the Z+jets background is taken directly from a data control region.
%The control region should be very pure in Z+jets events, but also sufficiently
%high statistics. The chosen region is the anti-tight region, with no requirement
%on track or calorimeter isolation. This is equivalent to regions C and D
%combined but without the track isolation requirement.
%
%% tty and WZ
%The background from \tty events is estimated from \ac{MC} and cross-checked in
%an $e\mu\gamma$ \ac{CR}, which validates the use of a $k$-factor of 1.44 to
%scale the \ac{MC} normalisation. WZ$jj$ events make a minor contribution to the
%background, this is estimated solely from \ac{MC}.
