% ====== Samples section ======

[All or part of this section will end up in an `Analysis Methods' chapter with
things common to both analyses. Here for now to complete the story.]

The two analyses discussed in this thesis share a common final state, signal
production mechanism, and backgrounds. The simulated event samples used are
therefore the same for both analyses, with each focusing on a different phase
space within these events.

\ac{EW} \Zyjj production, the signal sample, is generated with
\verb|MadGraph5_aMC@NLO| 2.6.5. %TODO < format and reference
This sample is at \ac{LO} accuracy (order $\alpha_\text{EW}^4$) with the
\verb|NNPDF3.1| \ac{LO} \ac{PDF} set. % TODO < format and reference
Parton showering, hadronisation, and underlying event activity are added through
\verb|Pythia| 8.240 %TODO < format and reference
(with ``dipoleRecoil'' enabled).

The dominant background process is \ac{QCD} production of \Zyjj. The nominal
sample used for this process is produced with
\verb|MadGraph5_aMC@NLO| 2.3.3 %TODO cite
using the
\verb|NNPDF3.0| \ac{NLO} \ac{PDF} set, %TODO cite
this includes all diagrams at order $\alpha_s^2\alpha_\text{EW}^2$ with up to
two additional partons in the final state, where one parton may be at \ac{NLO}.

Additional samples for \ac{QCD} \Zyjj are generated to evaluate uncertainties.
A sample made with
\verb|Sherpa| 2.2.4 %TODO cite
at \ac{LO} accuracy, with up to three additional parton emissions, is used to
measure generator differences.
The \verb|NNPDF3.0| \ac{NNLO} \ac{PDF} set %TODO cite
is used for this sample, in
conjunction with a dedicated parton shower tuning developed by the \verb|Sherpa|
authors.
%
Five more samples are used for evaluation of theoretical uncertainty. These are
generated at particle level using
\verb|Sherpa| 2.2.10 %TODO cite
with the \verb|NNPDF3.0| \ac{NNLO} \ac{PDF} set. %TODO cite? 
One of the five samples has a nominal value for the merging and resummation
scales and the other four have an up or down variation for either.

%TODO interference sample?

The Z+jets background, in which a jet is misidentified as a photon, is estimated
with a data-driven method. A \ac{MC} sample for this process is necessary to
evaluate correlation between the regions used in this method, as discussed in
Section X. %TODO
\verb|PowhegBox v1| %TODO cite
is used to generate this sample at \ac{NLO} accuracy with the CT10 \ac{NLO}
\ac{PDF} set. %TODO cite
\verb|Pythia| 8.210 %TODO cite
is used for parton showering in this sample, with the AZNLO %TODO cite
set of tuned parameters.

A \tty sample is generated at \ac{LO} accuracy with \verb|MadGraph5_aMC@NLO| and
the \verb|NNPDF| 2.3 \ac{LO} \ac{PDF} set. %TODO cite
The WZ background has both a \ac{QCD} and \ac{EW} component; with samples from
\verb|Sherpa| 2.2.2 %TODO cite
at \ac{NLO}, with \verb|NNPDF3.0| \ac{NNLO} \ac{PDF} set, and
\verb|MadGraph5_aMC@NLO| 2.6.2 %TODO cite
at \ac{LO}, with \verb|NNPDF3.0| \ac{LO} \ac{PDF} set, respectively.

Pileup (additional proton-proton interactions) is overlaid on simulated samples,
generated with \verb|Pythia| 8.186 using the A3 tune %TODO cite
and the \verb|NNPDF2.3| \ac{LO} \ac{PDF} set. %TODO cite
Data is used to reweight these \ac{MC} events to respect the mean number of
interactions per bunch crossing from the corresponding data-taking period.

Once physics events and pileup are combined, samples are passed through a
simulation of the \ac{ATLAS} detector with GEANT4 %TODO 2 citations
and then processed with offline reconstruction in the same manner as data
events. Additional scale factors and smearing are applied to more closely match
data events.
