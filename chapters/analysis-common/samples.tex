% ====== Samples section ======

Samples created from \ac{MC} simulations are used in the analyses to represent
the \ac{SM} prediction for the rate of a particular process (see Section
\ref{sec:theory-mc}). Beyond providing the \ac{SM} estimate to which data is
compared in the chosen sensitive phase space, the signal region, these
simulations are also used to design the analysis. This includes optimising
the selection cuts which define sensitive regions, training machine learning
discriminants, and estimating the expected sensitivity of the analysis before
unblinding.

The two analyses presented here have the same underlying signal process and the
same set of backgrounds, so the \ac{MC} samples used are common for both
analyses. Table \label{tab:methods-data-triggers} summarises how these samples
were produced, including the physics process which is simulated; the \ac{MC}
generator used for the hard scatter; the generator used to add parton showering,
hadronisation, and underlying event; the order to which the cross section is
calculated for the hard scatter; and the \ac{PDF} set used by the hard-scatter
generator. The remainder of this section gives more precise details for each of
these samples.

\begin{table}[tb]
  \centering
  \renewcommand\arraystretch{1.2}
  \caption{
    Samples from \ac{MC} simulation used in estimating signal and background
    processes. For each sample the generator used for the hard scatter process is listed
    as well as the generator used to add parton showering, hadronisation, and
    the underlying event (marked PS\&UE). The order to which the cross section
    is calculated and the \ac{PDF} set used are also given. Numbers on the right
    are used to label the samples in the text.
    Information on the listed \ac{PDF} sets can be found in References
    \cite{NNPDF3dot1,NNPDF3dot0,ct10,NNPDF2dot3}
  }
  \begin{tabular}{cccccc}
    \cmidrule{1-5}\morecmidrules\cmidrule{1-5}
    \bf Process   & \bf Hard scatter & \bf PS\&UE & \bf Order & \bf \ac{PDF} set & \\ \cmidrule{1-5}
    \ac{EW} \Zyjj & \madgraph & \pythia & \acs{LO} & NNPDF3.1 \acs{LO} & (1) \\ \cmidrule{1-5}
    \multirow{2}{*}{\ac{QCD} \Zy} & \madgraph & \pythia & \acs{NLO} & NNPDF3.0 \acs{NLO} & (2) \\
                                  & \sherpa & \sherpa & \acs{LO} & NNPDF3.0 \acs{NNLO} & (3) \\ \cmidrule{1-5}
    Z+jets    & \powhegbox  & \pythia & \acs{NLO} & CT10 \acs{NLO} & (4) \\ \cmidrule{1-5}
    \tty      & \madgraph  & \pythia & \acs{LO} & NNPDF2.3 \acs{LO} & (5) \\ \cmidrule{1-5}
    \ac{QCD} WZ & \sherpa  & \sherpa & \acs{NLO} & NNPDF3.0 \acs{NNLO} & (6) \\
    \ac{EW} \WZjj & \madgraph  & \pythia & \acs{LO} & NNPDF3.0 \acs{LO} & (7) \\
    \cmidrule{1-5}\morecmidrules\cmidrule{1-5}
  \end{tabular}
  \label{tab:methods-data-triggers}
\end{table}

%Signal: A14 tune
%QCD MG: pythia 8.212 A14 tune
%Z+jets: ERRATUM pythia v8.186
%tty: pythia 8.212 A14 tune
%WZ: pythia 8.235 A14 tune

The signal sample (Sample 1) uses \madgraphfiveamc 2.6.5 \cite{madgraph5amc} as
well as
\pythia 8.240 \cite{pythia8dot2}. The \acp{DSID}\footnote{
  This is an internal \acs{ATLAS} identifier for the sample, included for
  completeness.
}
for this sample are 363267-363268.
%The \ac{LO} accuracy includes diagrams of order $\alpha_\text{EW}^4$.
An alternate version of this sample is produced, with {\herwig}++ 2.7.1
\cite{herwigpp,herwigpp2dot7} in place of \pythia, to evaluate uncertainties due
to the choice of parton showering and underlying event model.

For the \ac{QCD} \Zy samples, Sample 2 (\acp{DSID} 345775-345782) is the nominal
sample and uses \madgraphfiveamc 2.3.3 and \pythia 8.212 and Sample 3
(\acp{DSID} 366140-366149) gives an alternate estimate using \sherpa 2.2.4
\cite{sherpa2dot2}. Both of these samples include additional parton emission
beyond the order at which the cross-section is calculated \cite{VBSZy-CONF}. An
additional five samples are generated at particle level for this process, using
\sherpa 2.2.10. These are used for evaluating theoretical uncertainty and have
varied values for merging and resummation scales (see Section
\ref{sec:theory-mc}).

Sample 4 (\acp{DSID} 361106-361107) uses \powhegbox v1
\cite{Nason2004,powheg,powhegbox} and \pythia 8.186 \cite{pythia8dot1}.
This sample is not used directly for a background estimate, but as part of the
data-driven estimate discussed in Section
\ref{sec:methods-backgrounds-fakephoton}.

Sample 5 (\ac{DSID} 410389) uses \madgraphfiveamc 2.3.3 and \pythia 8.212.
Sample 6 (\ac{DSID} 364253) uses \sherpa 2.2.2, and Sample 7 (\acp{DSID}
364739-364742) is produced with \madgraphfiveamc 2.6.2 and \pythia 8.235.

An additional truth-level sample is used to calculate interference between
\ac{EW} and \ac{QCD} \Zyjj production. This is estimated at \ac{LO} with
\madgraphfiveamc 2.3.3 with the NNPDF3.0 \ac{LO} \ac{PDF} set.

All samples interfaced with \pythia use a specific set of parameters derived
from data, a tune (as introduced in Section \ref{sec:theory-mc}). For samples
generated with \madgraph and \pythia, the A14 tune \cite{A14tune} is used. The
remaining \pythia sample, Sample 4, uses
the AZNLO tune \cite{AZNLOtune}.
% Also A3 tune used to add pileup (is this just UE?)

% I didn't include the below in the updated version of this chapter. Is it
% needed? I don't think so, this content is covered elsewhere (without
% specififcs e.g. A3 tune). Would need to rewrite if I do include.
%Pileup (additional proton-proton interactions) is overlaid on simulated samples,
%generated with \pythia 8.186 using the A3 tune \cite{A3tune}
%and the NNPDF2.3 \ac{LO} \ac{PDF} set.
%Data is used to reweight these \ac{MC} events to respect the mean number of
%interactions per bunch crossing from the corresponding data-taking period.
%
%Once physics events and pileup are combined, samples are passed through a
%simulation of the \ac{ATLAS} detector with GEANT4 \cite{ATLASsim1,geant4}
%and then processed with offline reconstruction in the same manner as data
%events. Additional scale factors and smearing are applied to more closely match
%data events.
