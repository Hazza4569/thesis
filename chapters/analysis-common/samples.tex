% ====== Samples section ======

% TODO check this covers everything from VZy
% TODO make changes from Dave's feedback
% TODO add a table to summarise samples?

% TODO explain methodology of using simulations to represent theory,
% as well as using blinding to avoid data bias?
% Where does this fit? Theory, or earlier in this chapter?
The signal and background processes for both analyses are the same, the same
simulated event samples can therefore be used for each.

\ac{EW} \Zyjj production, the signal sample, is generated with \madgraphfiveamc
2.6.5 \cite{madgraph5amc}.  This sample is at \ac{LO} accuracy (order
$\alpha_\text{EW}^4$) with the NNPDF3.1 \ac{LO} \ac{PDF} set \cite{NNPDF3dot1}.
Parton showering, hadronisation, and underlying event activity are added through
\pythia 8.240 \cite{pythia8dot2} (with ``dipoleRecoil'' enabled).

The dominant background process is \ac{QCD} production of \Zyjj. The nominal
sample used for this process is produced with \madgraphfiveamc 2.3.3
\cite{madgraph5amc} using the NNPDF3.0 \ac{NLO} \ac{PDF} set \cite{NNPDF3dot0},
this includes all diagrams at order $\alpha_s^2\alpha_\text{EW}^2$ with up to
two additional partons in the final state, where one parton may be at \ac{NLO}.

Additional samples for \ac{QCD} \Zyjj are generated to evaluate uncertainties.
A sample made with \sherpa 2.2.4 \cite{sherpa2dot2} at \ac{LO} accuracy, with up
to three additional parton emissions, is used to measure generator differences.
The NNPDF3.0 \ac{NNLO} \ac{PDF} set is used for this sample, in conjunction with
a dedicated parton shower tuning developed by the \sherpa authors.
%
Five more samples are used for evaluation of theoretical uncertainty. These are
generated at particle level using \sherpa 2.2.10 \cite{sherpa2dot2} with the
NNPDF3.0 \ac{NNLO} \ac{PDF} set.  One of the five samples has a nominal value
for the merging and resummation scales and the other four have an up or down
variation for either.

%TODO interference sample?

The Z+jets background, in which a jet is misidentified as a photon, is estimated
with a data-driven method. A \ac{MC} sample for this process is necessary to
evaluate correlation between the regions used in this method, as discussed in
Section \ref{sec:vbs-backgrounds}.
\powhegbox v1 \cite{Nason2004,powheg,powhegbox}
is used to generate this sample at \ac{NLO} accuracy with the CT10 \ac{NLO}
\ac{PDF} set \cite{ct10}.
\pythia 8.210 \cite{pythia8dot2}
is used for parton showering in this sample, with the AZNLO \cite{AZNLOtune}
set of tuned parameters.

A \tty sample is generated at \ac{LO} accuracy with \madgraphfiveamc and
the NNPDF2.3 \ac{LO} \ac{PDF} set \cite{NNPDF2dot3}.
The WZ background has both a \ac{QCD} and \ac{EW} component; with samples from
\sherpa 2.2.2 \cite{sherpa2dot2}
at \ac{NLO}, with NNPDF3.0 \ac{NNLO} \ac{PDF} set, and
\madgraphfiveamc 2.6.2 \cite{madgraph5amc}
at \ac{LO}, with NNPDF3.0 \ac{LO} \ac{PDF} set, respectively.

Pileup (additional proton-proton interactions) is overlaid on simulated samples,
generated with \pythia 8.186 using the A3 tune \cite{A3tune}
and the NNPDF2.3 \ac{LO} \ac{PDF} set.
Data is used to reweight these \ac{MC} events to respect the mean number of
interactions per bunch crossing from the corresponding data-taking period.

Once physics events and pileup are combined, samples are passed through a
simulation of the \ac{ATLAS} detector with GEANT4 \cite{ATLASsim1,geant4}
and then processed with offline reconstruction in the same manner as data
events. Additional scale factors and smearing are applied to more closely match
data events.
