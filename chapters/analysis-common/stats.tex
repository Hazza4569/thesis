% ========== STATISTICAL INFERENCE SECTION ==========

% Introduction: fitting used in both analyses for parameter estimation of signal
% strength

% Simple binned likelihood calculation

% FROM VZY BDT SECTION
\newcommand\nobsi{\ensuremath{n_\text{obs}^i}\xspace}
As the $m_{jj}$ distribution is not used for selection, it is used here to
calculate significance with a binned likelihood method. Consider $n_b^i$ as the
number of background events in bin $i$, and $n_s^i$ the number of signal events,
from the relevant \ac{MC} samples after selection.  The likelihood of observing
\nobsi events in bin $i$ is given by a Poisson distribution with a mean of
$n_b^i + \mu\cdot n_s^i$, where $\mu$ is a signal strength parameter with
$\mu=0$ for the background-only hypothesis or $\mu=1$ for alternate hypothesis
with signal included. The likelihood, $\mathcal{L}$, of observing the set of
$\{\nobsi\}$ in each bin is then the product of each of the per-bin likelihoods.

Constructing the likelihood ratio
\begin{equation*}
  \mathscr{\lambda} = \frac{ \mathcal{L}( \{\nobsi\} ; \mu=0 ) }
                           { \mathcal{L}( \{\nobsi\} ; \mu=1 ) },
\end{equation*}
enables a likelihood-ratio test, with the test statistic $-2\ln\lambda$
distributed as $\chi^2(1)$ \cite{Wilks1938}, to calculate the significance of
observing $\{\nobsi\}$.

To obtain integer values for \nobsi from the \ac{MC} prediction, as required by
the use of the Poisson distribution, random toy experiments are run. For each
experiment, \nobsi is picked at random from a Poisson distribution with mean
$n_b^i + n_s^i$. Running 1000 of these experiments, calculating the significance
for each, and taking the mean of the results gives an estimate for the
significance.

These significances are calculated for each selection tested, given as a number
of standard deviations.

% FROM VZY FIT SECTION
%Likelihood model
In order to find which values for \muEW and the nuisance parameters give the
best agreement with the data, a likelihood model is built. The likelihood is the
product of several terms, primarily: a Poisson term for each bin in the fit,
describing the probability of obtaining the observed data yield given the
estimates from the templates; and a constraint function for each nuisance
parameter. Given that each nuisance parameter has only an up variation, down
variation, and nominal value, the constraint functions must be interpolated.
This is done using a linear interpolation for shape uncertainties and an
exponential interpolation for normalisation uncertainties; these interpolations
are discussed in Reference \cite{Cranmer2012}.

%Minimisation
Once constructed, the maximum value for the likelihood must be found. This is
achieved by minimising the negative logarithm of the likelihood through the
Davidon-Fletcher-Powell approach \cite{Davidon1959,Fletcher1970,Powell1983}
implemented in Minuit's MIGRAD algorithm \cite{Minuit2}.
The values of parameters that minimise the negative log likelihood are taken as
the fitted values for the \ac{PoI} and nuisance parameters. Uncertainties for
these parameters are given by the covariance matrix calculated during
minimisation. The MINOS technique \cite{Minuit2} is used to obtain a more
accurate estimate of the uncertainties on \muEW.
