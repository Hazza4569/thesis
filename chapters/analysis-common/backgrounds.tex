% ========== Background estimation section ==========
\newcommand\nxzj[2]{\ensuremath{N_{#1,\text{#2}}^\text{Z+jets}}}

The two presented analyses share a common set of background processes. Due to
the differing phase space, estimation of the \ac{QCD} \Zy background is
different for each analysis. The remaining backgrounds however follow the same
estimation procedure for both analyses. This section discusses the procedure for
the common backgrounds: Z+jets, \tty, and \WZjj.

\subsection{Monte Carlo backgrounds}

The background from \tty events is estimated from \ac{MC} with a \ac{NLO}
$k$-factor of 1.44 applied, calculated in Reference \cite{ATLAStty2019}.
A conservative normalisation uncertainty of 15\% is applied to this background
estimate.

Events from \WZjj make a minor contribution to the background, this is
estimated solely from \ac{MC}. Again a simple normalisation uncertainty is
applied, here a value of 20\% is chosen.

\subsection{Fake photon estimation}
\label{sec:methods-backgrounds-fakephoton}

Background from Z+jets events mimics the analysis final state when a jet is
misidentified as a photon. Fake photons such as these are not well modelled in
\ac{MC}, and so the shape and normalisation of this background is
calculated with a data-driven method.

\subsubsection{Normalisation}

The ABCD method is used to estimate the normalisation for this process. This is
done by establishing three orthogonal control regions adjacent to the region of
interest (e.g. the \ac{SR}). Cuts in two different variables, here the photon
identification and isolation, are used to define these regions, as demonstrated
by Figure \ref{fig:anacom-background-crs}.  The region of interest is labelled
as region A, inverting the photon calorimeter isolation selection gives region
B, inverting the identification criteria gives region C, and inverting both
gives region D.  Track isolation is still required for the photon in all
regions.

\begin{figure}[h]
  \centering
  \includegraphics[width=.7\linewidth]{\resource{Zjets-CRs.pdf}}
  \caption{
    Schematic of the four regions used for fake photon background estimation.
    Region A represents the signal region; B, C, and D represent control
    regions obtained by relaxing isolation and/or identification requirements.
  }
  \label{fig:anacom-background-crs}
\end{figure}

These three control regions are used to infer the
amount of Z+jets background in the region of interest with the relationship
%
\begin{equation*}
  \newcommand\Zj{\text{Z+jets}}
  N_A^\Zj = R\frac{
    \nxzj{B}{data} \times \nxzj{C}{data}
  }{
    \nxzj{D}{data}
  }
\end{equation*}
%
where \nxzj{X}{data} is the number of Z+jets events in the given region
calculated by subtracting background and signal leakage from the data yield
i.e.
%
\begin{equation*}
  \newcommand\Zj{\text{Z+jets}}
  \nxzj{X}{data} = N_X^\text{data} - N_X^\text{bg} -
    c_X N_{A,\text{data}}^\text{sig},
  ~~\text{for}~X=B,C,D.
\end{equation*}
%
Background subtraction is performed for any background without a prompt Z boson
and photon, in this case \tty and \WZjj.
The correlation factor, $R$, is given by
%
\begin{equation*}
  \newcommand\Zj{\text{Z+jets}}
  R = \frac{ 
    \nxzj{A}{MC} \times \nxzj{D}{MC}
  }{
    \nxzj{B}{MC} \times \nxzj{C}{MC}
  }
\end{equation*}
%
where in this case each \nxzj{X}{MC} is the event yield observed in
Z+jets \ac{MC} in this region. Uncorrelated identification and isolation
requirements gives $R=1$, so the calculated value should be close to this.
Also defined are signal leakage parameters,
$c_X$, as
\begin{equation*}
  \newcommand\tsig{\text{sig}}
  c_X = \frac{N_{X,\text{MC}}^\tsig}{N_{A,\text{MC}}^\tsig},
  ~~\text{for}~X=B,C,D,
\end{equation*}
calculated from \ac{QCD} and \ac{EW} \Zy \ac{MC}. Signal leakage represents
prompt photon events that enter the \acp{CR}, hence both \ac{EW} and \ac{QCD}
\Zy production are considered as `signal' in this instance.


\subsubsection{Shape}
The shape of the Z+jets background is taken directly from a data control region.
The control region should be very pure in Z+jets events, but also sufficiently
high statistics. The chosen region is the anti-tight region, with no requirement
on track or calorimeter isolation. This is equivalent to regions C and D
combined but without the track isolation requirement.


\subsubsection{Uncertainties}
\label{sec:methods-backgrounds-zjets-uncertainty}

Several components of the normalisation of this background estimate have
associated uncertainties. These are propagated to the final normalisation and
included as a systematic uncertainty on the results.

% N_X
The \ac{MC} background subtraction is subject to any uncertainty on the
subtracted backgrounds. As this is predominantly from \tty, the 15\% \tty
uncertainty is used on the total subtracted background.

% c_X
The signal leakage fractions, $c_X$, are split into two components, \ac{EW} and
\ac{QCD}, representing the leakage from each source of prompt photons.  To find
the
uncertainty on the \ac{QCD} leakage fraction it is first calculated with both
the nominal and alternate sample. The difference between the calculated leakage
fractions is combined with the \ac{MC} statistical uncertainty on the nominal
sample to calculate the total \ac{QCD} uncertainty. The \ac{EW} leakage fraction is a
minor contribution to the total leakage fraction, and so the uncertainty is
taken as 50\%, combined with the \ac{MC} statistical uncertainty.

% R
The correlation factor, R, has an uncertainty calculated from data-\ac{MC}
comparisons in complementary regions where the photon fails track isolation
requirements.
The correlation factor is re-calculated for both data and \ac{MC}
with the track isolation requirement inverted. The difference between these two
$R$ values is combined with the \ac{MC} statistical uncertainty on the nominal
$R$ value to give its uncertainty.
This assumes the data-\ac{MC} agreement is consistent between these
complementary regions and the primary ABCD regions. Inverting the track
isolation selection ensures a fake-rich data sample which should be comparable
to the Z+jets \ac{MC} sample.

