% ========== DATA AND BLINDING STRATEGY SECTION ==========

% Data sample
The presented analyses use data collected by the \ac{ATLAS} experiment during
Run 2 of the \ac{LHC}, between 2015 and 2018. Including all data recorded in
stable beam conditions with all relevant systems operational, the integrated
luminosity for this sample is 139 fb$^{-1}$ \cite{ATLASdq2020}.

% Triggers
The unprescaled single lepton and dilepton triggers
\cite{ATLAStrigegam2020,ATLAStrigmuon2020}
were used to select data events, due to the requirement of a leptonically
decaying $Z$ boson in events. Table \ref{tab:methods-data-triggers} gives the
$p_T$ thresholds required by these triggers for isolated leptons, depending on
the lepton flavour and run period. Additional single lepton triggers with higher
$p_T$ thresholds and looser isolation requirements are also included to improve
efficiency.

This set of triggers was found to accept 99\% of events which would pass the
\acs{VBS} \Zy selection described in Section \ref{sec:vbs-selection}. This
efficiency is expected to be comparable for the semileptonic \VZy selection.

% Blinding strategy
Analyses are performed `blind', meaning that data yields in certain regions are
not looked at until the analysis strategy is decided. This is done to avoid data
bias, i.e. adapting the analysis procedure based on effects in the data (which
could be statistical fluctuations).

In each analysis certain control regions are used, both for estimating
backgrounds and validating data-\ac{MC} agreement. These regions were `unblinded'
first in order to validate the methods for which they are used. The signal
regions in the two analyses remained blinded until the fits had been finalised,
at which point unblinding and running the fit represents the final measurement
being taken.

\begin{table}[tb]
  \centering
  \renewcommand\arraystretch{1.2}
  \caption{
    Transverse momentum thresholds for triggers used data in presented analyses.
    Where two numbers are given, for the dilepton triggers, the first gives the
    threshold for the leading lepton and the second for the sub-leading.
  }
  \begin{tabular}{lcc}
    \hline\hline
    \multicolumn{1}{c}{\multirow{2}{*}{Signature}} & \multicolumn{2}{c}{Threshold $p_T$ [GeV]} \\
    \cline{2-3}
              & 2015    &   2016-18 \\
    \hline
    Single electron  &  24     &  26     \\   
    Single muon      &  20     &  26     \\   
    Dielectron       &  12, 12  &  24, 24  \\
    Dimuon           &  18, 8   &  22, 8   \\
    \hline\hline
  \end{tabular}
  \label{tab:methods-data-triggers}
\end{table}
