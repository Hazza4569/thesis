% ========== Object reconstruction section ==========
%TODO label WPs used for e/y/mu ID/isolation (loose,medium,tight) and cite.

A reconstruction procedure is applied in order to match signals in the detector
to the corresponding physics object that created them, and to calculate the
kinematic properties of the incident particles.
The procedure used is different for any class of physics object. This section
discusses the details of the reconstruction for each of the objects used in the
two presented analyses: photons, electrons, muons, and jets.

%Explain hits&tracks and clusters&topoclusters

\subsection{Photons}
\label{sec:methods-reconsruction-photon}

% Signature
Photon reconstruction covers two scenarios: unconverted photons where the photon
passes through the tracker and deposits its energy in the calorimeter, or
converted photons where the photon converts into an $e^+e^-$ pair inside the
tracker.
The signature for an unconverted photon is an \ac{EM} cluster with $E_T > 1.5$
GeV and no associated
track (due to the photon being electrically neutral). The signature for a
converted photon is two opposite-sign electron candidates with tracks from the
same vertex, within the tracking system, consistent with a massless particle.
These signatures are considered as photon candidates.

% Energy calculation and calibration
Transverse energies are calculated by combining candidate photon \ac{EM}
clusters with any further clusters within a $0.075\times0.125$
($\eta\times\phi$) area centred on the candidate cluster. Energy measurements
are corrected for scale and resolution effects due to variation in detector
response across $\eta$-$\phi$ and data-\ac{MC} differences
\cite{EGamCalibration2019}. Systematic uncertainties are included in the results
to account for uncertainties from these corrections.

% Identification
Jets can often produce similar signatures to photons, and so additional
requirements are placed on the calorimeter shower shape to discriminate against
these `fake photons'. The desired prompt photons typically result in more
collimated clusters contained within the \ac{EM} calorimeter, whereas fake
photons produce broader showers and leakage into the hadronic calorimeter.
An identification selection, consisting of a set of cuts on shower-shape
variables, is derived to minimise photon fakes.
% TODO cite a source for photon ID?

% Isolation
\newcommand\ptcone{$p_T^{\text{cone},20}$\xspace}
\newcommand\etcone{$E_T^{\text{cone},20}$\xspace}
Additionally, photons must be considered isolated. Non-prompt photons will
typically appear nearby other activity in the detector. Prompt photons can be
selected by ensuring that photon candidates are isolated in a region with little
activity around them. Two variables are used to define the isolation:
\ptcone and \etcone.
\ptcone is the sum of transverse momenta of all $p_T > 1$ GeV tracks originating from the
\ac{PV}, within a cone of $\Delta R < 0.2$ around the direction of the photon.
\etcone is the sum of \ac{EM} cluster transverse energies within the same cone,
minus the energy of the photon.
Cuts are defined on these variables as a function of the photon $p_T$:
\ptcone $<0.05\cdot p_T^\gamma$ (track isolation) and \etcone $<0.065\cdot
p_T^\gamma$ (calorimeter isolation).

Systematics uncertainties are included in the measurements to account of the efficiencies of photon
identification and isolation \cite{PhotonIDIsoEff2016}.


\subsection{Electrons}

% Signature and energy
The basic signature to seed an electron is an \ac{EM} cluster with $E_T>1$ GeV
with an associated track that has hits in at least four silicon layers. As with
photons, the energy for an electron candidate is calculated by summing the
energy of the seed cluster with any additional clusters in a $0.075\times0.125$
($\eta\times\phi$) area, as well as any clusters matched to the same track as
the seed. Energy scale and resolution effects are accounted for in the same
manner as for photons, and included in systematic uncertainties.

%Identification and Isolation
Electron candidates are also subject to identification and isolation
requirements, to minimise the impact of fakes. Identification is based on both
\ac{EM} shower shape in the calorimeter and transition radiation in the
\ac{TRT}. Similarly to the photon, isolation is determined in both the tracker
and calorimeter by requiring the energy or momenta within a cone around the seed
is below a threshold. The details of both identification and isolation of
electrons are discussed in Reference \cite{ElectronIDIso2019}.


\subsection{Muon}

% Basic signature
Muon candidates are seeded from tracks in either the \ac{MS} or \ac{ID}.  A seed
track in the \ac{MS} must be matched to a track in the \ac{ID}, and a seed track
in the \ac{ID} must be matched to at least three hits in the \ac{MS}.
% More complex requirements
Muon candidates are only considered here within the acceptance of the \ac{ID}
($|\eta| < 2.5$). Candidates must produce three hits in at least two \ac{MS}
stations, or in only one station for muons with $|\eta|<0.1$.
%TODO explain what a MS station is in detector chapter
% Energy
The transverse momentum of the muon is calculated from a combined track fit of the
tracks/hits in the \ac{ID} and \ac{MS} and the corresponding energy loss in the
calorimeters.

% Isolation
Muons must also be isolated to select prompt muons from, e.g., boson decays
rather than those from hadronic sources. Muon isolation is given by the total
$p_T$ in a cone around the muon divided by the muon $p_T$. As with electrons and
photons, this is calculated in both the \ac{ID} and the calorimeter.

Data-MC comparisons are used to measure the efficiency and
resolution of muon reconstruction, accounted for in systematic uncertainties
\cite{MuonReco2021}.


\subsection{Jets}
\label{sec:methods-reconstruction-jet}

% What is a jet? How does it represent a quark/gluon?
A jet is a physics object representing a localised grouping of hadrons, rather
than a single particle. These hadronic showers can be initiated by quarks or
gluons radiated from the hard scatter in the collision, since these free partons
will hadronise before interacting with any elements of the detector. Jets are
thus used as a reconstruction-level analogue of a quark or gluon produced in a
physics process.

% Jet algorithms
Reconstructing a jet requires use of a jet clustering algorithm in order to
combine clusters and tracks in the detector to collect all the particles likely
to have been produced by the incident hadron. Jet reconstruction in \ac{ATLAS}
uses the anti-$k_t$ jet clustering algorithm \cite{antikt}, with a distance
parameter of $R=0.4$.

% Jet collections
A jet can be reproduced differently depending on the information given as input
to the clustering algorithm. A `jet collection' is the name used for jets
produced from a certain set of inputs. The baseline jet collection used in the
two presented analyses is `particle-flow' jets, in the \ac{VBS} analysis
`topo-cluster' jets are also considered.

% Topo jets
Topo-cluster jets are formed using only calorimeter information, passing topo
clusters as input to the clustering algorithm. Topo clusters, or topological
clusters, are a pileup-resistant formulation of a calorimeter cluster: with
cells added to a cluster based on whether the measured energy exceeds a
threshold determined by the expected noise in that cell. Since only calorimeter
information is used to create the jets, topo-cluster jets rely heavily on the
resolution of the calorimeter \cite{Aad2017b}.

%PFlow jets
Particle-flow jets are an alternative jet collection created by using
`particle-flow objects' as input to the clustering algorithm. A particle-flow
object is a combination of calorimeter topo clusters and \ac{ID} tracks, with
calorimeter deposits produced by charged particles removed to avoid
energy/momentum double-counting, designed to represent a single particle.
Combining calorimeter and tracker information allows for improved resolution at
lower energies compared to topo-cluster jets \cite{Aaboud2017a}.

% Uncertainties
Systematic uncertainties are included to account for effects on the energy and
resolution of jets from detector calibration, properties of the jet such as
quark/gluon flavour composition, and data-\ac{MC} differences
\cite{JESUncerts2017}.
