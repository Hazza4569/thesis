% ========== Systematics section ==========

Many sources of uncertainty are considered for processes estimated in the
presented analyses. The subsections below cover uncertainties from theoretical
and experimental sources which are considered for \ac{EW} and \ac{QCD} \Zyjj
production processes.

Smaller sources of background, Z+jets, \tty, and \WZjj, are each assigned a
single normalisation uncertainty, as detailed in Section
\ref{sec:methods-backgrounds}.
Limited statistics in \ac{MC} samples also contributes uncertainties to all
processes.

All systematic uncertainties are included as nuisance parameters in the fit used
for each analysis, following the procedure given in Section
\ref{sec:methods-stats-llh-nps}.

% Theoretical uncertainties
\subsection{Theoretical uncertainties}
\label{sec:methods-systematics-theory}

Theoretical uncertainties are calculated for \ac{EW} and \ac{QCD} \Zy production
mechanisms. These come from a variety of sources: choice of \ac{PDF} set,
renormalisation and factorisation scales, \ac{QCD} modelling, choice of parton
showering and underlying event model, and \ac{EW}-\ac{QCD} interference.

% > PDF
Evaluating the uncertainty in \ac{PDF} set choice is done by reweighting events
using a number of replica \ac{PDF} sets, chosen in agreement with the PDF4LHC
recommendations \cite{Butterworth2016}. Taking the standard deviation of yields
under each of these weights gives the \ac{PDF} uncertainty on the event yield. 

% > Scale
The uncertainty due to scale choice is calculated by varying the default values of
renormalisation and factorisation scales in the nominal QCD \Zy \ac{MC} sample.
Each scale value is independently varied up and down by a factor 2.
The per-bin envelope of all deviations from combinations of these variations
is taken as the scale uncertainty.

% > QCD modelling
\ac{QCD} modelling uncertainty can be calculated conservatively by comparing
predictions from different generators or, alternatively, by evaluating the
effect of merging and resummation scales. Generator differences are calculated
by taking the difference in event yields predicted by the nominal and alternate
\ac{QCD} \Zy samples. This difference is considered as the \ac{QCD} modelling
uncertainty on the nominal yield.
Alternately, uncertainty from choice of merging (CKKW) and
resummation (QSF) scale is calculated using the dedicated samples described in
Section \ref{sec:methods-samples}. The latter method is used for the \ac{VBS}
analysis whilst the former is used for the triboson analysis, due to
availability of samples.

% > EW PS&UE
For the \ac{EW} signal, parton showering and underlying event uncertainties are
calculated by comparing the default \pythia samples to alternatives with \herwig
or with eigenvariations of the \pythia tune \cite{A14tune}.  The difference in
predicted yields between the default and \herwig samples is taken as the
uncertainty on parton showering. The envelope of the largest deviations from the
nominal sample with the tune eigenvariations applied is taken as the uncertainty
on the underlying event model.

% > Interference
The interference between \ac{EW} and \ac{QCD} \Zy production is not included in
either the signal or background, but instead taken as an additional uncertainty,
calculated using the dedicated interference sample.

\subsection{Experimental uncertainties}

\newcommand\both{ \checkmark & \checkmark }
\newcommand\justVBS{ \checkmark & }
\newcommand\justVZy{ & \checkmark }
\newcommand\systtabendwidth{5.5cm}
\newcommand\systtabletitles{%
  \multirow{2}{*}{Uncertainty name} & \multicolumn{2}{c}{Analysis} &
  \multirow{2}{*}{Accounts for uncertainty on} \\\cmidrule{2-3}
  & {\fontsize{8}{8}\selectfont VBS} & {\fontsize{8}{8}\selectfont \VZy} & \\
}
\begin{table}[p]
  \centering
  \caption{
    List of experimental systematic uncertainties, whether they are included in
    the \acs{VBS} \Zy and semileptonic \VZy analyses, and a brief description of
    what the uncertainty represents. The names of uncertainties are often
    abbreviated when shown in figures.
  }
  \input{\relpath{syst_table.tex}}
  \raggedleft
  \emph{continued on next page}
  \label{tab:methods-systematics-list}
\end{table}
\begin{table}[p]
  \centering
  Table \ref{tab:methods-systematics-list} continued \\[.5em]
  \input{\relpath{syst_table2.tex}}
\end{table}

Experimental systematic uncertainties cover errors in energy scale and
resolution of jets, photons, and electrons; momentum scale and resolution of
muons; scale factors used to reproduce trigger, reconstruction, identification,
and isolation efficiencies from data; suppression of pile-up jets; and flavour
tagging. The full list of experimental systematics considered
between the two analyses is given in Table \ref{tab:methods-systematics-list}.
The primary difference between the two analyses is that the \VZy analysis has no
flavour tagging systematics, as no b-tagging is used; though there are other
small changes due to a change in jet uncertainty configuration. These
uncertainties have a varying level of effect on the presented results, some of
the most impactful are discussed here.

\subsubsection{Pileup reweighting}
\label{sec:methods-systematics-prw}

\ac{MC} samples are typically generated before data-taking is complete. The
pileup distribution, i.e. the distribution of instantaneous luminosities, is
%TODO reference pileup distribution from detector chapter?
therefore only estimated and does not exactly match that in data. Events are
reweighted to align the pileup distributions between \ac{MC} and data; a scale
factor is calculated \cite{Buttinger2017} to account for the difference between
the predicted and measured inelastic proton-proton cross section
\cite{ATLASppxsec2016}.

The systematic uncertainty \verb|PRW_DATASF| represents the uncertainty on this
scale factor, and so also covers uncertainties from the inelastic cross-section
measurement and \ac{MC} pileup event modelling.

Due to limited data statistics in the signal regions, this uncertainty is one
of the most significant components of the error on the results of both
analyses.

\subsubsection{Jet flavour composition and response}
\label{sec:methods-systematics-jet-flavour}

Jets initiated by different quarks and by gluons exhibit differences in
fragmentation and showering properties. These properties will impact the jet
energy scale calibration, so the distribution of light-quark-, b-quark-,
and gluon-initiated jets, i.e. the distribution of jet flavour, and
its uncertainty affects the overall jet energy scale uncertainty.

The response of the calorimeter to different flavours of jet is not well
modelled in \ac{MC}, and so is corrected using comparisons with data.
Uncertainties on this correction are propagated as a `jet flavour response'
systematic uncertainty.

The jet response is itself dependent on the flavour composition of jets in the
\ac{MC} sample. This composition is dependent on the selection, so any jet
selection differing from those in the jet calibration schemes will not have a
well defined flavour composition. Uncertainties on the jet flavour composition
within the phase space are taken as a systematic uncertainty on analyses
\cite{JetFlavourUncerts2011}.

The gluon fraction is defined as
\begin{equation}
  f_g = \frac{N_g}{N_g + N_\text{LQ}},
  \label{eqn:methods-systematics-fgluon}
\end{equation}
where $N_g$ is the number of gluon-initiated jets in the phase space and
$N_\text{LQ}$ the number of light-quark-initiated jets.
This gluon fraction is used to determine the jet flavour uncertainties, but by
default its value is taken as
\begin{equation*}
  f_g = 0.5 \pm 0.5.
\end{equation*}
Therefore both jet flavour response and jet flavour composition uncertainties
can be reduced by explicitly calculating this gluon fraction and its error in
the analysis phase space. This additional step is taken in the \ac{VBS} analysis
to manage these uncertainties.
