% ========== Systematics section ==========

Many sources of uncertainty are considered for processes estimated in the
presented analyses. The subsections below cover uncertainties from theoretical
and experimental sources which are considered for \ac{EW} and \ac{QCD} \Zyjj
production processes.
Smaller backgrounds from \ac{MC} estimates use a simple overall uncertainty on
the normalisation: 15\% for \tty and 20\% for WZ$jj$. The Z+jets background has
a normalisation uncertainty calculated per-analysis considering the
contributions detailed in Section
\ref{sec:methods-backgrounds-zjets-uncertainty}. Limited statistics in \ac{MC}
samples also contributes uncertainties to all processes.

All systematic uncertainties are included as nuisance parameters in the fit used
for each analysis, and the final impact of each uncertainty is adjusted
according to the data.

% Theoretical uncertainties
\subsection{Theoretical uncertainties}

Theoretical uncertainties are calculated for \ac{EW} and \ac{QCD} \Zy production
mechanisms. These come from a variety of sources: choice of \ac{PDF} set,
renormalisation and factorisation scales, \ac{QCD} modelling, choice of parton
showering and underlying event model, and \ac{EW}-\ac{QCD} interference.

% > PDF
Evaluating uncertainty in \ac{PDF} set choice is done by reweighting events
using a number of replica \ac{PDF} sets. Taking the standard deviation of yields
under each of these weights gives the \ac{PDF} uncertainty on the event yield. 

% > Scale
Uncertainty due to scale choice is calculated by varying the default values of
renormalisation and factorisation scales in the nominal QCD \Zy \ac{MC} sample.
Dedicated samples (as discussed in Section \ref{sec:methods-samples}) are used
to evaluate the effect of varying these two values. The per-bin envelope of all
deviations is taken as the scale uncertainty.

% > QCD modelling
\ac{QCD} modelling uncertainty can be calculated conservatively by comparing
predictions from different generators or, more precisely, by evaluating the
effect of merging and resummation scales. Generator differences are calculated
by taking the difference in event yields predicted by the nominal and alternate
\ac{QCD} \Zy samples. This difference is considered as the \ac{QCD} modelling
uncertainty on the nominal yield.
Alternately, uncertainty from choice of merging (CKKW) and
resummation (QSF) scale is calculated using the dedicated samples described in
Section \ref{sec:methods-samples}. The latter method is used for the \ac{VBS}
analysis whilst the former is used for the triboson analysis.

% > EW PS&UE
For the \ac{EW} signal, parton showering and underlying event uncertainties are
calculated by comparing the default \pythia hadronisation to alternatives with
\herwig or with a varied \pythia tune.
%TODO add details on this tune to samples section?
%TODO write about tune in theory?

% > Interference
The interference between \ac{EW} and \ac{QCD} \Zy production is not included in
either the signal or background, but instead taken as an additional uncertainty,
calculated using the dedicated interference sample.


\subsection{Experimental uncertainties}

Experimental systematic uncertainties cover uncertainties in energy scale and
resolution of jets, photons, and electrons; momentum scale and resolution of
muons; scale factors used to reproduce trigger, reconstruction, identification,
and isolation efficiencies from data; suppression of pile-up jets; and flavour
tagging. These uncertainties have a varying level of effect on the presented
results, some of the most impactful uncertainties are discussed here.

\subsubsection{Pileup reweighting}

\ac{MC} samples are typically generated before data-taking is complete. The
pileup distribution, i.e. the distribution of instantaneous luminosities, is
therefore only estimated and does not exactly match that in data. Events are
reweighted to align the pileup distributions between \ac{MC} and data. The scale
factors used to achieve this reweighting are calculated with a corresponding
uncertainty. This uncertainty is propagated to physics results as a systematic
uncertainty.

\subsubsection{Jet flavour composition and response}
\label{sec:methods-systematics-jet-flavour}

Jets initiated by quarks and gluons exhibit differences in fragmentation and
showering properties. These properties will impact the jet energy scale
calibration and so the distribution of quark- and gluon-initiated jets and its
uncertainty affects the overall jet energy scale uncertainty. Whether a jet is
quark- or gluon-initiated is referred to as the flavour of the jet.

The response of the calorimeter to different flavours of jet is not well
modelled in \ac{MC}, and so is corrected using comparisons with data.
Uncertainties on this correction are propagated as a `jet flavour response'
systematic uncertainty.

The jet response is itself dependent on the flavour composition of jets in the
\ac{MC} sample. This composition is dependent on the selection, so any jet
selection differing from those in the jet calibration schemes will not have a
well defined flavour composition. Uncertainties on the jet flavour composition
within the phase space are taken as a systematic uncertainty on analyses
\cite{JetFlavourUncerts2011}.

Both jet flavour response and jet flavour composition uncertainties can be
reduced by studying the fraction of quark and gluon jets in the phase space.
This additional step is taken in the \ac{VBS} analysis to manage these
uncertainties.

%TODO write about fragmentaion in theory? Or in object reco
