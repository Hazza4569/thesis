% ========== Event selection section ==========

Events from data and simulation undergo a selection process to focus on a phase
space that matches the desired final state. This selection applies to data
samples as described in Section \ref{sec:methods-data} and \ac{MC} samples as in
Section \ref{sec:methods-mc}, with objects reconstructed following the procedure
in Section \ref{sec:methods-reconstruction}.

For an event to be selected, first the basic objects in the desired final state
need to be present. There must be at least one photon and precisely two
electrons or muons present, of the same flavour to each other but opposite
charge, with all of these passing the relevant isolation and identification
criteria specified in Section \ref{sec:methods-reconstruction}.  Both analyses
also require the presence of two jets, but their selection varies per analysis
and is discussed separately there.

Further selection is applied to the lepton-photon system in order to identify
events with a real Z boson and a photon not produced from \ac{FSR}. This \Zy selection is
detailed in Table \ref{tab:anacom-zy-selection}, and acts as a pre-selection for
both analyses before additional jet selection is applied.

\begin{table}
  \centering
  \renewcommand\arraystretch{1.3}
  \begin{tabular}{p{6em}l}
    \hline\hline
    \multicolumn{2}{c}{\Zy selection} \\
    \hline
    Photon & $N_\gamma \geq 1$ \\
           & $|\eta_\gamma| < 2.37$ \\
           & (excludes $1.37 < |\eta_\gamma| < 1.52$) \\
           & $p_T^\gamma > 25$ GeV \\
    \hline
    Lepton & $N_l = 2$ (OSSF)\\
           & $|\eta_e| < 2.47$ \\
           & (excludes $1.37 < |\eta_e| < 1.52$) \\
           & $|\eta_\mu| < 2.5$ \\
           & $p_T^{l,1} > 30$ GeV \\
           & $p_T^{l,2} > 20$ GeV \\
    \hline
    Boson  & $m_{ll} > 40$ GeV \\
           & $m_{ll} + m_{ll\gamma} > 182$ GeV \\
    \hline\hline
  \end{tabular}
  \caption{
    Cuts implemented for both analyses to select \Zy events. Here $p_T^{l,1}$
    denotes the $p_T$ of the leading (i.e. highest $p_T$) lepton, and
    $p_T^{l,2}$ denotes that of the sub-leading (second highest $p_T$) lepton.
    OSSF indicates that two opposite-sign same-flavour leptons are required.
  }
  \label{tab:anacom-zy-selection}
\end{table}

% Discuss FSR cut
The cut on the sum of the dilepton mass and the dilepton-photon mass, $m_{ll} +
m_{ll\gamma}$ is imposed to reject events with photons from \ac{FSR}.  In an
\ac{FSR} event, the photon is radiated from one of the final state leptons.
This means the two leptons and the photon all originate from the same Z boson,
and their invariant mass should be close to the mass of the Z boson. The
invariant mass of the dilepton system in this case would be less, and so the sum
of these masses should be less than twice the Z mass.  In a non-\ac{FSR} event,
the photon is radiated independently of the Z boson production, giving
a dilepton mass close to the Z mass and a larger dilepton-photon invariant mass.
The sum of the masses in this instance is typically greater than twice the Z
mass.  Figure \ref{fig:anacom-fsr-plot} shows a two-dimensional distribution of
these two invariant masses and how this cut rejects the population of events
with \ac{FSR} photons.

\begin{figure}[h]
  \centering
  \includegraphics[width=.95\textwidth]{\resource{FSR_cut_plot}}
  \caption{
    Distribution of events as a function of $m_{ll}$ and $m_{ll\gamma}$, for
    events in the signal sample passing the lepton and photon cuts given in
    Table \ref{tab:anacom-zy-selection}. The dashed line shows the threshold for
   the \acs{FSR}-rejection cut, events below the dashed line are discarded.
  }
  \label{fig:anacom-fsr-plot}
\end{figure}

This \ac{FSR} rejection is implemented since the photon emission from a final
state lepton excludes the possibility of the photon having been produced in a
multiboson interaction, which is the focus of these analyses.
