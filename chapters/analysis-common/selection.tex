% ========== Event selection section ==========

Events from data and simulation undergo a selection process to focus on a phase
space that matches the desired final state. Data events are required to pass a
trigger to be recorded, and therefore the same requirement is placed on
simulated events for parity. For both analyses, any events passing the single
lepton or dilepton trigger are considered.

For an event to be selected, first the basic objects in the desired final state
need to be present, subject to some reconstruction criteria. There must be at
least one photon passing isolation and identification requirements. There must
be precisely two electrons or muons present, of the same flavour to each other
but opposite charge, with both passing isolation and identification. At least
two jets must have been reconstructed in the event.

Further selection is applied to the lepton-photon system in order to identify
events with a real Z boson and an \ac{ISR} photon. This \Zy selection is
detailed in Table \ref{tab:anacom-zy-selection}, and acts as a pre-selection for
both analyses before additional jet selection is applied.

\begin{table}
  \centering
  \begin{tabular}{c}
    \hline\hline
    \Zy selection \\
    \hline
    $p_T^\gamma > 25$ GeV \\
    $p_T^{l,1} > 30$ GeV \\
    $p_T^{l,2} > 20$ GeV \\
    $m_{ll} > 40$ GeV \\
    $m_{ll} + m_{ll\gamma} > 182$ GeV \\
  \end{tabular}
  \caption{
    Cuts implemented for both analyses to select \Zy events. Here $p_T^{l,1}$
    denotes the $p_T$ of the leading (i.e. highest $p_T$) lepton, and
    $p_T^{l,2}$ denotes that of the sub-leading (second highest $p_T$) lepton.
  }
  \label{tab:anacom-zy-selection}
\end{table}
% TODO add eta cuts, check for anything else
% 2.37 photon and 2.47 lepton is correct btw

% Discuss FSR cut
The cut on the sum of the dilepton mass and the dilepton-photon mass, $m_{ll} +
m_{ll\gamma}$ is imposed to select events with \ac{ISR} photons rather than
photons from \ac{FSR}. In an \ac{FSR} event, the photon is radiated from one of
the final state leptons. This means the two leptons and the photon all originate
from the same Z boson, and their invariant mass should be close to the mass of
the Z boson. The invariant mass of the dilepton system in this case would be
less. In an \ac{ISR} event the photon is radiated before the Z boson is
produced, so this time the dilepton mass should be close to the Z mass and the
dilepton-photon invariant mass should be larger. Requiring the sum of these two
masses to be greater than twice the Z mass should isolate events from this
second instance. Figure X %TODO
shows a two-dimensional distribution of these two invariant masses and how this
cut selects events.

% TODO add ISR vs FSR Feynman diagrams
% TODO add FSR plot (need to make it from VZy samples :/ )
