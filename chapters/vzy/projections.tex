% ========== PROJECTED RESULTS SECTION ==========

To test what sensitivity might be possible with more time or data available for
this analysis, projected future results are explored. Firstly the reducible
uncertainties, jet flavour composition and response and \ac{MC} statistics, are
removed. This simulates processing larger datasets, to reduce \ac{MC} statistics
uncertainties, and including gluon fraction information, to reduce jet flavour
uncertainties. With the existing analysis and dataset this would give a
measurement of
%
\begin{equation*}
  \muEW = 1.43 \pm 1.08,
\end{equation*}
%
calculated from performing a fit without these uncertainties included.

This is a small improvement by itself, but combined with an expanded dataset
this could greatly enhance sensitivity. By scaling up the luminosity of the
templates in the fit, performing the analysis with a larger dataset can be
simulated. This is a naïve estimate as with a significantly larger dataset the
analysis would need to be re-optimised to take advantage of the available data.

% TODO explain LHC runs, Run 2 and Run 3, luminosities, energies
To estimate sensitivity possible with a Run 2 and Run 3 combined dataset,
templates are scaled to a luminosity of 420 fb$^{-1}$. The measured signal
strength from this fit is
%
\begin{equation*}
  \begin{split}
  \muEW &= 1.00 \pm 0.63 \\
  &= 1.00 \pm 0.53 \,(\text{stat.}) \pm 0.35 \,(\text{syst.}),
  \end{split}
  % mu_EWK  1.00017 +0.633561 -0.633561
  % mu_EWK  0.999978 +0.525063 -0.525063 (statonly)
\end{equation*}
%
corresponding to a significance of 2.09 standard deviations. This still falls
short of the evidence threshold of 3 standard deviations. Whilst proper
optimisation for the larger dataset might make obtaining evidence achievable,
the initial indication would be that additional data beyond Run 3 will be
needed to reach the 3$sigma$ threshold.

The largest systematic uncertainties for the 420 fb$^{-1}$ projected fit are
shown in Figure \ref{fig:vzy-projections-rank}. It is noticeably that pileup
reweighting is still the dominant systematic uncertainty, but much reduced from
its post-fit scale seen in Figure \ref{fig:vzy-results-ranking}. Indeed running
fits with luminosities scaled even higher sees further reduction in the impact
of this uncertainty; likely it is so dominant for the current analysis because
of the limited statistics in the phase space.

\begin{figure}[tbhp]
  \centering
  \includegraphics[width=.8\textwidth]{\resource{Ranking_mu_EWK_x3.pdf}}
  \caption{
    Systematic uncertainties ranked by their post-fit impact on \muEW, for a
    projected fit scaled to a luminosity of 420 fb$^{-1}$.
    Uncertainties from \ac{MC} statistics and jet flavour were removed from this
    fit.
  }
  \label{fig:vzy-projections-rank}
\end{figure}
