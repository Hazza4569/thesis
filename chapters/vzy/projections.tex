% ========== PROJECTED RESULTS SECTION ==========

\begin{figure}[tb]
  \centering
  \includegraphics[width=.8\textwidth]{\resource{Ranking_mu_EWK_x3.pdf}}
  \caption{
    Systematic uncertainties ranked by their post-fit impact on \muEW, for a
    projected fit scaled to a luminosity of 420 fb$^{-1}$.
    Uncertainties from \ac{MC} statistics and jet flavour were removed from this
    fit.
  }
  \label{fig:vzy-projections-rank}
\end{figure}

To test what sensitivity might be possible with further optimisations or
additional data available for this analysis, projected future results are
explored. Firstly the reducible uncertainties, jet flavour composition and
response and \ac{MC} statistics, are removed. This simulates processing larger
datasets, to reduce \ac{MC} statistics uncertainties, and including gluon
fraction information, to reduce jet flavour uncertainties. With the existing
analysis and dataset this gives a measurement of
%
\begin{equation*}
  \muEW = 1.43 \pm 1.08,
\end{equation*}
%
calculated from performing a fit without these uncertainties included.

This is a small improvement by itself, but combined with an expanded dataset
this could greatly enhance sensitivity. By scaling up the luminosity of the
templates in the fit, performing the analysis with a larger dataset can be
simulated. This is a naïve estimate as, with a significantly larger dataset, the
analysis would need to be re-optimised to take advantage of the available data.

To estimate sensitivity possible with a Run 2 and Run 3 combined dataset,
templates are scaled to a luminosity of 420 fb$^{-1}$. The measured signal
strength from this fit is
%
\begin{equation*}
  \begin{split}
  \muEW &= 1.00 \pm 0.63 \\
  &= 1.00 \pm 0.53 \,(\text{stat.}) \pm 0.35 \,(\text{syst.}),
  \end{split}
  % mu_EWK  1.00017 +0.633561 -0.633561
  % mu_EWK  0.999978 +0.525063 -0.525063 (statonly)
\end{equation*}
%
corresponding to a significance of 2.09 standard deviations. This falls
short of the evidence threshold of 3 standard deviations, but with a
re-optimised analysis this channel could get close to the sensitivity required. 
The potential sensitivity from adding a merged jet channel may be enough to give
a significant measurement for this process.

The largest systematic uncertainties for the 420 fb$^{-1}$ projected fit are
shown in Figure \ref{fig:vzy-projections-rank}. It is noticeable that pileup
reweighting is still the dominant systematic uncertainty, but much reduced from
its post-fit scale seen in Figure \ref{fig:vzy-results-ranking}. From running
fits with luminosities scaled beyond 420 fb$^{-1}$ a continued reduction in the impact
of this uncertainty is observed; the conclusion is that this uncertainty is
inflated by the small phase space of the analysis.

