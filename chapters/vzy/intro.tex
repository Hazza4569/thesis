% VZy introduction

%TODO add discussion of previous/related measurements

% intro and motivation
Triboson production of a Z boson; a photon; and an additional, hadronically
decaying, vector boson is the second production mechanism for the \Zyjj final
state explored in this thesis. This additional boson, denoted as a `V' boson,
can be a W or a Z boson.  Measuring this semileptonic V\Zy triboson process thus
constitutes an inclusive measurement of both W\Zy and Z\Zy triboson production.

\begin{figure}
  \centering
  \includegraphics[width=\textwidth]{\resource{vzy-feynman.pdf}}
  \caption{
    A selection of \acs{SM} production mechanisms for the V\Zy triboson final
    state, depicted in Feynman diagrams.
  }
  \label{fig:vzy-feynman}
\end{figure}

Figure \ref{fig:vzy-feynman} gives \ac{LO} Feynman diagrams for W\Zy and Z\Zy
production. Notably, W\Zy production is sensitive to the same \ac{QGC} and
\ac{TGC} vertices as \ac{VBS} \Zy production.
The Z\Zy process does not feature these interactions in the \ac{SM} description
as there are no neutral \acp{QGC} or \acp{TGC} in the model.
%
This makes the
combined semileptonic channel something of a hybrid, it is sensitive both to the
\ac{SM} multiboson interactions but also sensitive to any beyond-\ac{SM} physics
which might introduce these `anomalous' neutral couplings. As with the \ac{VBS}
\Zy analysis, any anomalous couplings introduced from new physics would affect
the rate of this process and, particularly if the cross section would be
enhanced, could result in measuring a significant deviation from the \ac{SM}
prediction.

The high number of electroweak interactions necessary at \ac{LO} to
facilitate this process means that \VZy production has a very low cross section,
similar to that of \ac{VBS} \Zy (see Figure \ref{fig:theory-sm-xsecs}).
Without the distinct \ac{VBS} jet signature to select on, measuring this low
cross-section process is challenging.

% kinematics and discrimination techniques
The two jets, here a product of a boson decay, have kinematic properties that
help
distinguish them from background events, notably: a dijet mass peaked around the
W/Z masses, small rapidity separation between the two jets,
and an angular distribution consistent with boson decay products.
%^TODO cite something on W/Z cos theta distributions, have a reference if
%needed.
Additionally, more subtle differences are also present in other variables.
Selecting W/Z boson decays to leptons is relatively straightforward, but the
more limited jet resolution and more dominant jet background makes doing this in
the hadronic decay channel more difficult.
This analysis employs machine-learning techniques to interpret this complicated
phase space; pushing sensitivity beyond what is achievable with a traditional
cut-based analysis.

%backgrounds
Despite the differing jet phase space, the backgrounds for this analysis include
the same processes as the \ac{VBS} \Zy analysis. \QCDZy production is the dominant
competing process; the key difference between this background and the signal
is the kinematics of the jets, as both have a real Z boson and photon produced.
%
The \QCDZy background has a yield 140 times larger than the signal after
applying preselection cuts (i.e. the analysis region selection as in Table
\ref{tab:vzy-selection}).  This illustrates the need for effective jet selection
to manage this background.

%Overall strategy/aim
The goal of this analysis is to measure the signal strength of this rare
process in order to compare it to the standard model expectation.  This
measurement is extracted from a template fit to the signal rarity
distribution, derived from the output of a machine-learning model. If the
observed significance is sufficient, this will provide evidence for, or an
observation of, this process. If the significance does not meet these
thresholds, the measurement will be used to place limits on the rate of this
process. These limits can be used to constrain theories that might enhance the
cross section of this process.

% Related measurements
This measurement represents the first of its kind, no measurements have been
published on semileptonic \VZy production. There is however some overlap with
other published measurements. Of the two included processes, W\Zy and Z\Zy, W\Zy
has been observed by \ac{ATLAS} through fully leptonic decay modes
\cite{ATLASwzy2023} and studied in a semileptonic final state in a \ac{CMS}
VW$\gamma$ measurement \cite{CMSvwy2014}, using a similar principle to this
analysis with a generic hadronically decaying massive boson. However, no
measurements have been published of the Z\Zy final state. These measurements
contribute to the broader study of \ac{EW} triboson processes, which includes
recent measurements of
VVV \cite{CMSvvv2020}, 
WWW \cite{ATLASwww2022},
Z$\gamma\gamma$ \cite{ATLASzyy2023,CMSwyyzyy2021},
and W$\gamma\gamma$ \cite{ATLASwyy2023,CMSwyyzyy2021}
processes.
%WW$\gamma$ \cite{CMSwwy2023}%
%\footnote{
%  This measurement is inclusive of phase spaces where the photon originates
%  from \ac{FSR}, whether this should be considered a triboson process is open to
%  debate.
%}%

% + Overlapping FS triboson measurements
%   ATLAS
%   > WZy (6.3sigma) \cite{ATLASwzy2023}
%   CMS
%   > VWy (Run 1, limit 3.4xSM expectation) \cite{CMSvwy2014}
% + Additional observed triboson measurements
%   ATLAS
%   > WWW \cite{ATLASwww2022}
%   CMS
%   > WWy \cite{CMSwwy2023} (includes FSR)
%   > VVV \cite{CMSvvv2020}
%   

% Chapter breakdown
The remainder of this chapter details the different elements of this analysis.
An initial event selection and the definition for the signal process is given in
Section \ref{sec:vzy-selection}. Before the full \ac{SR} selection can be
introduced, the development of a \ac{BDT} discriminant is discussed in Section
\ref{sec:vzy-bdt}; this section motivates the need for this by first creating a
cut-based selection for the analysis. Section \ref{sec:vzy-srcr} then defines
the full \ac{SR}, and some \acp{CR}, by making use of the \ac{BDT} output.
Background estimation procedures are reviewed in Section \ref{sec:vzy-bg} before
the systematic uncertainties are discussed in Section \ref{sec:vzy-systematics}.
Section \ref{sec:vzy-fit} details the fitting procedure used to extract the
measurement from data before the results are given in Section
\ref{sec:vzy-results}. The discussion concludes with projections of future
results and extensions to this analysis in Sections \ref{sec:vzy-projection} and
\ref{sec:vzy-extensions}.


%TODO section in common analysis methods about fiducial cross-sections
%TODO section in theory about BSM theories which affect cross-sections.
