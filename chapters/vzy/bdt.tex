% ==== DISCRIMINATING AGAINST QCD Zy PRODUCTION SECTION ====

The biggest challenge in this analysis is managing the dominant background,
\ac{QCD} \Zyjj production. Like the signal process, this background has a real Z
boson and photon. The difference is the origin of the jets, here not from a
boson decay but more likely radiated from the initial or final state.
%
Identifying and exploiting the differences in jet kinematics between this
background and the signal is therefore key to maximising the sensitivity of the
measurement. This section is dedicated to discussing this problem; the word
`signal' is used here to refer to \ac{EW} V\Zy production and `background'
refers solely to \ac{QCD} \Zyjj production.

%TODO give Feynman diagrams? Or reference previous Feynmans

% Talk about kinematic differences
There are a small number of kinematic distributions which exhibit a large
difference between signal and background that could be exploited effectively by
a cut. The dijet mass, $m_{jj}$, is an obvious example as for the signal it
peaks around the W/Z boson mass but for the background resembles a continuum.

For many more variables however, the differences are more subtle. There may be
an obvious difference in shape between signal and background
but there is no obvious cut or set of cuts that would create a signal-rich
region. Figure \ref{fig:vzy-bdt-ewvqcd} shows some distributions with the
largest signal-background discrepancies.

% m_jj and cos_theta_CS_jj show cuttable differences

\begin{figure}[tbp]
  \includegraphics[width=.49\textwidth]{\resource{EWvQCD/m_jj.pdf}}
  \includegraphics[width=.49\textwidth]{\resource{EWvQCD/cos_theta_CS_jj.pdf}}
  \\
  \includegraphics[width=.49\textwidth]{\resource{EWvQCD/Dy_j_j.pdf}}
  \includegraphics[width=.49\textwidth]{\resource{EWvQCD/Dphi_j_j.pdf}}
  \\
  \includegraphics[width=.49\textwidth]{\resource{EWvQCD/pT_jj.pdf}}
  \includegraphics[width=.49\textwidth]{\resource{EWvQCD/Dphi_lly_jj.pdf}}
  %
  \caption{
    Kinematic distributions, comparing \acs{EW} V\Zy production (red) to \acs{QCD}
    \Zyjj production (blue). Generated from the corresponding \ac{MC} samples
    with V\Zy preselection applied. Events are normalised to compare the shape
    of distributions between the two samples.
  }
  \label{fig:vzy-bdt-ewvqcd}
\end{figure}

Building a cut-based selection with sensitivity to the signal is difficult, more
advanced methods might push the background rejection further. This section
explores and compares two methods for defining a signal-sensitive phase space
for the analysis: a cut-based approach and a \ac{BDT}, a machine learning
classifier introduced in Section [BDT section in theory chapter]. %TODO

The dijet mass variable is excluded from being used for selection in either of
these methods. This allows it to be used to define \acp{CR} with a low signal
purity in order to validate background estimates with comparisons to data. For
more detail on the definition and use of these \acp{CR}, see Section X. %TODO

These initial investigations were performed before many details of the analysis
were established and so have a unique phase space, detailed below.

% Section giving samples and phase-space for this study
\subsection{Phase space for preliminary studies}

The studies presented in this section use events from the \ac{EW} \VZy sample
(as defined in Section \ref{sec:vzy-selection-vzy}) as the signal and from the
\ac{QCD} \Zy sample as the background. All events are subject to the
preselection in Table \ref{tab:vzy-bdt-preliminaryselection}.  These cuts select
\Zy events with an earlier version of the full \Zy selection presented in
Section \ref{sec:methods-selection}.  No cuts are placed on the jets at this
stage.  Isolation, identification, and overlap removal for all objects are the
same as discussed in Section \ref{sec:methods-selection}.

\begin{table}
  \centering
  \renewcommand\arraystretch{1.3}
  \caption{
    Selection for events used in background rejection studies for the \VZy
    triboson analysis. This is the same as the \Zy selection in Table
    \ref{tab:anacom-zy-selection} but with a looser photon $p_T$ cut and no
    \acs{FSR} cut.
  }
  \begin{tabular}{p{6em}l}
    \hline\hline
    \multicolumn{2}{c}{Background rejection studies preselection} \\
    \hline
    Photon & $N_\gamma \geq 1$ \\
           & $|\eta_\gamma| < 2.37$ \\
           & (excludes $1.37 < |\eta_\gamma| < 1.52$) \\
           & $p_T^\gamma > 15$ GeV \\
    \hline
    Lepton & $N_l = 2$ (OSSF)\\
           & $|\eta_e| < 2.47$ \\
           & (excludes $1.37 < |\eta_e| < 1.52$) \\
           & $|\eta_\mu| < 2.5$ \\
           & $p_T^{l,1} > 30$ GeV \\
           & $p_T^{l,2} > 20$ GeV \\
    \hline
    Boson  & $m_{ll} > 40$ GeV \\
    \hline\hline
  \end{tabular}
  \label{tab:vzy-bdt-preliminaryselection}
\end{table}

\subsection{Comparison metric}
\label{sec:vzy-bdt-significance}

A metric is needed in order to evaluate the performance of a given selection.
Since the desired selection will be one that grants the most sensitivity to the
\VZy signal, a significance of the signal considering a background-only
hypothesis is used. This will emulate the significance calculation used for the
final measurement, though much simplified as it deals with only a single
background and no systematic uncertainties. Whilst the significances given here
are not comparable to that from a full measurement, they are comparable with
each other and will indicate which selection generates more sensitivity to the
signal process.

\newcommand\nobsi{\ensuremath{n_\text{obs}^i}\xspace}
As the $m_{jj}$ distribution is not used for selection, it is used here to
calculate significance with a binned likelihood method. Consider $n_b^i$ as the
number of background events in bin $i$, and $n_s^i$ the number of signal events,
from the relevant \ac{MC} samples after selection.  The likelihood of observing
\nobsi events in bin $i$ is given by a Poisson distribution with a mean of
$n_b^i + \mu\cdot n_s^i$, where $\mu$ is a signal strength parameter with
$\mu=0$ for the background-only hypothesis or $\mu=1$ for alternate hypothesis
with signal included. The likelihood, $\mathcal{L}$, of observing the set of
$\{\nobsi\}$ in each bin is then the product of each of the per-bin likelihoods.

Constructing the likelihood ratio
\begin{equation*}
  \mathscr{\lambda} = \frac{ \mathcal{L}( \{\nobsi\} ; \mu=0 ) }
                           { \mathcal{L}( \{\nobsi\} ; \mu=1 ) },
\end{equation*}
enables a likelihood-ratio test, with the test statistic $-2\ln\lambda$
distributed as $\chi^2(1)$ \cite{Wilks1938}, to calculate the significance of
observing $\{\nobsi\}$.

To obtain integer values for \nobsi from the \ac{MC} prediction, as required by
the use of the Poisson distribution, random toy experiments are run. For each
experiment, \nobsi is picked at random from a Poisson distribution with mean
$n_b^i + n_s^i$. Running 1000 of these experiments, calculating the significance
for each, and taking the mean of the results gives an estimate for the
significance.

These significances are calculated for each selection tested, given as a number
of standard deviations.
%TODO somewhere in theory will have some discussion of significances and
%hypothesis testing. Might have to reduce or adjust this section once that is
%written.

\subsection{Selection variables}

Building a selection to reject the \ac{QCD} \Zy background relies on identifying
differences in jet kinematics, and therefore placing selection requirements on
jet-based kinematic variables. A number of variables are considered, with the
full list given in Table \ref{tab:vzy-bdt-variables}.

\newcommand\ptbalance{\ensuremath{p_T^\text{balance}}\xspace}

\begin{table}[!p]
  \centering
  \renewcommand\arraystretch{1.3}
  \caption{
    Variables considered for selection to reject \ac{QCD} \Zy events for the
    \VZy triboson analysis.
  }
  \begin{tabular}{c|p{10cm}}
    \hline\hline
    Variable & Definition \\
    \hline
    $y_{j,1}$ &
    Rapidity of the leading jet in the event.
    \\
    $y_{j,2}$ &
    Rapidity of the sub-leading jet in the event.
    \\
    $y_{jj}$ &
    Rapidity of the $jj$ system.
    \\
    $p_T^{j,1}$ &
    Transverse momentum of the leading jet in the event.
    \\
    $p_T^{j,2}$ &
    Transverse momentum of the sub-leading jet in the event.
    \\
    $p_T^{jj}$ &
    Transverse momentum of the $jj$ system.
    \\
    \ptbalance &
    Relative difference between transverse momenta of the $jj$ and $ll\gamma$
    systems, given by Equation \ref{eqn:vzy-bdt-ptbalance}.
    \\
    $N_j$ &
    Number of jets in the event, reconstructed with a minimum $p_T$ of 25 GeV.
    \\
    $N_j^\text{gap}$ &
    Number of jets, satisfying $p_T > 25$ GeV found in the rapidity region
    between the two leading jets.
    \\
    $m_{j,1}$ &
    Mass of the leading jet in the event.
    \\
    $m_{j,2}$ &
    Mass of the sub-leading jet in the event.
    \\
    $m(ll\gamma jj)$ &
    Mass of the triboson system.
    \\
    $|\Delta y_{jj}|$ &
    Absolute rapidity difference between the two leading jets.
    \\
    $\Delta\phi_{jj}$ &
    Smallest difference between the azimuthal angles of the two leading jets.
    \\
    $\Delta R_{jj}$ &
    $\Delta R$ value between the two leading jets.
    \\
    $|\Delta y(ll\gamma, jj)|$ &
    Absolute rapidity difference between the $ll\gamma$ and $jj$ systems.
    \\
    $\Delta\phi(ll\gamma, jj)$ &
    Smallest difference between the azimuthal angles of the $ll\gamma$ and $jj$
    systems.
    \\
    $\Delta R(ll\gamma, jj)$ &
    $\Delta R$ value between the $ll\gamma$ and $jj$ systems.
    \\
    $\Delta R_\text{min}(\gamma, j)$ &
    Minimum $\Delta R$ value between any photon and jet in the event.
    \\
    $\cos{\theta^*(jj)}$ &
    Cosine of $\theta^*(jj)$, the angle of the leading jet in the dijet
    centre-of-mass frame relative to the direction of motion of the $jj$ system.
    \\
    $\cos{\theta_\text{CS}(jj)}$ &
    Cosine of $\theta_\text{CS}(jj)$, the angle between the two jets in the
    Collins-Soper frame \cite{Collins1977}. Jet charge information isn't
    available so the angle is taken relative to the leading jet.
    \\
    $\zeta(ll\gamma)$ &
    Centrality of the $ll\gamma$ system, given by Equation
    \ref{eqn:vbs-selection-centrality}.
    \\
    \hline\hline
  \end{tabular}
  \label{tab:vzy-bdt-variables}
\end{table}

The variable \ptbalance is given by the equation

\begin{equation}
  \ptbalance = \frac{ (p_T^{jj} - p_T^{ll\gamma}) }
                    { (p_T^{jj} + p_T^{ll\gamma}) }.
  \label{eqn:vzy-bdt-ptbalance}
\end{equation}

% Section describing cut-based selection optimisation
\subsection{Cut-based background rejection}
% See plots here, from old codebase
% /mnt/naf/code/VZy-prep/extract_VZy/build/signifscan_nomjj2_SR1.pdf

The task at hand is to find a set of cuts to make, on variables from Table
\ref{tab:vzy-bdt-variables}, in order to maximise sensitivity to the signal
process. Truly optimising this, finding the best value for each cut given the
values of every other cut, is a many-dimensional problem with no reasonable
solution. Instead an iterative approach is taken: find the best cut on each
variable individually, take the variable with the cut gives the best improvement
in sensitivity and add it to the selection, then re-test all other cuts on the
new subset of events.

Identifying the `best' cut to make at any stage is a little subjective. For
instance, when applying the first cut, the selection that would result in the best
significance for the signal sample is likely too aggressive to allow for
multiple effective cuts afterwards. The method used is to calculate background
rejection ($1/$fraction of background events passing a cut) as a function
of signal efficiency (fraction of signal events passing a cut) for each
variable. By eye, these distributions can then be scanned to identify a possible
cut which gives large background rejection but maintains a high signal
efficiency. This allows for multiple variables to be included in the selection
before the phase space becomes too constrained.

\begin{figure}[tbhp]
  \centering
  \includegraphics[width=.48\textwidth]{\resource{integral_pT_j2_gt_fix.pdf}}
  \includegraphics[width=.48\textwidth]{\resource{rejection_pT_j2_gt_fix.pdf}}
  \caption{
    Distributions to identify a cut on $p_T^{j,2}$. Shown are fraction of events
    for each sample that are above a given threshold value in $p_T^{j,2}$ (left)
    and background rejection as a function of the signal efficiency achievable
    using the same $p_T^{j,2}$ threshold (right).
  }
  \label{fig:vzy-bdt-rejection}
\end{figure}

Figure \ref{fig:vzy-bdt-rejection} shows the background rejection against signal
efficiency for $p_T^j{j,2}$, which is the variable chosen to apply a cut on
first. A cut of $p_T^{j,2} > 35$ GeV is chosen, with a signal efficiency of 74\%
and a background rejection factor of 2.6.

Continuing this process, the most performant selection found consisted of five
cuts, listed in Table \ref{tab:vzy-bdt-cutbased}. Using the method described in
Section \ref{sec:vzy-bdt-significance}, the significance calculated for events
passing this selection is 1.2 standard deviations.

\begin{table}[tbh]
  \centering
  \renewcommand\arraystretch{1.3}
  \caption{
    Selection derived for baseline cut-based version of the analysis. Cuts are
    applied to the \VZy signal sample and the \ac{QCD} \Zy background for events
    passing the preliminary selection given in Table
    \ref{tab:vzy-bdt-preliminaryselection}.
  }
  \begin{tabular}{c}
    \hline\hline
    Cut-based selection \\
    \hline
    $p_T^{j,2} > 35 GeV$ \\
    $|\Delta y_{jj}| < 1.5$ \\
    $\Delta R(ll\gamma, jj) > 3.0$ \\
    $\Delta\phi(ll\gamma, jj) > 2.8$ \\
    $\ptbalance > -0.1$ \\
    \hline\hline
  \end{tabular}
  \label{tab:vzy-bdt-cutbased}
\end{table}


% Section describing BDT optimisation
\subsection{\acs{BDT} for background rejection}

The cut-based selection provides a baseline performance against which to
evaluate a \ac{BDT}-based selection. The \ac{BDT} can take many variables as
input and determine how likely an event is to be signal or background based on
the value of those variables, having first learned how the variables are
distributed differently between signal and background events.

The first step is to train a \ac{BDT} to identify these differences between
signal and background. Once trained, the \ac{BDT} is tested on an independent
set of events to evaluate its performance and test for overtraining.
To accommodate this train-test cycle, the signal and background samples are each
split evenly into two, one half used for training and the other for testing.

%TODO remove this note:
[Overtraining will be discussed in theory section]
% Something to fill in the gap here?

Several aspects of the \ac{BDT} are tuned to improve performance: the input
variables used by the \ac{BDT}, preselection applied to events before training,
and hyper-parameters of the \ac{BDT} itself. These are discussed in the sections
below.

%What to write?
% Input variable optimisation
\subsubsection{Input variable selection}
\label{sec:vzy-bdt-variables}

The benefit of the \ac{BDT} is its ability to handle many input variables and
generate a phase space sensitive to the signal. However, giving too many
variables to the \ac{BDT} creates an overly complex model and allows for
overtraining. Many iterations of input variables were tested to find a set that
is sufficiently small to prevent overtraining but with enough variables to allow
the \ac{BDT} to maximise the sensitivity.

For each set of variables tested, a simple overtraining check is used. For a
cut on the \ac{BDT} output resulting in a background rejection factor of 10, the
corresponding signal efficiency is compared between the training sample and the
test sample. Overtraining would result in a higher signal efficiency in the
training sample than in the test sample. A requirement that the test sample
signal efficiency is within $10\%$ of the training sample is used to mitigate
overtraining in the \ac{BDT} model.

Sensitivity attained by a \ac{BDT} trained on a given variable set is evaluated
by calculating the significance through the method discussed in Section
\ref{sec:vzy-bdt-significance}. To do this, a cut must first be placed on the
\ac{BDT} output. The value chosen for this cut will affect the sensitivity, so
in each instance many cut values are tested and the highest significance is
taken to represent the sensitivity of the \ac{BDT}.

After using these tests to compare many combinations of variables, the most
performant set was chosen.  The final set of 16 input variables is shown in
Table \ref{tab:vzy-bdt-ranking} ranked by their `importance' as determined by
the \ac{BDT}. See section [BDT section in theory chapter] for details on how
variable importance is calculated.

  %TODO add this (more detail?) to theory section
%Variable importance is based on the number of times that variable
%is used to split nodes in the decision tree, the separation gain from each
%split, and the number of events in the split nodes \cite{TMVAguide}.

\begin{table}[tbh]
  \centering
  \renewcommand\arraystretch{1.2}
  \caption{
    Ranking of variables used by the \ac{BDT} to discriminate between signal and
    background for the \VZy analysis.
  }
  \begin{tabular}{lp{4cm}r}
    \hline \hline
    Rank & Variable       & Relative importance\\
    \hline
    1  & $|\Delta y_{jj}|$          & $7.46\times10^{-2}$ \\
    2  & $p_T^{j,2}$                & $7.27\times10^{-2}$ \\
    3  & $\Delta\phi_{jj}$          & $7.24\times10^{-2}$ \\
    4  & $m_{j,2}$                  & $7.06\times10^{-2}$ \\
    5  & \ptbalance                 & $7.05\times10^{-2}$ \\
    6  & $\Delta R_\text{min}(y,j)$ & $6.50\times10^{-2}$ \\
    7  & $y_{j,2}$                  & $6.32\times10^{-2}$ \\
    8  & $\Delta\phi(ll\gamma, jj)$ & $6.15\times10^{-2}$ \\
    9  & $\cos\theta_\text{CS}(jj)$ & $6.10\times10^{-2}$ \\
    10 & $p_T^{j,1}$                & $5.76\times10^{-2}$ \\
    11 & $y_{j,1}$                  & $5.70\times10^{-2}$ \\
    12 & $p_T^{jj}$                 & $5.68\times10^{-2}$ \\
    13 & $\Delta R(ll\gamma,jj)$    & $5.68\times10^{-2}$ \\
    14 & $m_{j,1}$                  & $5.60\times10^{-2}$ \\
    15 & $\log{\zeta(ll\gamma)}$    & $5.48\times10^{-2}$ \\
    16 & $y_{jj}$                   & $4.96\times10^{-2}$ \\
    \hline\hline
  \end{tabular}
  \label{tab:vzy-bdt-ranking}
\end{table}

The logarithm of the centrality, $\zeta(ll\gamma)$, is taken rather than the
centrality itself as it extends to very high values. The method of binning
variables employed by the \ac{BDT} left little available discrimination power
with the original form of the variable. Figure \ref{fig:vzy-bdt-centrality}
shows the effect of the logarithm.

\begin{figure}[tbhp]
  \centering
  \includegraphics[width=.48\textwidth]{\resource{EWvQCD/Zy_centrality.pdf}}
  \includegraphics[width=.48\textwidth]{\resource{EWvQCD/log_Zy_centrality.pdf}}
  \caption{
    Distribution of centrality, $\zeta(ll\gamma)$, both without (left) and with
    (right) the logarithm applied. Normalised events are shown for the \VZy
    signal sample and the \ac{QCD} \Zy background.
  }
  \label{fig:vzy-bdt-centrality}
\end{figure}

% Training cuts
\subsubsection{Preselection and training cuts}

Another route to improving performance of the \ac{BDT} is constraining the phase
space to further simplify the signature the \ac{BDT} is trying to identify. Even
in cases where there is no performance increase, reducing the phase space
without significant loss in signal efficiency is still beneficial as it can help
to reduce the impact of systematic uncertainties. It also improves the
interpretability of the analysis phase space; cuts on simple kinematic variables
are much more easily understood than a cut on a \ac{BDT} output.

Two types of selection are used for this purpose: preselection applied to all
events, including those input to the \ac{BDT}, and training cuts applied only
to events when provided to the \ac{BDT} for training. Preselection will narrow
the whole analysis phase space whilst training cuts give the \ac{BDT} a more
focused view of the signal and background.

Three preselection cuts are applied, on top of the baseline selection for these
studies given in Table \ref{tab:vzy-bdt-preliminaryselection}. Minimum jet
transverse momentum is included for both leading and sub-leading jets. Each is
set to the highest value that did not degrade the sensitivity of the \ac{BDT}:
$p_T^{j,1} > 40$ GeV and $p_T^{j,2} > 30$ GeV. This reduces the impact of
any systematic uncertainties that behave poorly for low-$p_T$ jets.
A requirement is also placed on the rapidity difference $|\Delta y_{jj}|$.
Artefacts were found in the \ac{BDT} response for background events with
high $|\Delta y_{jj}|$; a cut of $|\Delta y_{jj}| < 2$ was found to remove these
issues and have no impact on sensitivity.

A training cut is made on the dijet mass, $m_{jj}$, to focus on a more
signal-rich region. Applying a training cut of $60 < m_{jj} < 115$ GeV was found
to improve \ac{BDT} performance. Tighter mass window cuts were tested and no
further improvements were found.  This cut is only applied for training the
\ac{BDT}, and not as a preselection cut, to preserve its use for defining
\acp{CR}.

\subsubsection{Hyper-parameter optimisation}
% TODO revisit section after BDT theory written

A \ac{BDT} implementation has hyper-parameters that instruct it on how to build
decision trees during training, as discussed in Section [theory:BDT].
%TODO update reference
\newcommand\ntrees{\ensuremath{N_\text{trees}}\xspace}
\newcommand\ncuts{\ensuremath{N_\text{cuts}}\xspace}
\newcommand\dmax{\ensuremath{d_\text{max}}\xspace}
\newcommand\boostbeta{\ensuremath{\beta}\xspace}
Four hyper-parameters were investigated to optimise performance of the \ac{BDT}
used for this analysis: the number of cuts tested across the range of a variable
when splitting nodes, \ncuts; the number of trees in the forest, \ntrees; the
maximum allowed depth of each decision tree, \dmax; and the \boostbeta
parameter which controls the rate of learning by modifying the boost weights.

Each parameter was tested in turn, training and testing the \ac{BDT} to evaluate
overtraining and sensitivity through the same procedure as in Section
\ref{sec:vzy-bdt-variables}. Values for \ncuts between 2 and 500 were tested and
the greatest sensitivity was achieved with $\ncuts=90$, with no significant
overtraining. Numbers of trees between 300 and 1500 were tested, with optimal
sensitivity obtained for $\ntrees=850$. The \dmax hyper-parameter was tested
for values from 1 to 9 and the sensitivity was found to increase for increasing
\dmax. However, deeper trees also became more prone to overtraining. A value
of $\dmax=3$ was chosen as the best balance between sensitivity and
overtraining. The boost \boostbeta parameter was tested with a range of values
between 0 and 1, $\boostbeta=0.5$ was chosen with the best sensitivity and no
significant overtraining.

\subsubsection{Overall performance}

With all of the optimisations made, the best significance obtained for events
passing a \ac{BDT} cut is 1.5 standard deviations. This represents a sizeable
improvement over the 1.2 standard deviations obtained with the cut-based
approach, and motivates use of the \ac{BDT} in this analysis.

% TODO write BDT section in theory
% Include something on overtraining?
% sources:
%http://dx.doi.org/10.1016/j.nima.2004.12.018
%https://arxiv.org/abs/2206.09645

