% ========== SYSTEMATIC UNCERTAINTIES SECTION ==========

This analysis considers the sources of systematic uncertainty discussed in
Section \ref{sec:methods-systematics}. As well as being applied to the signal
and \QCDZy background, experimental and theory systematic
uncertainties are applied for the \ac{EW} \Zyjj background. Due to the adoption
of more standardised \ac{ATLAS} tools \cite{trexfitter}, a different pruning
procedure is used here to that of the \ac{VBS} analysis, and is discussed below.

Some of the theory uncertainties discussed in Section
\ref{sec:methods-systematics-theory} are omitted from this analysis.
Uncertainties on the signal process from choice of parton showering and
underlying event model are not included, and the uncertainty for interference
between \ac{EW} and \QCDZy production is also missing. These are not
expected to be a significant omission due to the small signal yield.
Uncertainties from choice of scale and \ac{PDF} set are included for the
\QCDZy background and both \ac{EW} \Zyjj samples. The QCD modelling uncertainty is
included and calculated using the difference of generators method.

%Largest uncertainties

\subsection{Pruning}
Given the large number of systematic uncertainties, a pruning procedure is
implemented in order to reduce the number of nuisance parameters necessary in
the fit. The pruning used for this analysis is less detailed than the one used
for the \ac{VBS} \Zy analysis, it does not rely on statistical uncertainties on
the estimates of systematic uncertainties.

Pruning is done individually in each of the four regions used in the fit.
The impact of each uncertainty on the normalisation and shape of the signal
rarity distribution is considered; shape and normalisation impact are decoupled
such that either can be removed if the effect is small. As a result, there are
four outcomes for each systematic uncertainty: it is retained in full with
normalisation and shape effect, its shape effect is dropped but normalisation
kept, its normalisation effect is dropped but shape effect retained, or the
uncertainty is dropped entirely.

If the normalisation effect of an uncertainty is retained in the fit, one
nuisance parameter is included which allows the uncertainty estimate to be
adjusted by the fit, changing the overall normalisation for the associated
background or signal estimate. When the shape effect of a systematic uncertainty
is used in the fit, per-bin nuisance parameters are used which allow the yield
in each bin to be adjusted by the fit; these per-bin parameters are constrained
so as not to affect the overall uncertainty.

The threshold for dropping a normalisation component of an uncertainty is set at
$0.2\%$, i.e.  the normalisation is dropped from the fit if its estimated effect
on the overall normalisation of the sample is less than 0.2\% of the yield. The
threshold for dropping a shape component is set at 99.8\%. In this case there
is a threshold in the probability of the uncertainty having a different shape
to the nominal distribution. The probability is calculated through the
Kolmogorov-Smirnov (KS) test \cite{Kolmogoroff1933,Smirnov1939,Massey1951}; the
$p$-value given by the test represents compatibility between the nominal and
systematic varied distributions. If the $p$-value is greater than 0.998 (99.8\%)
then the differences are considered sufficiently small and the shape component
is dropped.

The results of the pruning are shown in Figure
\ref{fig:vzy-systematics-prune}, where for each sample, region, and background
the treatment of each systematic uncertainty is indicated.

\begin{figure}[p]
  \centering
  \includegraphics[height=.9\textheight]{\resource{Pruning.pdf}}
  \caption{
    Pruning results for systematic uncertainties in the \VZy analysis. The colours indicate
    whether a systematics shape and normalisation uncertainty components were
    each retained for the fit or dropped, for each sample and region used in the
    fit.
  }
  \label{fig:vzy-systematics-prune}
\end{figure}
