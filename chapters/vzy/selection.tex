% ========== VZy Event selection section ==========

Events in data and simulation, from the samples discussed in Section
\ref{sec:methods-samples}, undergo selection to create an analysis region
sensitive to \VZy triboson production. This section discusses an initial
pre-selection applied to samples and the additional requirements used
to define the signal sample. Measures to further refine the analysis region are
discussed in Section \ref{sec:vzy-bdt}.

% Uses Zy selection from anacom

\subsection{Analysis region definition}

A selection is applied to events to impose a loose triboson-like phase space,
before more precise signal and control regions are defined with the help of the
\ac{BDT} discriminant discussed in Section \ref{sec:vzy-bdt}.

The \Zy selection defined in Section \ref{sec:methods-selection} is first
applied to events. Events are then required to have at least two jets, each with
rapidity $|y_j| < 4.4$. The leading jet must have a transverse momentum of at
least 40 GeV, and the sub-leading jet at least 30 GeV. The invariant mass of the
dijet system must satisfy $m_{jj} < 150$ GeV, and the rapidity separation of the
jets $|\Delta y|_{jj} < 2$. These cuts are summarised in Table
\ref{tab:vzy-selection}.

\begin{table}
  \centering
  \renewcommand\arraystretch{1.3}
  \begin{tabular}{p{6em}l}
    \hline\hline
    \multicolumn{2}{c}{\VZy selection} \\
    \hline
    \Zy system & \Zy selection (Table \ref{tab:anacom-zy-selection}) \\
    \hline
    Jet & $N_j \geq 2$ \\
        & $|y_j| < 4.4$ \\
        & $p_T^{j,1} > 40$ GeV \\
        & $p_T^{j,2} > 30$ GeV \\
    \hline
    Dijet system & $m_{jj} < 150$ GeV \\
                 & $|\Delta y_{jj}| < 2$ \\
    \hline\hline
  \end{tabular}
  \caption{
    Summary of event selection criteria defining the \VZy analysis region.
  }
  \label{tab:vzy-selection}
\end{table}

% Justify cut values
Many of these jet variables are later employed by the \ac{BDT} to refine
selection but adding loose preselection reduces complexity at no cost to
performance, see Section \ref{sec:vzy-bdt} for a more detailed discussion.
The $m_{jj}$ cut ensures that this analysis is orthogonal to the \ac{VBS} \Zy
analysis (which uses a cut of $m_{jj} > 150$ GeV), and is also compatible with
the expected signal values of $m_{jj} \sim m_W,m_Z$.

These cuts define the full analysis region, further cuts on the \ac{BDT} output
and $m_{jj}$ are used to define the \ac{SR} and \acp{CR}, discussed in Section
X.  %TODO -- what section are these discussed in?

% Dyjj made BDT cleaner -- discuss in BDT section, doesn't make much sense here.

\subsection{\acs{EW} \VZy definition}
\label{sec:vzy-selection-vzy}

Triboson \VZy production forms a subset of the processes under the umbrella of
\ac{EW} \Zyjj production. Only interactions where the two jets are a product of
a boson decay should contribute to the signal process. Other forms of \ac{EW}
\Zyjj production, such as the diagrams in Figures \ref{fig:vbs-vbsfeynman} and
\ref{fig:vbs-nonvbsfeynman}, should ideally be considered as a source of
background.

\newcommand\mjjtruth{\ensuremath{m_{jj}^\text{truth}}\xspace}
\newcommand\partonid{\texttt{PartonTruthLabelID}\xspace}
This analysis defines two samples as orthogonal subsets of the \ac{EW} \Zyjj
production sample detailed in Section \ref{sec:methods-samples}: \ac{EW} \VZy
(the signal sample) and \ac{EW} \Zyjj background (or the \ac{EW}
background). These samples are separated using truth information.

Two variables are used to test if the jets are products of a W or Z boson decay:
\mjjtruth, the invariant mass of the dijet system calculated at truth level, and
\partonid, which indicates the flavour of the parton initiating each
jet\footnote{
  This variable informs on whether the parton is a quark or a gluon and the
  quark flavour (e.g. up, down, strange, etc.) but does not distinguish between
  a quark and an anti-quark, that information was not available in the sample.
}.
The constraint $74 \leq \mjjtruth \geq 99$ GeV is applied for events included in
the \VZy sample, chosen as it contains 95\% of the combined W and Z boson
lineshape and so should select 95\% of W/Z($\to$jj) events. Events included in
\VZy are also required to have {\partonid} values compatible with quark flavours
from a W or Z decay, i.e. both jets are quark-initiated and either both the same
flavour (e.g. both up quarks as in Z$\to u\bar{u}$) or one up-type and one
down-type quark (e.g. an up and a strange quark as in W$\to u\bar{s}$). Any
events failing either of these cuts are included in the \ac{EW} background
sample.
% 95% of WZ combined lineshape (approx integral gave 0.9406)

This selection is not 100\% efficient and as such there is some
cross-contamination between the samples. Nevertheless, applying this truth
selection increases the probability that any event considered signal contains
the physics processes of interest, direct multiboson interactions.  Of the
events passing the preselection in Table \ref{tab:vzy-selection} for the full
\ac{EW} \Zyjj sample, 31\% are accepted to the \VZy signal sample and the
remaining 69\% make up the \ac{EW} background.
