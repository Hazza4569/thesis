% ========== TEMPLATE FIT SECTION==========

\newcommand\muEW{\ensuremath{\mu_{\text{EW}}}\xspace}
The signal process is measured through a fit to the signal rarity distribution
in the signal and control regions. The signal strength of \ac{EW} \VZy,
\muEW, is the \ac{PoI} in the fit. This parameter follows the
definition in Equation \ref{eqn:vbs-fit-mu}, such that a measured value of
$\muEW = 1$ means that the process is measured to occur at rate expected in the
\ac{SM}.

Estimates for each background are used as templates for the fit, and combine
with a signal estimate taken from \ac{MC} to give the total predicted events
per-bin in signal rarity. The fit adjusts the value of \muEW, as well as the
values of the nuisance parameters representing systematic uncertainties, to best
match these templates to the data yield observed in each bin of the
distribution.

% Binning and MC stats
Four bins are used for the signal rarity distribution in the \ac{SR}. This
binning creates some significant \ac{MC} statistical uncertainties, but provides
a balance between these uncertainties and sensitivity to the signal. The low
\ac{MC} statistics is a side effect of the heavily constrained phase space
necessary to measure such a low-rate process; these could be reduced by creating
larger \ac{MC} samples or samples weighted towards the relevant phase space, but
there was not sufficient time to achieve this.

% This subsection could be moved to theory or anacom?
\subsection{Fitting parameter values}
%Likelihood model
In order to find which values for \muEW and the nuisance parameters give the
best agreement with the data, a likelihood model is built. The likelihood is the
product of several terms, primarily: a Poisson term for each bin in the fit,
describing the probability of obtaining the observed data yield given the
estimates from the templates; and a constraint function for each nuisance
parameter. Given that each nuisance parameter has only an up variation, down
variation, and nominal value, the constraint functions must be interpolated.
This is done using a linear interpolation for shape uncertainties and an
exponential interpolation for normalisation uncertainties; these interpolations
are discussed in Reference \cite{Cranmer2012}.

%Minimisation
Once constructed, the maximum value for the likelihood must be found. This is
achieved by minimising the negative logarithm of the likelihood through the
Davidon-Fletcher-Powell approach \cite{Davidon1959,Fletcher1970,Powell1983}
implemented in Minuit's MIGRAD algorithm \cite{Minuit2}.
The values of parameters that minimise the negative log likelihood are taken as
the fitted values for the \ac{PoI} and nuisance parameters. Uncertainties for
these parameters are given by the covariance matrix calculated during
minimisation. The MINOS technique \cite{Minuit2} is used to obtain a more
accurate estimate of the uncertainties on \muEW.

%explain closure issues
%TODO explain Asimov datasets in theory?
\subsection{Fit closure}

To test the fitting procedure is stable and self-consistent, a fit is performed
using Asimov datasets in all regions. This pseudo-data setup runs the fit with
`data' yields equal to the total expected \ac{MC} yield in all regions. By
construction, this should give a fitted value of $\muEW = 1$; any significant
deviation would indicate a problem with the fit. This Asimov-only fit produces
the result
%
\begin{equation*}
  \muEW = 0.9994 \pm 1.1272,
\end{equation*}
%
deviating only slightly from 1 and thus not indicating any issues.

\subsection{Mixed fit}

In order to estimate the full sensitivity of the analysis without using observed
data in the \ac{SR}, a mixed data-Asimov fit is used: here the observed data
yields are used in the three \acp{CR} and an Asimov dataset in the \ac{SR}. This
Asimov dataset is generated by first performing a fit to data in the \acp{CR}
with the value of \muEW fixed to 1. This allows the values of the systematic
uncertainties to vary and account for any small data-\ac{MC} discrepancies. The
post-fit values for these parameters are then used in the estimate for the
number of events in the \ac{SR} used for the Asimov data.

The results of this fit represent the \ac{SM} expectation for the analysis
results, and thus demonstrate the sensitivity. Full expected results are
presented in Section \ref{sec:vzy-results} alongside the observed results.

Running this mixed fit gives a fitted value for the \muEW parameter of
%
\begin{equation}
  \begin{split}
  \muEW &= 1.599 ^{+1.204}_{-1.149} \\
        &= 1.599 ^{+0.945}_{-0.915} \,(\text{stat.})
                 ^{+0.677}_{-0.632} \,(\text{syst.})
                 ^{+0.313}_{-0.288} \,(\text{MC stat.})
  \end{split}
  \label{eqn:vzy-fit-mixed-muEW}
\end{equation}
% mu_EWK  1.59905 +1.20357 -1.14875 (full)
% mu_EWK  1.58123 +0.995001 -0.958983 (stat+MCstat only)
% mu_EWK  1.58928 +0.944594 -0.914811 (stat only)
% mu_EWK  1.58896 +/- 0.92973  
%
where the component of the error from \ac{MC} statistics has been factored out
of the systematic uncertainty (for this instance only). This result shows an
apparent deviation from the expected \ac{SM} value of $\muEW = 1$, despite not
including data in the \ac{SR}. This bias may be introduced by the large \ac{MC}
statistical uncertainties, with their contribution to the total uncertainty
indicated in Equation \ref{eqn:vzy-fit-mixed-muEW}.
It may be possible to correct for this effect, but there was not sufficient time
to gain a sufficient understanding as to be able to assign an uncertainty to
this correction.
For the purposes of this analysis, this \muEW value is considered an upward
fluctuation of the \ac{MC} statistical error (compatible at the $\sim2\sigma$
level) and thus consistent with the expectation.


\subsection{Data fit}

Once the \ac{SR} is unblinded, the fit can be performed using observed data
yields in all four regions. As with the mixed fit, the \muEW value and all
nuisance parameters are minimised simultaneously across all regions, allowing
their values to be constrained by data in \acp{CR} as well as the \ac{SR}.
Results from this fit are presented in Section \ref{sec:vzy-results}.

