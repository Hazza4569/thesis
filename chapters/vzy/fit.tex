% ========== TEMPLATE FIT SECTION==========

\newcommand\muEW{\ensuremath{\mu_{\text{EW}}}\xspace}
The signal process is measured through a fit to the signal rarity distribution
in the signal and control regions. The signal strength of \ac{EW} \VZy,
\muEW, is the \ac{PoI} in the fit. This parameter follows the
definition in Equation \ref{eqn:vbs-fit-mu}, such that a measured value of
$\muEW = 1$ means that the process is measured to occur at rate expected in the
\ac{SM}.

Estimates for each background are given as templates to the fit, and combine
with a signal estimate taken from \ac{MC} to give the total predicted events
per-bin in signal rarity. The fit adjusts the value of \muEW, as well as the
values of the nuisance parameters representing systematic uncertainties, to best
match these templates to the data yield observed in each bin of the
distribution. This is achieved through the likelihood construction and
maximisation techniques discussed in Section \ref{sec:methods-stats}.

% Binning and MC stats
Four bins are used for the signal rarity distribution in the \ac{SR}. This
binning creates some significant \ac{MC} statistical uncertainties, but provides
a balance between these uncertainties and sensitivity to the signal. The low
\ac{MC} statistics is a side effect of the heavily constrained phase space
necessary to measure such a low-rate process.
%though this could be countered by
%creating larger \ac{MC} samples or samples weighted towards the relevant phase
%space.

\subsection{Fit closure}

To test whether the fitting procedure is stable and self-consistent, a fit is
performed using 'pseudo-data` in all regions. This pseudo-data setup runs the
fit with `data' yields equal to the total expected \ac{MC} yield in all regions.
By construction, this should give a fitted value of $\muEW = 1$; any significant
deviation would indicate a problem with the fit. This gives a signal strength
and expected data uncertainty of
%
\begin{equation*}
  \muEW = 1.00 \pm 1.13,
  %\muEW = 0.9994 \pm 1.1272,
\end{equation*}
%
indicating a healthy fit.

\subsection{Mixed fit}
\label{sec:vzy-fit-mixed}

In order to estimate the full sensitivity of the analysis without using observed
data in the \ac{SR}, a mixed data--pseudo-data fit is used: here the observed data
yields are used in the three \acp{CR} and pseudo-data in the \ac{SR}. These
pseudo-data are generated by first performing a fit to data in the \acp{CR}
with the value of \muEW fixed to 1. This allows the values of the systematic
uncertainties to vary and account for any small data-\ac{MC} discrepancies. The
post-fit values for these parameters are then used in the estimate for the
number of events in the \ac{SR} used to generate the pseudo-data.

The results of this fit represent the \ac{SM} expectation for the analysis
results, and thus demonstrate the sensitivity. Full expected results are
presented in Section \ref{sec:vzy-results} alongside the observed results.

Running this mixed fit gives a fitted value for the \muEW parameter of
%
\begin{equation}
  \begin{split}
  \muEW &= 1.60 ^{+1.20}_{-1.15} \\
        &= 1.60 ^{+0.95}_{-0.92} \,(\text{stat.})
                ^{+0.68}_{-0.63} \,(\text{syst.})
                ^{+0.31}_{-0.29} \,(\text{MC stat.})
  %\muEW &= 1.599 ^{+1.204}_{-1.149} \\
  %      &= 1.599 ^{+0.945}_{-0.915} \,(\text{stat.})
  %               ^{+0.677}_{-0.632} \,(\text{syst.})
  %               ^{+0.313}_{-0.288} \,(\text{MC stat.})
  \end{split}
  \label{eqn:vzy-fit-mixed-muEW}
\end{equation}
% mu_EWK  1.59905 +1.20357 -1.14875 (full)
% mu_EWK  1.58123 +0.995001 -0.958983 (stat+MCstat only)
% mu_EWK  1.58928 +0.944594 -0.914811 (stat only)
% mu_EWK  1.58896 +/- 0.92973  
%
where the component of the error from \ac{MC} statistics has been factored out
of the systematic uncertainty (for this instance only).
This gives a signal strength that appears to be greater than one,
%apparent deviation from the expected \ac{SM} value of $\muEW = 1$,
despite not
including data in the \ac{SR}. This may be a bias introduced by the large \ac{MC}
statistical uncertainties, with their contribution to the total uncertainty
indicated in Equation \ref{eqn:vzy-fit-mixed-muEW}.
However, since this \muEW value is consistent with one, at the $\sim2\sigma$
level considering the \ac{MC}-statistics error, the effect is not significant.

\subsection{Data fit}

Once the \ac{SR} is unblinded, the fit can be performed using observed data
yields in all four regions. As with the mixed fit, the \muEW value and all
nuisance parameters are minimised simultaneously across all regions, allowing
their values to be constrained by data in \acp{CR} as well as the \ac{SR}.
Results from this fit are presented in Section \ref{sec:vzy-results}.

