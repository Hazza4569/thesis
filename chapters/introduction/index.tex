Elementary particle physics is the study of the mechanics of nature at the most
fundamental scale. The field is guided by a theory, the Standard Model
(\acsfirst{SM}), describing a handful of elementary particles that account for almost
the entirety of known matter and interactions in the universe. At the same time,
experiments in the field have been growing larger and larger over recent
decades, in attempts to create the higher energy environments needed to probe
the incredibly small distance scales on which these particles operate.

This thesis presents a number of contributions to the upgrade and research
programmes of, to date, the largest particle physics experiment in history, the
\ac{ATLAS} experiment. The focus of this work is on analysing rare electroweak
processes in the \ac{SM} sensitive to multiboson interactions, an established
feature of the theory that occur at such low rates in experiment that make them
difficult to observe.

Two analyses are presented in this thesis to probe these interactions, and both
analyses rely on the same final state: Z($\to$ll)$\gamma$jj. The first analysis
interprets the pair of jets as an artefact of a \ac{VBS} event, featuring a
$2\to2$ scattering between electroweak bosons. The second analysis treats the
jets as products of the hadronic decay of a third boson, either a W or another Z
boson, to create a triboson final state. These processes lead to two analyses
with some convenient overlap in methodology, but a unique set of challenges for
each.

Meanwhile, a separate body of work is presented on studies and tools made to
help with the upgrade programme for the \ac{L1Calo} trigger. This includes a
tool to visualise trigger algorithms and help to debug discrepancies between
software and firmware codebases, a study of early Run-3 data to analyse
performance of the trigger in commissioning, and development of a software
prototype of a future firmware algorithm to improve trigger performance in
future runs.

The reader will be introduced to some of the necessary concepts from theory in
Chapter \ref{sec:theory} and to the experimental setting, the collider and
detector, in Chapter \ref{sec:detector}. Chapter \ref{sec:trig} presents the
work on \ac{L1Calo} trigger upgrades, with some further information about the
context and timeline for the upgrades. The \ac{VBS} \Zy analysis is presented in
Chapter \ref{sec:vbs} and the semileptonic \VZy analysis in Chapter
\ref{sec:vzy}.  Meanwhile, some of the shared methods for the two analyses and
some additional background is given before this in Chapter \ref{sec:methods}.
