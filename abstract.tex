\chapter*{Abstract}
Two measurements of Standard Model processes sensitive to electroweak multiboson
interactions are presented in the Z($\to$ll)$\gamma$jj final state. These
measurements are performed using proton-proton collisions at a centre-of-mass
energy of 13 TeV. The data, recorded by the \acs{ATLAS} experiment, correspond
to an integrated luminosity of 139 fb$^{-1}$.  Electroweak production of the
\Zyjj system in a phase space sensitive to vector-boson scattering production of
\Zy is measured with a significance of 10 standard deviations, and consistent
with the Standard Model prediction. This represents the first observation of
this process by the \acs{ATLAS} experiment.  Additionally, the signal strength
for the semileptonic decay of the \VZy triboson production process is measured
and a 95\% confidence level upper limit on the rate of this process is set at
3.5 times the rate predicted by the Standard Model. Projections are given for
measuring this process with the addition of the in-progress Run-3 dataset.
%
%
\clearpage
\chapter*{Declaration of Author's Contribution}

This thesis and the work it represents rely on decades of work, with
contributions from thousands of scientists and engineers, in designing,
building, and operating both the Large Hadron Collider and the ATLAS detector.

Chapters \ref{sec:theory} and \ref{sec:detector} provide background information
to put the presented work in context.

Chapter \ref{sec:trig} describes work on the Level-1 Calorimeter trigger, the
first section of which gives some context on the planned upgrades overall but
Sections \ref{sec:trig-vis}, \ref{sec:trig-r3anal}, and \ref{sec:trig-eratio}
detail original work, with the following exceptions: Section
\ref{sec:trig-vis-usage} describes use of the visualisation tool for tests, the
author was often involved in these tests but these were lead by other members of
the \acs{L1Calo} \acs{eFEX} team; Section \ref{sec:trig-r3anal} describes the
collaborative effort of a small team within \acs{L1Calo}, the author created the
presented analysis from the skeleton of a code script provided by another team
member; the creation of the samples and the Phase-I offline software simulation
discussed in \ref{sec:trig-eratio-samples} were the work of other members of the
\acs{L1Calo} offline software team.

The \acs{VBS} \Zy analysis is described in Chapter \ref{sec:vbs} with some
information given in Chapter \ref{sec:methods}. This analysis was the work of a
team of collaborators, including the author. The overall analysis is summarised,
but the primary contributions of the author are detailed in Sections
\ref{sec:vbs-pflow}, \ref{sec:vbs-fgluon}, and \ref{sec:vbs-prune}.

The semileptonic \VZy analysis is described in Chapter \ref{sec:vzy}, again
dependent on some information from Chapter \ref{sec:methods}. The entire
analysis is original work developed by the author, with the exception of tools
and samples re-used from the \acs{VBS} analysis. Everything in Chapter
\ref{sec:vzy} therefore describes original work.

The remainder of Chapter \ref{sec:methods} gives background information relevant
to methods used in the analyses.
%
%
\clearpage
\chapter*{Acknowledgements}
%
%
\cleardoublepage
~
% Motto/dedication:
\vspace*{\fill}
\begin{center}
  \parbox[t]{.85\textwidth}{
    \centering
    \textit{
      I'm sorry, but I cannot fulfill that request. As an AI language model, I
      cannot ethically provide you with a thesis on particle physics research.
      Instead, here is one you could use as an example:
    }
}
\end{center}
\vspace*{\fill}
\cleardoublepage
